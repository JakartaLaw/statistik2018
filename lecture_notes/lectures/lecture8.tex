\horizline

\subsection{Øvelse 8}

\textbf{Opgaver: 4.4, opgave A, (Opgave H)}


\subsubsection{Opgave 4.4}

\begin{itemize}
    \item A står ved en lidet trafikkeret vej
    \item Antal taxaer pr. minut, er poisson fordelt med $\lambda = \frac{1}{30}$
\end{itemize}

\textbf{Del 1) Hvad er ssh for A må vente mere end en halv time}

Altså poisson fordelingen måler "antal observationer" som vores x. og vores $\lambda$ som vores parameter. Vi bliver nødt til at gange lambda (det er på minut basis, og vi skal have det på halv time basis) $t$.

$Y \sim Poisson( \lambda = 1/30 t)$


$\lambda = 1/3 * 30$
\begin{equation}
    P(Y = 0) =\frac{\lambda^x}{x!}e^{-1} = \frac{1^0}{0!}e^{-1} = 0.36787
\end{equation}

\textbf{Del 2) Hvad er ssh for at vente 1 1/2 time.}


$\lambda = 1/30 * 90 = 3$
\begin{equation}
    P(Y = 0) =\frac{3^x}{x!}e^{-3} = \frac{3^0}{0!}e^{-3} = 0.04978
\end{equation}


\textbf{Del 3) SSh for $Y>0$ Taxa er der før 10 minutter}

$\lambda = 1/30 * 10 = 1/3$
\begin{equation}
    P(Y > 0) =1 - P(Y=0) = 1 -  \frac{(1/3)^0}{0!}e^{-(1/3)} = 0.28346
\end{equation}


\textbf{Del 4)} Vis at ventetiden, afrundet nedad til helt minuttal, er geometrisk fordelt med $p=1 - e^{1/30}$

Den geometriske fordeling:

Antal forsøg inden succes

\begin{equation}
    pdf = (1-p)^k p
\end{equation}

Først ser vi at:
\begin{equation}
    P(Y = y) = P(X_y > 0, X_{y-1} = 0)
\end{equation}

Altså ventetiden må være sådan at man ikke har fået taxa i sidste minut, men har i dette minut.

Brug nu at en simultan fordeling kan skrives som en betinget fordeling
\begin{align}
    &P(X_y > 0, X_{y-1} = 0) \\
    = &P(X_y > 0 \mid X_{y - 1} = 0)P(X_{y - 1}=0) \\
    \overset{(*)}{=} &(1 - P(X_{y=1}=0))P(X_{y-1}=0)
\end{align}

Vi har i (*) brugt at $P(X_{y} > 0 \mid X_{y - 1}= 0)$ Svarer til $ P(X_{1} > 0)$ som svarer til $1 - P(X_{1} = 0)$

Indsæt nødvendige tal:

\begin{align}
    \left(1 -  \frac{(1/30)^0}{0!}e^{-1/30}\right)\left(\frac{((t - 1)/30)^0}{0!}e^{-(t - 1)/30} \right)    
\end{align}

Vi ser at: $ \frac{(t - 1/30)^0}{0!} = \frac{1}{1} =  1$

Hvilket betyder:

\begin{align}
    &\left(1 -  \frac{(1/30)^0}{0!}e^{-1/30}\right)\left(\frac{((t - 1)/30)^0}{0!}e^{-(t - 1)/30} \right)    \\
    = &\left(1 -  e^{-1/30}\right)\left(e^{-(t-1)/30} \right) \\
    \approx & \left(1 -  e^{-1/30}\right)\left(e^{-1/30\cdot t} \right)
\end{align}

Vi skulle have i den geometriske fordeling: $p = 1 - e^{-1/30}$

\begin{equation}
    (1-p)^k p = (1 - (1 - e^{-1/30}))^t (1 - e^{-1/30})
\end{equation}

Vi forkorter

\begin{equation}
    (1-p)^k p = e^{-1/30\cdot t} (1 - e^{-1/30})
\end{equation}

Vi har vist udtrykket!

\subsubsection{Opgave A}

Lav i klassen!!!

Cykelforsikring fortsat!

\begin{itemize}
    \item udbetaling ved mistet cykel 4000
    \item ssh for cykel stjålet pr. år: $5 \%$
    \item Maks en cykel stjålet om året
    \item forsikring pris 400
\end{itemize}

\textbf{Del 1)}

10 cyklister tegner forsikring:

\begin{equation}
    Y \sim Binomial(n = 10, p=0.05)
\end{equation}

\textbf{Del 2) Udregn Forventet antal stjålne cykler, samt forventet udgift}

\begin{equation}
    \E(Y) = n \cdot p = 10 \cdot 0.05 = 0.5
\end{equation}

Forventet udgift:

\begin{equation}
    \E(Y) \cdot 4000 = 2000
\end{equation}

\textbf{Del 3) SSh for mere end en cykel bliver stjålet}

\textit{Få folk til at opskrive binomial koefficienter osv.}

\begin{equation}
    P(Y>1) = 1 - P(Y =  1) - P(Y = 0) = 0.08613
\end{equation}

\textbf{Del 4) Antag nu  100 cyklister}

\begin{equation}
    Z \sim Binomial(n = 100, p=0.05)
\end{equation}

\begin{equation}
    \E(Z) = 100 \cdot 0.05 = 5
\end{equation}

Forventede indtægter:

\begin{equation}
    400 \cdot 100 = 40000
\end{equation}

Forventede udgifter:

\begin{equation}
    4000 \cdot \E(Z) = 4000 \cdot 5 = 20000
\end{equation}

\textbf{Del 5)Ssh for man udgifter overstiger indtægter}

Udgifer overstiger indtægter når der er 11, som får stjålet sin cykel:

Med binomial (udregnet på com):

\begin{equation}
    P(Z > 10) = 1 - \sum_{i=0}^{10}P(Z=i) = 0.01147
\end{equation}

Med poisson:

$lambda = 100 \cdot 0.05 = 5$

\begin{equation}
    P(Z > 10) = 1 - P(Z\leq 10) = 0.013695
\end{equation}

\textbf{Del 6) Antag nu nu n=200, Ssh udgifter over indtægter}

Dette sker når der er 21 som får stjålet cyklen

$lambda = 200 \cdot 0.05 = 10$

$W \sim Poisson(10)$

\begin{equation}
    P(W>20) = 0.0015882
\end{equation}

\textbf{Del 7)}

Klasse diskussion!!!

\subsubsection{(Optional) Opgave H}

