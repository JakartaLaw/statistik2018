\horizline

\subsection{Øvelse 6}

\textbf{28/09/2018 - C.4 \& Opgave 1}

\subsubsection{Opgave C.4}

\begin{itemize}
    \item Poisson distribution
    \item Antal opkald kan modelleres med en stokastisk variabel kaldet $X:= Poisson(\lambda)$.
\end{itemize}

Om Poisson fordelingen:
En ventetidsfordeling! Citat wikipedia:

\begin{displayquote}
"[Poisson fordelingen] is a discrete probability distribution that expresses the probability of a given number of events occurring in a fixed interval of time or space if these events occur with a known constant rate and independently of the time since the last event." - Wikipedia
\end{displayquote}

Den har den egenskab at:
$\E(X) = \Var(X) = \lambda$

Om denne fordeling kan vi sige at sandsynligheden for et givent udfald er (pdf):

\begin{equation}
    p(x) = \frac{\lambda^x}{x!}e^{-\lambda}
\end{equation}

Lad dem regne selv

\textbf{Ssh for præcis 7}

\begin{equation}
    p(7) = \frac{7^{10}}{7!}e^{-10} = 0.090079 
\end{equation}

\textbf{Ssh for max 7 opkald}

\begin{equation}
    P(X \leq 7) = \sum_{i=0}^7 \frac{i^{10}}{i!}e^{-10} = 0.22022
\end{equation}

\begin{equation}
    P(3 \leq X \leq 7) = \sum_{i=3}^7 \frac{i^{10}}{i!}e^{-10} = \sum_{i=0}^7 \frac{i^{10}}{i!}e^{-10} - \sum_{i=0}^2 \frac{i^{10}}{i!}e^{-10} 
\end{equation}

Indsæt værdier udregnet i python

\begin{equation}
    0.22022 - 0.002769 = 0.217451
\end{equation}

\subsubsection{Opgave 1}

\begin{itemize}
    \item Værdi af cykel 4000 kr
    \item ssh for den bliver stjålet 5 \%
    \item man kan tegne en cykel så den bliver erstattet for hele dens værdi
\end{itemize}

\textbf{Del 1) Hvor meget er man villig til at betale for en sådan forsikring?}

Spørg klassen - Intet rigtigt svar?

\textbf{del 2) Udregn værdi af cykel (på et år)}

Vi definerer X stokastiske variable:

$X:= \textbf{Cykel værdi}$

$P(X=0) = 0.05$ og $P(X=4000) = 0.95$


Så ganger vi værdien på $X$ bagefter.

\begin{equation}
    \E(X) = 0.95 \cdot 4000 = 3800
\end{equation}

\textbf{del 3) Cykel forsikring!}

\begin{equation}
    Y:= \textbf{Værdi af cykel minus forsikring 1}
\end{equation}

\begin{align}
    (Y\mid X=0) &= 0 - 400 + 4000 = 3600\\ (Y \mid X=4000) &= 4000 - 400 = 3600
\end{align}

\begin{equation}
    \E(Y) = 0.95 \cdot 3600 + 0.05 \cdot 3600 = 3600
\end{equation}

\textbf{Del 4) Forsikring med selvrisiko på 1000 kr!}

pris = 150 årligt, selvrisiko = 1000.

$Z := \textbf{Værdi af cykel minus forsikring 2}$

\begin{align}
    (Z \mid X = 0) &= 0 - 150 - 1000 + 4000 = 2850 \\
    (Z \mid X = 4000) &= 4000 - 150 = 3850
\end{align}

\begin{equation}
    \E(Z) = 0.05\cdot 2850 + \cdot 0.95 \cdot 3850 = 3800
\end{equation}

\textbf{Del 5) Sammenlign middel værdier}

Klassediskussion

\textbf{Del 6) Nytte af af $X$, $Y$, $Z$}

nyttefunktion:
\begin{equation}
    u(v) = 10 v - 0.001 v^2, \qquad v \in \{0, 1, \cdots 4000 \}
\end{equation}

Transformér de enkelte stokastiske variable først! X:

\begin{align}
    u(X \mid X = 0) &= 0  \\
    u(X \mid X = 1) &= 10\cdot 4000 - 0.001 \cdot 4000^2 = 24000
\end{align}

transformation af Y:
\begin{align}
    u(Y \mid Y = 3600) &= 10 \cdot 3600 - 0.001 \cdot 3600^2 = 23040\\
\end{align}
    
Transformation af Z:
\begin{align}
    u(Z \mid Z = 3850) = 10 \cdot 3850 - 0.001 \cdot 3850^2 = 23677.5\\
    u(Z \mid Z = 2850) = 10 \cdot 2850 - 0.001 \cdot 2850^2 = 20377.5  
\end{align}

\begin{equation}
    \E(u(X)) = 0.95 \cdot 24000 + 0.05 \cdot 0 = 22800
\end{equation}

\begin{equation}
    \E(u(Y)) = 0.95 \cdot 23040 +  0.05 \cdot 23040 = 23040
\end{equation}

\begin{equation}
    \E(u(Z)) = 0.95 \cdot 23677.5 + 0.05 \cdot  20377.5  = 23512.5
\end{equation}

\textbf{Del 7) Vis generelt udtryk for den forventede værdi af $u(W)$}

\begin{equation}
    u(v) = 10 v - 0.001 v^2, \qquad v \in \{0, 1, \cdots 4000 \}
\end{equation}

lad $W$ være koncentreret på mængden $T$:
\begin{equation}
    \E(u(W)) = \sum_{w \in T} (10\cdot w - 0.001 w^2)p(w)
\end{equation}

\begin{equation}
    \E(u(W)) = \sum_{w \in T} (10\cdot w)p(w) - \sum_{w \in T} (0.001 w^2)p(w)
\end{equation}


\begin{equation}
    \E(u(W)) = 10\cdot\sum_{w \in T} ( w)p(w) - 0.001\cdot \sum_{w \in T} ( w^2)p(w)
\end{equation}

\begin{equation}
    \E(u(W)) = 10\E(W) - 0.001\cdot \E(W^2)
\end{equation}

Vi ved at:
\begin{equation}
    \Var(X) = \E(X^2) - (\E(X))^2 \implies \Var(X) + (\E(X))^2 = \E(X^2)
\end{equation}

Vi bruger dette:

\begin{equation}
    \E(u(W)) = 10\cdot\E(W) - 0.001\cdot (\E(W))^2 + \Var(W)
\end{equation}

Som var det ønskede udtryk

\textbf{Del 8) Udregn variansen af $X$, $Y$, $Z$}

Vi bruger formlen for den varians:

\begin{equation}
    \sum_{x \in T}(x - \E(X))^2p(x)
\end{equation}

Varians af X
\begin{equation}
    0.95 \cdot (3800 - 4000)^2 + 0.05 \cdot (0 - 4000)^2 = 760000
\end{equation}

Varians af Y: Den er $\Var(Y)=0$. Vi får altid udbetalt det samme! \textbf{Definition 3.7.13}

Varians af Z
\begin{equation}
    0.95 \cdot (3850 -3800)^2 + 0.05 \cdot (2850 - 3800)^2 = 47500 
\end{equation}