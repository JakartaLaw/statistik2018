\horizline

\subsection{Øvelse 17}

\textbf{22/10/2018, opgaver: 1, 2, 3}

\subsubsection{Opgave 1}

Kig do file do\_1\_17

\subsubsection{Opgave 2}

\textbf{Del 1}

Figur A: Histogram (og kernel density estimation). Viser tæthedsfunktionen (eller en approksimation).

Figur B: Viser den empiriske CDF.

Figur C: Q-Q plot er et plot der viser der modholder den empiriske distribution med en parametrisk - i dette tilfælde den gaussiske distribution. Dette er gjort ved at sammenligner quantiler.

Figur D: Boxplot - giver indblik i antal outliers samt hvordan de kvartiler, og median er fordelt.

\textbf{Del 2}

A) Medianen er den observation som er svarer til det punkt hvor $F(x) = 0.5$. Altså med andre ord $F^{-1}(0.5) =\text{median} $

B) Kig boxplot. Ja det synes der at være. Det er dog altid svært at vurdere outliers.

C) Ja, vi ser at fordelingen er centreret omkring en middelværdi og er stort set symmetrisk og unimodal.

D) Kig CDF. Omkring halvdelen af alle firmaerne.

C) 10\% fraktilen (hvilket kaldes 10\%  percentilen). Angiver det punkt hvor $F(x)= 0.1 $. I vores konkrete tilfælde cirka $- 500$

E) Vi kan se på Q-Q plottet at distribution er lidt lang i halerne, men det er ikke klart om den er venstre eller højre skæv. Derudover er det ikke klart om disse afvigelser i halerne er nok, til at antage den skulle være venstre skæv eller højre skæv.

\subsubsection{Opgave 3}

\textbf{Del 1}

$gns\_gym$: Man finder at gennemsnittet fra gymnasiet er kontinuær.

$studietimer$: Man finder at antal timer brugt på studie er tælle data.




