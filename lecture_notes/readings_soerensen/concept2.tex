\textbf{readings:} Sørensen 1.4-1.5

\subsection{Betingede sandsynligheder og uafhængighed}

\subsubsection{Den betingede sandsynlighed af $B$ giver $A$ skrevet $P(B\mid A)$, er defineret ved (definition 1.4.1}

\begin{equation}
    P(B\mid A) = \frac{P(A\cap B)}{P(A)}
\end{equation}

\subsubsection{Regneregler for betingede sandsynligheder}

$A_1, A_2, \cdots A_n$ være n hændelser, hvor $P(A_1 \cap A_2 \cap \cdots A_{n-1}>0$. Da:
\begin{align}
    &P(A_1 \cap A_2 \cap \cdots A_n) = \\ &P(A_{1})P(A_2\mid A_1)P(A_3\mid A_1 \cap A_2) \cdots P(A_n \mid A_1 \cap \cdots \cap A_{n-1})
\end{align}

\textbf{Endnu en regneregel}

Hvis $A_1, A_2, \cdots A_n$ er $n$ disjunkte hændelser, hvor at $E = \bigcup_{i=1}^n A_i$ samt $P(A_i)>0$, da gælder for en vilkærlig hændelse $B$:

\begin{equation}
    P(B) = \sum_{j=1}^n P(B\mid A_j)P(A_j)
\end{equation}

\subsubsection{Omvendingsformel - simpel bayes'}

\begin{equation}
    P(A\mid B) = P(B \mid A) \frac{P(A)}{P(B)}
\end{equation}

\subsubsection{Bayes' formel}

$A_1, A_2, \cdots A_n$ er $n$ disjunkte hændelser, hvor at $E = \bigcup_{i=1}^n A_i$ samt $P(A_i)>0$. For en hændelse $B$ med $P(B)>0$, da gælder for en enhver hændelse $k$:

\begin{equation}
    P(A_k \mid B) = \frac{P(B\mid A_k) P(A_k)}{\sum_{j=1}^n P(B \mid A_j) P(A_j)}
\end{equation}

\subsection{Stokastisk Uafhængighed}

\textbf{Uafhængighed} tænkes oftest som:

\begin{equation}
    P(A \mid B) = P(A)
\end{equation}

Altså at sandsynligheden for $A$ ikke er påvirket af udfaldet af $B$. 

\subsubsection{Definition af uafhængighed}

hændelse $A$ og $B$ er uafhængige siges at være stokastisk uafhængige når (definition 1.5.1) :

\begin{equation}
    P(A\cap B) = P(A) \cdot P(B)
\end{equation}

Dette udsagn kan let udvides til $n$ hændelser (se p.34 \textbf{definition 1.5.4})

\subsubsection{Regler for indbyrdes uafhængighed}

Tegn for uafhængighed $\independent$.
\newline

$A$, $B$ og $C$ er indbyrdes uafhængige hændelser. Følgende gælder:

\begin{enumerate}
    \item $A\setminus B \independent C$
    \item $A \cap B \independent C$
    \item $A \cup B \independent C$
    \item $E\setminus A, B \independent C$
\end{enumerate}

\subsubsection{forenings mængdens uafhængighed}

$A$, $B$, $C$ er hændelser. $A$ og $B$ er betinget afhængige givet $C$ hvis:

\begin{equation}
    P(A \cap B \mid C) = P(A\mid C) \cdot P(B\mid C)
\end{equation}

denne kan generaliseres (se p. 37 \textbf{definition 1.5.7})

\horizline