\subsection{Grænseresultater for stokastiske variable}

\textbf{readings:}  Sørensen 7

\subsubsection{Store tals lov}

For $n$ ukorrelerede stokastiske variable med middelværdi $\mu$ og varians $\sigma^2$ vil gennemsnittet (det empiriske) være tæt på middelværdien $\mu$.

Altså gennemsnittet vil konvergere med middelværdien. Gennemsnittet er stadig en stokastisk variabel

\subsubsection{Den centrale grænseværdi sætning}

Her er det værd at notere:

\begin{equation}
    \E(\bar{X}_n) = \mu
\end{equation}

Og måske mindre klart:

\begin{equation}
    \Var(\bar{X}_n) = \frac{1}{n^2} ( \Var(X_1) + \Var(X_2) + \cdots + \Var(X_n)) = \frac{1}{n^2}(n \Var(X_i)) = \frac{1}{n^2} n \sigma^2 = \frac{\sigma^2}{n}
\end{equation}

Man altså se at vi kan standardisere den stokastiske variabel som er middelværdien:

\begin{equation}
    U_n = \frac{\bar{X}_n - \mu}{\sigma / \sqrt{n}}
\end{equation}

Nu er reskaberne til sætningen klar:

En række af stokastiske uafhængige varible, som er identiske. Da vil fordelingen af $U_n$ konvergere mod en standard normalfordeling når $n\rightarrow \infty$

Dette kaldes konvergens i fordeling
