\documentclass[a4paper,12pt]{article}
\usepackage[utf8]{inputenc}

%usepackages
\usepackage{amsmath}
\usepackage{amssymb}
\usepackage{bbm}
\usepackage{booktabs}
\usepackage{csquotes}
\usepackage[parfill]{parskip} % lineskip when new paragraph


%new commands
\newcommand{\eps}{\mathcal{E}}
\newcommand{\indic}{\mathbbm{1}}
\newcommand{\horizline}{ \noindent\makebox[\linewidth]{\rule{\paperwidth}{0.4pt}} }
\newcommand\independent{\protect\mathpalette{\protect\independenT}{\perp}}
\def\independenT#1#2{\mathrel{\rlap{$#1#2$}\mkern2mu{#1#2}}}
\newcommand{\E}{\text{E}}
\newcommand{\Var}{\text{Var}}

\newcommand{\Cov}{\text{Cov}}
\newcommand{\Corr}{\text{Corr}}

\newcommand{\N}{\mathbb{N}}
\newcommand{\R}{\mathbb{R}}
\newcommand{\1}{\mathbbm{1}}
\newcommand{\Y}{\mathbb{Y}}

\newcommand{\for}{\text{for }}

\newcommand{\ra}{\rightarrow}
\newcommand{\la}{\leftarrow}
\newcommand{\lra}{\Leftrightarrow}

\newcommand{\lp}{\left(}
\newcommand{\rp}{\right)}
\newcommand{\lsp}{\left[}
\newcommand{\rsp}{\right]}
\newcommand{\lcp}{\left\{ }
\newcommand{\rcp}{\right\} }

\newcommand{\indefint}{\int_{-\infty}^{\infty}}

\newcommand{\ellers}{\text{ellers}}

\newcommand{\prodn}{\prod_{i=1}^{n}}
\newcommand{\sumn}{\sum_{i=1}^{n}}


%info
\title{Lecture Notes}
\author{Jeppe Johansen}
\makeindex
\begin{document}


\tableofcontents

\section{Disclaimer}

Disse noter blev udarbejdet i forbindelse med jeg underviste i kurset \textbf{Sandsynlighedsteori og statistik} udbudt af Økonomisk Institut, Københavns Universitet.

Dette er ikke blevet gennemlæst, rettet eller på anden måde redigeret af en tredje person, som ville kunne fange evt. fejl og mangler. Derfor \textbf{forvent} at der er fejl i dette dokument. Forhold dig kritisk til resultaterne, og hvis du er sikker på der er en fejl, så tag udgangspunkt i det.

Dokumentet indeholder rettevejledninger til øvelsesseddlerne forbundet med faget. Der er et tilhørende github-repository: 


\textit{https://github.com/JakartaLaw/statistik2018}.



\maketitle

\section{Topics}

% Sørens Concepts
%\iffalse


\textbf{readings:} Sørensen 1.2-1.3

\subsection{Det endelige sandsynlighedsfelt}

Et endelig sandsynlighedsfelt har følgende egenskaber:

\begin{itemize}
    \item En endelig mængde $E = \{e_1, e_2, \ldots, e_j\}$
    \item En funktion $p$ fra $E$ ind i intervallet $[0,1]$
    \item Summen af samtlige sandsynligheder skal være 1:
    \begin{equation}
    \sum_{j=1}^{k}p(e_j) = 1
    \end{equation}
\end{itemize}  

\subsubsection{Hændelser}

A  er en hændelse:
\begin{equation}
	A \subseteq E
\end{equation}

Sandsynligheden for A:

\begin{equation}
    P(A) = \sum_{x\in A} p(x)
\end{equation}

Sandsynligheden for to disjunkte mængder $A$, $B$:

\begin{equation}
    P(A \cup B) = P(A) + P(B), \qquad P(A) \cap P(B) = \emptyset
\end{equation}

\subsection{Ved konstant ssh funktion}

\begin{equation}
    \text{ssh. for hændelse} = \frac{\text{\# gunstige udfald}}{\text{\#  mulige udfald}}
\end{equation}


\subsubsection{Termer}

\begin{itemize}
    \item \textbf{Udfald} de enkelte elementer i $E$
    \item \textbf{Hændelse} en delmængde
    \item \textbf{Sandsynlighedsfunktionen} er $p(\cdot)$
    \item \textbf{punktsandsynligheden} for $e_j$ er $p(e_j)$
    \item \textbf{disjunkte} er to mængde som har den tomme mængde som fællesmængde
    \item \textbf{Sandsynlighedsmål} er funktionen $P$ fra klassen af delmængder af E. (har ekstra krav, se p. 14 i \textbf{Sørensen})
    \item \textbf{\#} antal elementer i et sæt
\end{itemize}

\subsection{Det generelle sandsynlighedsfelt}

Hvis mængden er uendelig stor, (både tællelig og utællelig) kigger man på delintervaller af $E$.

Under antagelse af ligefordeling:

\begin{equation}
    P(I) = c\lvert I \rvert
\end{equation}

$\lvert I \rvert$ betegner længden af linjestykket på den reelle akse.

\subsubsection{Definition af sandsynlighedsfelt}

\begin{itemize}
    \item Et udfaldsrum $E$ 
    \item En klasse $\eps$ af delmængder fra $E$
    \item En funktion $P$ fra $\eps$ ind i $[0,1]$
    \item $\eps$ skal indeholde både $E$ og $\emptyset$
    \item $P$ skal opfylde
    \begin{align}
        &P(E) = 1 \\
        &P(A \cup B) = P(A) + P(B), \quad A \cap B = \emptyset
    \end{align}
\end{itemize}

$\eps$ er kun en klasse af pæne delmængder. Dette er ikke et problem på dette kursus (eller andre på økonomisk institut).

\subsubsection{Indikator funktion}

En indikator funktion kan tage en af de to værdier $\{0,1\}$.
$\indic_{A}(x) = 1$ hvis $x\in A$, ellers $0$

\subsubsection{Regneregler for sandsynlighedsmål $P$ (sætning 1.3.4)}

\begin{enumerate}
    \item regler hvis $B \subseteq A$: \begin{align}
        &P(A \setminus B) = P(A) - P(B) \\
        &P(B) \leq P(A)
    \end{align}
    \item regler for den komplementære hændelse til $B$. i.e. $E\setminus B$ \begin{align}
        &P(E\setminus B) = 1 - P(B)
    \end{align}
    \item \begin{equation}
        P(\emptyset) = 0
    \end{equation}
    \item \begin{equation}
        P(A \cup B) = P(A) + P(B) - P(A\cap B)
    \end{equation}
    \item \begin{equation}\label{eq:r1:1}
        P(A) + P(B) \leq P(A) + P(B)
    \end{equation}
\end{enumerate}

\textbf{ligning \ref{eq:r1:1}} kan udvides til vilkårligt mange mængder. Det bliver en lighed hvis samtlige vilkårlige mængder er disjunkte.

\horizline
\textbf{readings:} Sørensen 1.4-1.5

\subsection{Betingede sandsynligheder og uafhængighed}

\subsubsection{Den betingede sandsynlighed af $B$ giver $A$ skrevet $P(B\mid A)$, er defineret ved (definition 1.4.1}

\begin{equation}
    P(B\mid A) = \frac{P(A\cap B)}{P(A)}
\end{equation}

\subsubsection{Regneregler for betingede sandsynligheder}

$A_1, A_2, \cdots A_n$ være n hændelser, hvor $P(A_1 \cap A_2 \cap \cdots A_{n-1}>0$. Da:
\begin{align}
    &P(A_1 \cap A_2 \cap \cdots A_n) = \\ &P(A_{1})P(A_2\mid A_1)P(A_3\mid A_1 \cap A_2) \cdots P(A_n \mid A_1 \cap \cdots \cap A_{n-1})
\end{align}

\textbf{Endnu en regneregel}

Hvis $A_1, A_2, \cdots A_n$ er $n$ disjunkte hændelser, hvor at $E = \bigcup_{i=1}^n A_i$ samt $P(A_i)>0$, da gælder for en vilkærlig hændelse $B$:

\begin{equation}
    P(B) = \sum_{j=1}^n P(B\mid A_j)P(A_j)
\end{equation}

\subsubsection{Omvendingsformel - simpel bayes'}

\begin{equation}
    P(A\mid B) = P(B \mid A) \frac{P(A)}{P(B)}
\end{equation}

\subsubsection{Bayes' formel}

$A_1, A_2, \cdots A_n$ er $n$ disjunkte hændelser, hvor at $E = \bigcup_{i=1}^n A_i$ samt $P(A_i)>0$. For en hændelse $B$ med $P(B)>0$, da gælder for en enhver hændelse $k$:

\begin{equation}
    P(A_k \mid B) = \frac{P(B\mid A_k) P(A_k)}{\sum_{j=1}^n P(B \mid A_j) P(A_j)}
\end{equation}

\subsection{Stokastisk Uafhængighed}

\textbf{Uafhængighed} tænkes oftest som:

\begin{equation}
    P(A \mid B) = P(A)
\end{equation}

Altså at sandsynligheden for $A$ ikke er påvirket af udfaldet af $B$. 

\subsubsection{Definition af uafhængighed}

hændelse $A$ og $B$ er uafhængige siges at være stokastisk uafhængige når (definition 1.5.1) :

\begin{equation}
    P(A\cap B) = P(A) \cdot P(B)
\end{equation}

Dette udsagn kan let udvides til $n$ hændelser (se p.34 \textbf{definition 1.5.4})

\subsubsection{Regler for indbyrdes uafhængighed}

Tegn for uafhængighed $\independent$.
\newline

$A$, $B$ og $C$ er indbyrdes uafhængige hændelser. Følgende gælder:

\begin{enumerate}
    \item $A\setminus B \independent C$
    \item $A \cap B \independent C$
    \item $A \cup B \independent C$
    \item $E\setminus A, B \independent C$
\end{enumerate}

\subsubsection{forenings mængdens uafhængighed}

$A$, $B$, $C$ er hændelser. $A$ og $B$ er betinget afhængige givet $C$ hvis:

\begin{equation}
    P(A \cap B \mid C) = P(A\mid C) \cdot P(B\mid C)
\end{equation}

denne kan generaliseres (se p. 37 \textbf{definition 1.5.7})

\horizline
\subsection{Kapitel 3: Likelihood funktionen}

Under i.i.d antagelser kan man udlede

\begin{equation}
    \prod_{i=1}^{n} f_{Y_i} (y_i \mid \theta)
\end{equation}

Likelihoid contribution

\begin{equation}
    l(\theta \mid y_i) = f_{Y_i} (y_i \mid \theta )
\end{equation}

\begin{equation}
    L(\theta \mid y_1, y_2 , \cdots , y_n) = \prod_{i=1}^n l (\theta  \mid y)
\end{equation}

Tag logaritmen:

\begin{equation}
    \sum_{i=1}^n \log l (\theta \mid y_i)
\end{equation}

Maximér via differentiation.




\subsection{Flerdimensionale stokastiske variable, uafhængighed}

\textbf{readings:} 	Sørensen 3.5, 4.1, 4.4

\horizline

\subsubsection{Polynomial fordelingen}

\subsubsection{Poisson fordelingen}

\begin{equation}
    p(x) = \frac{\lambda^x}{x!}e^{-x}, x\in N_{0}
\end{equation}

\subsubsection{Generelle diskrete fordelinger}

\begin{itemize}
    \item Fordelinger på uendelige tællelige mængder 
    \item $\sum_{i=1}{^\infty} p(x_i) = 1$ 
\end{itemize}
\begin{itemize}
    \item Kontinuert fordeling af 1 dimension
    \item middelværdi
    \item varians
    \item normalfordelingen
    \item transformationer
    \item $\chi^2$ fordeling
\end{itemize}

\subsection{Middelværdi}
$X$ har en middelværdi hvis: 

\begin{equation}
    \int_{- \infty}^{\infty} \lvert x \rvert p(x) dx < \infty
\end{equation}


Middelværdien for $X$ er:

\begin{equation}
    \E(X) = \int_{- \infty}^{\infty} x p(x) dx < \infty
\end{equation}

\subsection{Varians}

Variansen ekstisterer hvis:
\begin{equation}
    \int_{-\infty}^{\infty}x^2 p(x) dx < \infty    
\end{equation}

Variansen er:
\begin{equation}
    \Var(X) = \E([X - \E(X)]^2)
\end{equation}

\subsubsection{Normalfordelingen}

Standard normalfordeling:

\begin{equation}
    \phi(x) = \frac{1}{\sqrt{2\pi}}e^{-x^2/2},\qquad x \in \R
\end{equation}

Fordelingsfunktionen (CDF)

\begin{align}
    \Phi(x) = \int_{-\infty}^{x} \phi(y)dy
\end{align}

Den generelle statndard normalfordeling:

$Y = \mu + \sigma X$

\begin{equation}
    p(y) = \frac{1}{\sqrt{2\pi \sigma^2}}exp\left( - \frac{(y-\mu)^2}{2\sigma^2}\right)
\end{equation}

med middelværdi $\mu$ og varians $\sigma^2$

\subsection{Transformationer}

\begin{equation}
    q(y) = 
    \begin{cases}
        p(t^{-1}(y)) \lvert \frac{d}{dy} t^{-1}(y) \rvert , \qquad &y \in (v,h) \\
        0, \qquad &y \notin (v,h)
    \end{cases}
\end{equation}

hvor $v = \inf t(I), h = \sup t(I)$ og $I$ er intervallet $(a,b)$

\subsection{Kapitel 6: Konfidens Intervaller og hypotese test}

\subsubsection{Konfindens Intervaller}

Vi har fra Theorem 4.1

\begin{equation}
    \sqrt{n}(\hat{\theta} - \theta_0) \rightarrow N(0, \Omega_0)
\end{equation}

Side (98) viser vi kan herfra komme til udtrykket:

\begin{equation}
    \frac{\hat{\theta} - \theta_0}{\se(\hat{\theta})} \sim N(0,1)
\end{equation}

Altså vi har nu konstrueret en stokastisk variabel som er standard normalt fordelt. Vi ved at 95\% af sandsynlighedsmassen ligger i intervallet:

\begin{equation}
    -1.96 < Z < 1.96
\end{equation}

Hvor $Z$ er en standard normalt fordelt stokastisk variabel. Alternativt kunne man sige:

\begin{equation}
    P(-1.96 < Z < 1.96) = 0.95
\end{equation}

Derfra kan man altså let udlede at:

\begin{equation}
    P(\hat{\theta} - 1.96\cdot\se(\hat{\theta}) < \theta_0 < \hat{\theta} + 1.96\cdot\se(\hat{\theta})) = 0.95
\end{equation}

\subsubsection{Hypotese test}

\begin{equation}
    H_0 : \theta_0 =a
\end{equation}

\begin{equation}
    H_A : \theta_0 \neq a
\end{equation}

Altså vi har en $H_0$ (det vi tester). Og en $H_A$, alternativet. Den urestriktere model kaldes $H_U$. Den urestrikterede model er den vi har arbejdet med i kurset op til dette punkt. Vi husker at $\Theta$ (vores parameter rum) er defineret af $H_U$ Vi kan nu sige at:

Under $H_0$ er parameterrummet:

\begin{equation}
    \theta \in \Theta_0 = \{a\}
\end{equation}

Undwe $H_A$ er parameterrummet

\begin{equation}
    \theta \in \Theta_A = \{\theta \in \Theta : \theta \neq a\}
\end{equation}

Man ser altså at: $\Theta_0 \cap \Theta_A = \emptyset$ og at: $\Theta_0 \cup \Theta_A = \Theta$


\begin{table}[ht]
\centering
\begin{tabular}{@{}lll@{}}
\toprule
                  & $H_0$ er sand & $H_0$ er falsk \\ \midrule
Test afvises ikke & Korrekt       & Type 2 fejl    \\
test afvises      & Type 1 fejl   & Korrekt        \\ \bottomrule
\end{tabular}
\end{table}


\subsubsection{Wald-Test}

Kig bog for eksempel.

$H_0 : \theta_0 = a$ og $H_A : \theta_0 \neq a$

\begin{equation}
    p(\theta_0 = a) = \frac{\hat{\theta} - a}{\se(\hat{\theta})} = k
\end{equation}

Hvor vi afviser $H_0$ hvis $k$ er større end 1.96 eller mindre end $-1.96$.

Læs nærmere i kapitel for p-værdi.

Man kan også lave en \textit{squared wald-test} Hvor man kvadrere $Z$. Her skal man teste i en $\chi^2$-fordeling.

\subsubsection{LR-test}

Vi definere $\tilde{\theta}$ som:

\begin{equation}
    \tilde{\theta} = \underset{\theta \in \Theta_0}{\argmax} \sumn \log l(\theta \mid y_i)
\end{equation}

Faldet i likelihood under restriktionen af $H_0$ can blive måldt med

\begin{equation}
    \frac{L_n \tilde{\theta}}{L_n \hat{\theta}}
\end{equation}

Log likelihood ratio $(LR)$

\begin{equation}
    LR_n (H_0) = -2 \log \lp \frac{ L_n \tilde{\theta}}{ L_n \hat{\theta}} \rp
\end{equation}

\begin{equation}
    = 2 \lsp \log L_n (\hat{\theta}) - \log L_n (\tilde{\theta}) \rsp
\end{equation}

Det kan vises at $ LR_n (H_0)$ konvergerer mod en $\chi^2(v)$-fordeling, under en sand 0 hypotese, hvor $v$ er \textit{antal frihedsgrader}. Som i dette tilfælde er antal restriktioner under $H_0$.

Kig bog for eksempler.

\subsection{Normalfordelingsteori}

$\chi^2$ fordelingens tæthed med k-frihedsgrader
\begin{equation}
    p(x) = \frac{x^{\frac{k}{2}}e^{-x/2}}{2^{k/2}c_k}
\end{equation}

Kig nærmere i bogen for den 2-dimensionelle normalfordeling!
\subsection{Grænseresultater for stokastiske variable}

\textbf{readings:}  Sørensen 7

\subsubsection{Store tals lov}

For $n$ ukorrelerede stokastiske variable med middelværdi $\mu$ og varians $\sigma^2$ vil gennemsnittet (det empiriske) være tæt på middelværdien $\mu$.

Altså gennemsnittet vil konvergere med middelværdien. Gennemsnittet er stadig en stokastisk variabel

\subsubsection{Den centrale grænseværdi sætning}

Her er det værd at notere:

\begin{equation}
    \E(\bar{X}_n) = \mu
\end{equation}

Og måske mindre klart:

\begin{equation}
    \Var(\bar{X}_n) = \frac{1}{n^2} ( \Var(X_1) + \Var(X_2) + \cdots + \Var(X_n)) = \frac{1}{n^2}(n \Var(X_i)) = \frac{1}{n^2} n \sigma^2 = \frac{\sigma^2}{n}
\end{equation}

Man altså se at vi kan standardisere den stokastiske variabel som er middelværdien:

\begin{equation}
    U_n = \frac{\bar{X}_n - \mu}{\sigma / \sqrt{n}}
\end{equation}

Nu er reskaberne til sætningen klar:

En række af stokastiske uafhængige varible, som er identiske. Da vil fordelingen af $U_n$ konvergere mod en standard normalfordeling når $n\rightarrow \infty$

Dette kaldes konvergens i fordeling



% Heino Nielsen Concepts
\subsection{Deskriptiv statistik}

Vi har indsamlet noget data:

\begin{equation}
    y_1, y_2, \cdots ,y_n
\end{equation}

Man kan forestille sig en DGP (data generende proces) have forskellige karakteristika, hvormed den mapper til en respons variabel, som kan være:

\begin{itemize}
    \item binær
    \item tælle
    \item diskret
    \item kontinuær
\end{itemize}

frekvens for $j$'te element i $\Y$ udregnes ved:

\begin{equation}
    f_{y = j} \sum_{i=0}^N \1(y_i = j)
\end{equation}

Den empiriske cumulative distribution

\begin{equation}
    F(y) = \sum_{i=1}^N \frac{\1(y_i \leq y)}{n}
\end{equation}

Empriske momenter:

\begin{itemize}
    \item mean (gennemsnit) = $\frac{1}{N}\sum_{i=1}^N y_ i$
    \item Varians = $\frac{1}{N}\sum (y_i - \Bar{y})^2$
    \item Skewness = $\frac{1}{N}\sum (y_i - \Bar{y})^3$
    \item Kurtosis = $\frac{1}{N}\sum (y_i - \Bar{y})^4$
\end{itemize}

standard afvigelse:

\begin{equation}
    std(Y) = \sqrt{Var(Y)}
\end{equation}

Excess Kurtosis vil sige Kurtosis - 3. Da en standard normal fordeling har kurtosis  på  3.

Man vil ofte standardisere data når man kigger på skewness og kurtosis.

Standardisering af data er:

\begin{equation}
    y_{i, standardised} = \frac{y_i - \Bar{y}}{std(y)}
\end{equation}

hvor $std$ betyder standard deviation (standard afvigelse).

\subsection{Kapitel 3: Likelihood funktionen}

Under i.i.d antagelser kan man udlede

\begin{equation}
    \prod_{i=1}^{n} f_{Y_i} (y_i \mid \theta)
\end{equation}

Likelihoid contribution

\begin{equation}
    l(\theta \mid y_i) = f_{Y_i} (y_i \mid \theta )
\end{equation}

\begin{equation}
    L(\theta \mid y_1, y_2 , \cdots , y_n) = \prod_{i=1}^n l (\theta  \mid y)
\end{equation}

Tag logaritmen:

\begin{equation}
    \sum_{i=1}^n \log l (\theta \mid y_i)
\end{equation}

Maximér via differentiation.





%\subsection{Flerdimensionale stokastiske variable, uafhængighed}

\textbf{readings:} 	Sørensen 3.5, 4.1, 4.4

\horizline

\subsubsection{Polynomial fordelingen}

\subsubsection{Poisson fordelingen}

\begin{equation}
    p(x) = \frac{\lambda^x}{x!}e^{-x}, x\in N_{0}
\end{equation}

\subsubsection{Generelle diskrete fordelinger}

\begin{itemize}
    \item Fordelinger på uendelige tællelige mængder 
    \item $\sum_{i=1}{^\infty} p(x_i) = 1$ 
\end{itemize}

%\fi

%\iffalse
\section{Lectures}


\horizline

\subsection{Øvelse 1}

\textbf{10/09/2018, opgaver: 1.1, 1.4, 1.17, 1.24 (og 1.13 hvis der er tid)}

\subsubsection{Opgave 1.1}

\begin{itemize}
    \item en fair mønt
    \item 3 kast
\end{itemize}

Udfaldsrummet $E$ har $2^3$ udfald:

$E=\{(0,0,0),(0,0,1),(0,1,0),(0,1,1),(1,0,0),(1,0,1),((1,1,0),(1,1,1)\}$

\textbf{SSH for 1 mønt=krone:}

\begin{equation}
    p((1,1,1)) = \frac{\text{Gunstige udfald}}{\text{mulige udfald}} = \frac{1}{8}
\end{equation}

\textbf{SSH for mindst 1 mønt=krone}. \newline

brug den komplementere sandsynlighed: $p((0,0,0)) = \frac{1}{8}$. Kald denne hændelse $B$.

\begin{equation}
    P(E\setminus B) = 1 - \frac{1}{8} = \frac{7}{8}
\end{equation}

\textbf{SSH for præcis et kast viser mønt=krone}

Vi definere 3 hændelser

\begin{itemize}
    \item $A = \{ \text{Det første kast bliver krone}\}$, 
    \item $B = \{\text{Det andet kast bliver krone}\}$, 
    \item $C = \{\text{Det tredje kast blive krone}\}$
\end{itemize}

Undersøg om dette er korrekt:
\begin{equation}
    P(A \cup B \cup C) = P(A) + P(B) + P(C)  = 3\times \frac{1}{8} = \frac{3}{8}
\end{equation}

\textbf{spørgsmål: hvorfor kan vi ignorere fællesmængden: denoted} $A \cup B \cup C$? => Den er disjunkt. Kig i opgave 1.13 for at se hvordan man skulle have inkluderet fællesmængderne

\subsubsection{Opgave 1.4}

\begin{itemize}
    \item 3 slag med terninger  $\{1, 2, 3, 4, 5, 6 \}$
    \item Ssh summen er 10
\end{itemize}

1) Vi ser at summen kan antage alle hele tal mellem 3 og 18. 

2) Vi kan se det ikke er en ligefordeling af summer: dvs. summen 3 er ikke så hyppig som summen 10. 

3) Det samlede antal udfald er $6^3$

4) Via computer fandt jeg det gunstige antal udfald: 
\begin{equation}
    \frac{\text{antal gunstige udfald}}{\text{antal mulige udfald}}=\frac{27}{6^3}=\frac{27}{216}
\end{equation}


\subsubsection{Opgave 1.17}

\begin{itemize}
    \item 1 sort terning
    \item 1 hvid terning
\end{itemize}

\textbf{del 1) Hvad er den betingede ssh. for at summen er 12 givet summen er mindst 11}

Brug reglen for betingede sandsynligheder

\begin{equation}
    P(A\mid B) = \frac{P (A \cap B)}{P(B)}
\end{equation}

Lad $A$ være sandsynligheden for summen er 12.

Lad $B$ være sandsynligheden for summen er mindst 11.

\begin{equation}
    P(A) = p((6,6)) = \frac{1}{36}
\end{equation}

\begin{equation}
    P(B) = P(\{(6,6),(5,6),(6,5)\}) = \frac{3}{36}
\end{equation}

 vi ser at $A\subset B \Rightarrow P(A \cap B) = P(A)$ 

\begin{equation}
    P(A\mid B) = \frac{P (A )}{P(B)} = \frac{\frac{1}{36}}{\frac{3}{36}} = \frac{1}{3}
\end{equation}

\textbf{Del 2) Hvad er den betingede SSH for at de to terninger viser det samme, givet summen er 7}:

$A$ er hændelsen for begge er terninger viser det samme.

$B$ er hændelsen summer af terningerne er 7.

$A = \{(1,1),(2,2),(3,3),(4,4),(5,5),(6,6)$

$B = \{(1,6),(2,5),(3,4),(4,3),(5,2),(6,1)\}$

Vi ser at $P(A\cap B) = \emptyset$

Man husker $P(\emptyset)=0$ Givet fra definitioner af sandsynlighedsmål.

\begin{equation}
    P(A \mid B) = 0
\end{equation}

\textbf{Del 3) Ssh for den hvide terning viser 3, givet den sorte viser 5}

$A$: er hændelsen at den hvide terning er 3.

$B$: er hændelsen den sorte terning er 5.

Hændelserne er uafhængige!

\begin{equation}
    A = \{ (3,1), (3,2), (3,3), (3,4), (3,5), (3,6) \}
\end{equation}

\begin{equation}
    B = \{ (1,5), (2,5), \cdots , (6,5) \}
\end{equation}

Vi ser: $P(A \cup B) = P\{(3, 5)\} = \frac{1}{6^2}$.

Vi ser: $P(B) = \frac{1}{6}$
\begin{equation}
    P(A \mid B) = \frac{P(A \cap B)}{P(B)} = \frac{\frac{1}{36}}{\frac{1}{6}} = \frac{1}{6}
\end{equation}

\textbf{Del 4) Ssh. for den mindste terning viser 2, givet den terning med det højeste andel højest viser 5}

$A$: hændelsen at den mindste terning viser 2.

$B$: hændelsen at den terning med det højeste antal øjne viser 5.

$A = \{(2,2),(2,3),\cdots, (2,6), (3,2), (4,2), \cdots, (6,2)$

$B = \bigcup_{i,j \in \{1,2,3,4,5\}} (i,j) =  E \setminus \{(1,6),(2,6),\cdots,(5,6),(1,6),(2,6) \cdots (5,6),(6,6) \} $

Vi kan finde $A\cap B$:

\begin{equation}
    A \cap B = (2,2), (2,3),(2,4),(2,5),(3,2),(4,2),(5,2)
\end{equation}

\begin{equation}
    P(A \cap B) = \frac{7}{6^2}= \frac{7}{36}
\end{equation}

\begin{align}
    P(B) &= P(E) - P(\{(1,6),(2,6),\cdots,(5,6),(1,6),(2,6) \cdots (5,6),(6,6) \}) \\ &= 1 - \frac{11}{36} = \frac{25}{36}
\end{align}

\begin{equation}
    P(A \mid B) = \frac{\frac{7}{36}}{\frac{25}{36}} = \frac{7}{25}
\end{equation}

\subsubsection{Opgave 1.24}

\begin{itemize}
    \item 1 hvid terning
    \item 1 sort terning
    \begin{itemize}
        \item A = \{den hvide terning viser 4\}
        \item B = \{den sorte terning viser 1\}
        \item C = \{terningen med det højeste antal øjne viser 4 \}
        \item D = \{summen af øjene er 5 \}
        \item F = \{summen af øjnene er 7 \}
    \end{itemize}
\end{itemize}

hvilke par er indbyrdes uafhængige:

Husk uafhængighed er: $P(A \cap B) = P(A)P(B)$

\begin{align}
    P(A) &= \frac{1}{6} = \frac{6}{36}\\
    P(B) &= \frac{1}{6} = \frac{6}{36} \\
    P(C) &= P(\{(1,4),(2,4),(3,4),(4,4), (4,1),(4,2),(4,3)\})= \frac{7}{36} \\
    P(D) &= P(\{(1,4),(2,3),(3,2),(4,1)=\frac{4}{36} \\
    P(F) &= \{(1,6),(6,1) \} = \frac{1}{12}= \frac{3}{36}
\end{align}

Elementer i hver fællesmængde:

\begin{figure}[ht]
    \centering
        \begin{tabular}{rrrrrl}
        \toprule
         A &  B &  C &  D &  F & set \\
        \midrule
         6 &  1 &  4 &  1 &  1 &   A \\
         1 &  6 &  1 &  1 &  1 &   B \\
         4 &  1 &  7 &  2 &  2 &   C \\
         1 &  1 &  2 &  4 &  0 &   D \\
         1 &  1 &  2 &  0 &  6 &   F \\
        \bottomrule
        \end{tabular}
    \caption{opg. 1.24 - elementer i hver fællesmængde}
    \label{tab:1.24}
\end{figure}

\newpage

\begin{figure}[ht]
    \centering
        \begin{tabular}{rrrrrl}
        \toprule
             A &      B &      C &      D &      F & set \\
        \midrule
         0.167 &  0.028 &  0.111 &  0.028 &  0.028 &   A \\
         0.028 &  0.167 &  0.028 &  0.028 &  0.028 &   B \\
         0.111 &  0.028 &  0.194 &  0.056 &  0.056 &   C \\
         0.028 &  0.028 &  0.056 &  0.111 &  0.000 &   D \\
         0.028 &  0.028 &  0.056 &  0.000 &  0.167 &   F \\
        \bottomrule
        \end{tabular}
    \caption{opg 1.24 - Sandsynlighed for fællesmængde}
    \label{tab:1.24-probs}
\end{figure}

\begin{figure}[ht]
    \centering
        \begin{tabular}{rrrrrl}
        \toprule
        A &      B &      C &      D &      F & set \\
        \midrule
        0.028 &  0.028 &  0.032 &  0.019 &  0.028 &   A \\
        0.028 &  0.028 &  0.032 &  0.019 &  0.028 &   B \\
        0.032 &  0.032 &  0.038 &  0.022 &  0.032 &   C \\
        0.019 &  0.019 &  0.022 &  0.012 &  0.019 &   D \\
        0.028 &  0.028 &  0.032 &  0.019 &  0.028 &   F \\
        \bottomrule
        \end{tabular}
    \caption{opg. 1.24 - Sandsynligheden for $P(A)\cdot P(B)$}
    \label{tab:1.24_3}
\end{figure}


De uafhængige par er: $(A,B), (A,F), (B,F)$

\subsubsection{Opgave 1.13}

\begin{align}
    P(A\cup B \cup C) &= P(A) + P(B\cup C) - P(A \cap (B \cup C) \\
    &= P(A) + P(B) + P(C) - P(B\cup C ) - P(A \cap (B \cup C) 
\end{align}

Hvis vi ser nærmere på den sidste del  \textbf{Lav tegning af mængder! A, B, C har en intersektion}.

\begin{equation}
    P(A \cap (B \cup C) = P(A \cup B) + P(A \cup C) - P(A \cup B \cup C)
\end{equation}

Man husker at der er minus foran denne mængde, sådan at:

\begin{align}
    P(A\cup B \cup C) &= \\ &P(A) + P(B) + P(C) - P(B\cup C ) \\ &- (P(A \cup B) + P(A \cup C) - P(A \cup B \cup C))
\end{align}
\horizline

\subsection{Øvelse 2}

\textbf{15/09/2018, opgaver: 1.6, 1.7, 1.9, 1.15, 1.18, 1.28 og 1.30 (og 1.12 hvis der er tid)}

\subsubsection{1.6}
\begin{itemize}
    \item 1 ternning
    \item 2 slag
\end{itemize}

\textbf{Ssh for mindst 1 sekser}

\begin{align}
    &P(\{\text{mindst en sekser}\}) = \\
    &P(\{(1,6),(2,6),\cdots,(6,6),(6,1),\cdots, (6,5) = \\
    &\frac{5 + 6}{36} = \frac{11}{36}
\end{align}

\textbf{Ssh. for mindst 1 sekser eller mindst 1 toer}

\begin{align}
    &P(\{\text{mindst en sekser} \}) = \\
    &P(\{(1,6),(2,6),\cdots,(6,6),(6,1),\cdots,(5,6), \\
    &(1,2),\cdots(5,2),(2,1)\cdots(2,5) \}) = \\
    &\frac{6+5+5+4}{36} = \frac{20}{36}
\end{align}

\subsubsection{Opgave 1.7}

\begin{itemize}
    \item 1 mønt
    \item 10 kast
\end{itemize}

\textbf{Hvad er ssh. for mindst 2 plat}

Find sandsynligheden for komplimenter hændelsen: 

$A:$ Er hændelsen for at få mindst 2 plat.

$A^C :$ Er Komplementær hændelsen - altså maks 1 plat:

\begin{equation}
    A^C = \{\text{slå 0 plat} \} \cup \{\text{slå 1 plat} \}
\end{equation}

\begin{equation}
    P(\{\text{slå 0 plat}\}) = \frac{1}{2^{10}}
\end{equation}

\begin{equation}
    P(\{\text{slå 1 plat}\}) = \frac{10}{2^{10}}
\end{equation}

Noter at $\{\text{slå 0 plat}\} \cap \{\text{slå 1 plat}\} = \emptyset$

\begin{equation}
    P(A^C)=\frac{1}{2^{10}} + \frac{10}{2^{10}} = \frac{11}{2^{10}}
\end{equation}

\begin{equation}
    P(A) = 1 - P(A^C) = 1 - \frac{11}{2^{10}} = \frac{1013}{2^{10}}
\end{equation}

\subsubsection{Opgave 1.9}

\begin{itemize}
    \item 1 spil kort (52 kort)
    \item 13 kort trækkes
\end{itemize}

\textbf{Hvad er Ssh. for 0 billedkort eller esser}

Antal billedkort og esser (kaldet billedkort fra nu): $4*4=16$

Kig på komplementær hændelsen:
\begin{equation}
    P(\{\text{kort 1 ikke billedkort}\}) = \frac{52 - 16}{52}
\end{equation}

Vi har trukket 1 kort nu $\implies$ $51$ kort tilbage, men stadig $12$ billedkort
\begin{equation}
    P(\{\text{kort 2 er billedkort}\}) = \frac{51 - 16}{51}
\end{equation}

\begin{equation}
    P(\{\text{man trækker 0 billedkort}\}) = \prod_{i=0}^{12} \frac{52 - i - 16}{52 - i} = 0.0036
\end{equation}

\textbf{Alternativt}

\begin{equation}
    \# E = 52 \cdot 51 \cdots 40 = \frac{52!}{39!}
\end{equation}

\begin{equation}
    \# A = 36 \cdot 35 \cdots 24 = \frac{36!}{23!}
\end{equation}

\begin{equation}
    P(\{\text{man trækker 0 billedkort}\}) = \frac{\# A}{ \# E} = 0.0036
\end{equation}

\subsubsection{Opgave 1.15}


\begin{itemize}
    \item 4 slag med terning
    \item mindst 1 sekser
    \item demere mente $4\times \frac{1}{6}$
\end{itemize}

\textbf{Hvorfor tog han fejl?}

Klasse diskussion:

Kig på komplementærhændelsen: \textit{Ingen seksere}

\begin{equation}
    P(\{\textbf{Ingen seksere}\}) = (\frac{5}{6})^{4} = \frac{5^4}{6^4} = 0.49
\end{equation}

Da dette er komplementær hændelsen kan vi i stedet sige: 

\begin{equation}
    P(\{\textbf{mindst 1 sekser}\} = 1 - 0.49 = 0.51
\end{equation}

\textbf{Ssh for en dobbelt sekser i 24 kast}

\begin{itemize}
    \item 24 kast
    \item mindst 1 dobbelt sekser
\end{itemize}

Sandsynligheden for 1 dobbelt sekser i et slag.
\begin{equation}
    P(\{\textbf{En dobbelt sekser} \}) = \frac{1}{6}\frac{1}{6} = \frac{1}{36}
\end{equation}

Brug komplementær hændelsen: Dvs. ssh for ikke at få en dobbelt sekser i 24 slag:

\begin{equation}
    P(\{\textbf{Ingen dobbelt sekser i 24 slag} \}) = (\frac{35}{36})^{24} = 0.509
\end{equation}

\begin{equation}
    P(\{\textbf{mindst en dobbelt sekser i 24 slag} \}) = 1 - 0.509 = 0.491
\end{equation}

Så ikke langt fra!

\subsubsection{Opgave 1.18}

\begin{itemize}
    \item 1 mønt
    \item 10 kast
\end{itemize}

\textbf{Hvad er ssh. for at få krone den 10'ende gang givet 9 plat}

Lad os definerer hændelserne:

$A:$ Man har fået 9 plat på de første 9 slag af de 10 slag

$B:$ Man får krone på det sidste slag ud af de 10 slag

Brug definition for betingede ssh (1.4.1):

\begin{equation}
    P(A \mid B) = \frac{P(A \cap B)}{P(A)}
\end{equation}

\begin{equation}
    P(A \cap B) = (\frac{1}{2})^{10} = \frac{1}{2^{10}}
\end{equation}

\begin{equation}
    P(A) = (\frac{1}{2})^{9} = \frac{1}{2^9}
\end{equation}

\begin{equation}
    P(B \mid A) = \frac{P(A \cap B)}{P(A)} = \frac{ \frac{1}{2^{10}}}{\frac{1}{2^9}} = \frac{1}{2}
\end{equation}

\textbf{SSh for den 10 bliver krone, givet 9 af de 10 kast blive plat}

Lad os definerer hændelserne:

$A:$ Man har fået 9 plat ud af de 10 slag

$B:$ Man får krone på det sidste slag ud af de 10 slag

\begin{equation}
    P(A \cap B) = (\frac{1}{2})^{10} = \frac{1}{2^{10}}
\end{equation}

\begin{equation}
    P(A) = 10 \times (\frac{1}{2})^{10} = \frac{10}{2^{10}}
\end{equation}

\begin{equation}
    P(B \mid A) = \frac{1}{10}
\end{equation}

\subsubsection{Opgave 1.28}

\begin{itemize}
    \item 1 terning
    \item 1 kast
    \item Hændelse $A:$ kast er 1,2,3
    \item Hændelse $B:$ kast er 1 eller 4
\end{itemize}

\textbf{Vis at $A$ og $B$ er uafhængige}

Brug Definition 1.5.1: 

\begin{equation}
    P(A \cap B) = P(A) \dot P(B)
\end{equation}

\begin{equation}
    P(A) = \frac{1}{2}    
\end{equation}

\begin{equation}
    P(B) = \frac{1}{3}
\end{equation}

Hvad er fælles mængden af de to hændelser: \textit{at terningen bliver 1}

\begin{equation}
    P(A \cap B) = P(\{\textbf{Terningen bliver 1} \}) = \frac{1}{6} = P(A) \dot P(B)
\end{equation}

Og vi har herved vist, at hændelserne er uafhængige!

\subsubsection{Opgave 1.30}

\textbf{Lad eleverne prøve!}

\begin{itemize}
    \item 3 hændelser: $A, B, C$
    \item $A \independent B$
    \item $A \independent C$
\end{itemize}

\textbf{Kan man fra ovenstående slutte at: 
 $A \independent B \cup C$}

\begin{equation}
    A \independent B \implies P(A)\cdot P(B) = P(A \cap B)
\end{equation}

\begin{equation}
    A \independent C \implies P(A)\cdot P(C) = P(A \cap C)
\end{equation}

Bevis via. modeksempel

$A = \{\textbf{Spar eller hjerter} \}$

$ B = \{\textbf{Spar eller ruder}\}$

$C = \{\textbf{hjerter eller ruder}\}$

$P(A\cap B) = \frac{1}{4} = P(A) P(B)$

$P(A \cap C) = \frac{1}{4} = P(A)P(C)$

$P(A \cap (B \cup C)) = \frac{1}{2} \neq P(A)P(B \cup C) = \frac{1}{2}\frac{3}{4}$

\subsubsection{Opgave 1.12}

\begin{itemize}
    \item 1 slag
    \item 5 terninger
\end{itemize}

\textbf{Sandsynligheden for at få mindst 1 sekser}

Udregn ssh for komplementærhændelsen at få 0 seksere!

Definér hændelsen $A:$ At få mindst 1 sekser
\begin{equation}
    P(A^C) = P(\{\textbf{0 seksere}\} = \left( \frac{5}{6} \right)^5 = 0.402
\end{equation}

\begin{equation}
    1 - A^C = 0.598
\end{equation}
\horizline

\subsection{Øvelse 3}

\textbf{17/9/2017, Opgaver: 2.1 og 2.3 fra Sørensen (2015) samt opgaverne B.1, B.2 og B.3}
\subsubsection{Opgave 2.1}

\begin{itemize}
    \item 1 rød terning
    \item 1 sort terning
    \item $Y := \min(r,s)$
    \item $Z := \max(r,s)$
\end{itemize}

\textbf{Fordelingen for Y}

\textbf{TEGN TERNINGEMATRICEN}

\begin{align}
    P(Y = 1) &= P(\{(1,1),(1,2),\cdots,(1,6),(2,1),\cdots,(6,1)\}) = \frac{11}{36} \\
    P(Y = 2) &= P(\{(2,2),(2,3), \cdots (2,6),(3,2) \cdots (6,2) \})= \frac{9}{36} \\
    P(Y = 3 ) &= \cdots = \frac{7}{36}
\end{align}

Den resterende fordeling for $Y$ er: $P(Y=4) = \frac{5}{36}, P(Y=5) = \frac{3}{36}, P(Y=6) = \frac{1}{36}$.

\textbf{Fordelingen for Z}

\textbf{TEGN TERNINGEMATRICEN}

\begin{align}
    P(Z = 1) &= P(\{(1,1)\}) = \frac{1}{36} \\
    P(Z = 2) &= P(\{(2,1),(2,2),(2,1) = \frac{3}{36} \\
    P(Z = 3 ) &= \cdots = \frac{5}{36}
\end{align}

Den resterende fordeling for $Z$ er $P(Z=4) = \frac{7}{36}, P(Z=5) = \frac{9}{36}, P(Z=6) = \frac{11}{36}$.

Den simultane fordeling er \ref{tab:Y_Z_terning}:


$Y$ er vandret, $Z$ lodret: Vi ved at det må være en øvre trekantsmatrice.

Til diagonalen: Vi ved at der er kun måde at min og maks kan være ens $min(T_1,T_2) = max(T_1, T_2) \implies T_1 = T_2$.

Til den øvre trekant: $Y = 1, Z_2 \implies T_1 = 1, T_2 = 2 \lor T_1=2, T_2 = 1$. Dette kan gøres for alle elementer af den øvre trekant

\begin{table}[ht]
\label{tab:Y_Z_terning}
\centering
\caption{Simultan fordeling}
\begin{tabular}{|l|l|l|l|l|l|l|}
\hline
      & $Y = 1$ & $Y=2$ & $Y=3$ & $Y=4$ & $Y=5$ & $Y=6$ \\ \hline
$Z=1$ & 1/36    & 2/36  & 2/36  & 2/36  & 2/36  & 2/36  \\ \hline
$Z=2$ & 0       & 1/36  & 2/36  & 2/36  & 2/36  & 2/36  \\ \hline
$Z=3$ & 0       & 0     & 1/36  & 2/36  & 2/36  & 2/36  \\ \hline
$Z=4$ & 0       & 0     & 0     & 1/36  & 2/36  & 2/36  \\ \hline
$Z=5$ & 0       & 0     & 0     & 0     & 1/36  & 2/36  \\ \hline
$Z=6$ & 0       & 0     & 0     & 0     & 0     & 1/36  \\ \hline
\end{tabular}
\end{table}


\subsubsection{Opgave 2.3}

\begin{itemize}
    \item Stokastisk variabel er beskrevet i bogen
\end{itemize}

$Y=t(X)$

\begin{align}
    P(Y=1) &= P(X\in\{1,2,3\}) = 0.12 + 0.8 + 0.20 = 0.4 \\
    P(Y=2) &= P(X \in \{4,5\}) = 0.11 + 0.19 = 0.30 \\
    P(Y=3) &= P(X \in \{6,7\}) = 0.14 + 0.06 = 0.20 \\
    P(Y=4) &= P(X \in \{8\}) =0.10
\end{align}

Fordelingsfunktion (CDF):

\begin{align}
    P(Y \leq 0) &= 0 \\
    P(Y \leq 1) &= 0.4 \\
    P(Y \leq 2) &= 0.7 \\
    P(Y \leq 3) &= 0.9 \\
    P(Y \leq 4) &= 1.0 
\end{align}{}

\subsubsection{Opgave B.1}

\begin{itemize}
    \item stokastiske variable $X_1, X_2$
    \item $X_1 = 1$ hvis der var en stor nyhed (ellers 0)
    \item $X_2 = 1$ hvis aktiemarkedet steg/faldt (0 hvis ikke)
    \item $P(X_1 = 1) = \frac{6}{10}$
    \item $P(X_2 = 1) = \frac{3}{10}$
\end{itemize}

\textbf{Simultane fordeling under antagelse af uafhængighed!}

Brug definition 2.4.1 (sørensen)

\begin{align}
    P(X_1 = 0, X_2 = 0) &= P(X_1=0)P(X_2 = 0) = \frac{4}{10} \frac{7}{10} = \frac{28}{100} \\
    P(X_1 = 0, X_2 = 1) &= P(X_1=0)P(X_2=1) = \frac{4}{10}\frac{3}{10} = \frac{12}{100} \\
    P(X_1 = 1, X_2 = 0) &= P(X_1=1)P(X_2=0) = \frac{6}{10}\frac{7}{10} = \frac{42}{10} \\
    P(X_1 = 1, X_2 = 1) &= P(X_1 =1 )P(X_2 = 1) = \frac{6}{10}\frac{3}{10} = \frac{18}{10}
\end{align}

\textit{TEGN BI-MATRICE}

\textbf{DEL 2: Antag IKKE uafhængighed - Hvad er den simultane fordeling $(X_1, X_2)$}

\textit{tegn bimatrice og fyld værdier i løbende!}

\begin{equation}
    P(X_2 = 1 \mid X_1 = 1) = \frac{4}{10}
\end{equation}

Udvid \textbf{Definition 1.4.3}

\begin{equation}
    P(B) = \sum_{j=1}^n P(B \mid A_j)P(A_j) = \sum_{j=1}^n P(B, A_j)
\end{equation}

\begin{equation}
    P(X_2 = 1) = P(X_2 = 1 \mid X_1 =0)P(X_1 = 0) + P(X_2 = 1 \mid X_1 = 1) P(X_1 = 1)  
\end{equation}

\begin{equation}
    P(X_2 = 1) = P(X_2 = 1, X_1 =0) + P(X_2 = 1, X_1 = 0)  
\end{equation}

Vi husker at $P(X_2) = \frac{3}{10}$

\begin{equation}
    \frac{3}{10}= \underset{P(X_1 = 1, X_2 =1)}{\underbrace{\frac{4}{10} \frac{6}{10}}} + P(X_2 =1 \mid X_1 = 0 )P(X_1 = 0)
\end{equation}

\begin{equation}
    \implies P(X_1=0 , X_2 =1) = \frac{6}{100}
\end{equation}

Vi har allerede set at:

\begin{equation}
    P(X_1 = 1, X_2 = 1) = \frac{24}{100}
\end{equation}

Vi går videre:

\begin{equation}
    P(X_1 = 1) = P(X_1 = 1, X_2 = 0) + P(X_1 = 1, X_2 =1)
\end{equation}

Vi indsætter de værdier vi kender:

\begin{equation}
    \frac{6}{10} =   P(X_1 = 1, X_2 = 0) + \frac{24}{100} \implies P(X_1 = 1, X_2 = 0) = \frac{36}{100}
\end{equation}

Vi mangler kun sidste værdi nu:

\begin{equation}
    P(X_2 = 0) = P(X_2=0, X_1=0) + P(X_2=0, X_1 = 1)
\end{equation}

Husker værdier: $P(X_2 = 0) = \frac{7}{10}$ og $P(X_1 = 1, X_2 = 0) = \frac{36}{100}$

\begin{equation}
    \frac{7}{10} = \frac{36}{100} + P(X_1 =0, X_2=0) \implies P(X_1 = 0, X_2=0) = \frac{34}{100}
\end{equation}{}

\textbf{Ændrer fordelingen sig for $X_1$, $X_2$, $X$}

\textit{Spørg klassen}

De marginale distributioner er ens,
De betingede og den simultane er forskellig

\subsubsection{Opgave B.2}

\begin{itemize}
    \item Test for cancer
    \item Den gætter rigtig med 95 \% ssh.
    \item 1 ud af 100.000 mennesker har denne kræft form
\end{itemize}

Lad $X$ for cancer testen $X=1$ implicerer positiv test . Lad $Y$ være en stokastisk variabel som angiver om man har kræft $Y=1$ betyder man har kræft.

Vi kan skitserer nogle sandsynligheder:

\begin{equation}
    P(X=1 \mid Y=1)=0.95, \quad P(X=0 \mid Y=1)=0.05
\end{equation}

\begin{equation}
    P(X=0 \mid Y=0)=0.95, \quad P(X=1 \mid Y=0)=0.05
\end{equation}


\begin{equation}
    P(Y=1) = \frac{1}{100000} = 0.00001
\end{equation}

Brug bayes formel (sætning 1.4.7):

\begin{equation}
    P(A_k \mid B) = \frac{P(B \mid A_k)P(A_k)}{\sum_{j=1}^n P(B\mid A_j)P(A_j)}
\end{equation}


\begin{align}
    &P(Y=1 \mid X=1) \\ &= \frac{P(X=1\mid Y=1)P(Y=1)}{P(X=1 \mid Y=1)P(Y=1) + P(X=1 \mid Y=0)P(Y=0)} 
\end{align}

\begin{equation}
    P(Y=1 \mid X=1) = \frac{0.95 \cdot 0.00001}{0.95 \cdot 0.00001 + 0.05 \cdot 0.99999} = 0.0001899 
\end{equation}

\subsubsection{Opgave B.3}

Kig github!




\subsection{Øvelse 4}

\textbf{21/9/2018, Øvelser: B.4  og 2.4, 2.5, og 2.9 fra Sørensen (2015)}

\subsubsection{Opgave B.4}

\textbf{Lav i klassen}

\begin{itemize}
    \item 1 mønt
    \item 1 terning
    \item $X$ er stokastisk variabel med summen af antal øjne på terning + (0/1) (1 hvis krone).
\end{itemize}

\begin{equation}
    T := \text{Ternings øjne}, \qquad M := \text{Mønt}
\end{equation}

\begin{equation}
    X := T + M
\end{equation}

\textbf{Del 4 - Find $P(X>3)$}

Definer hændelser:

$A = \{ X>3 \} $

$A^{C} = \{ X \leq 3 \}$

udfaldsrummet for den simultane fordeling af T of M $\{0, 1\} \times \{1, 2, 3, 4, 5, 6 \}$


\begin{align}
    &P(A^{C}) = \\ &P(\{(M=0, T=1), (M=0, T=2), (M=0, T=3), \\ &(M=1, T=1), (M=1, T=2)\})
\end{align}

Dette var kun komplementær hændelsen
\begin{equation}
    P(A^{C}) = \frac{5}{12}
\end{equation}


\begin{equation}
    P(A) = \frac{7}{12}
\end{equation}

\textbf{Del 5 - SSh for ulige nummer}

Definér hændelsen.

$A = \{X \in \textbf{Ulige numre}\}$

Disse er alle indbyrdes disjunkte hændelser
$A = \{X = 1 \}  \cup \{ X= 3 \} \cup \{X = 5 \} \cup \{X= 7 \}$

\begin{align}
    P(A) = P(\{X = 1 \}) + P(\{ X= 3 \}) + P(\{X = 5 \}) + P(\{X= 7 \})
\end{align}

\begin{equation}
    P(A) = \frac{1}{12} + \frac{2}{12} + \frac{2}{12} + \frac{1}{12} = \frac{1}{2}
\end{equation}
    
\subsubsection{Opgave 2.4}

L\begin{itemize}
    \item $X$ er en stokastisk variabel som kan antage værdierne $\{1, 2, 3\}$
    \item $P(X =1) = P(X=2) = P(X=3) =  \frac{1}{3}$
    \item En stokastisk variabel $Y=1/X$
\end{itemize}

\textbf{Tegn fordelingsfunktionen for $X$ og $Y$}

Kig Github!

\subsubsection{Opgave 2.5}

Lav første del i klassen

\begin{itemize}
    \item $X_1$, $X_2$ er stokastiske variable.
    \item begge har udfaldsrummet $\{0, 1\}$
    \item $X_1$ marginale fordeling:
    \begin{itemize}
        \item $P(X_1=0)=0.4$
        \item $P(X_1=1)=0.6$
    \end{itemize}
    \item $X_2$ marginale fordeling
    \begin{itemize}
        \item $P(X_2=0)=0.3$
        \item $P(X_2=1)=0.7$
    \end{itemize}
    \item Vi har en stokastisk vektor $X = (X_1, X_2)$
\end{itemize}

\textbf{Del 1) Undersøg uafhængighed når den simultane fordeling af $X$ er:}

\begin{table}[ht]
\centering
\caption{Simultan fordeling af $X$}
\begin{tabular}{|l|l|l|}
\hline
          & $X_1=0$ & $X_1 = 1$ \\ \hline
$X_2 = 0$ & 0.12    & 0.18      \\ \hline
$X_2 = 1$ & 0.28    & 0.42      \\ \hline
\end{tabular}
\end{table}

Se \textbf{definition 2.4.1}: Skriv den op på tavlen!

Vi tester for uafhængighed:

\begin{align}
    P(X_1 = 0)P(X_2 = 0) &= 0.4 \cdot 0.3 = 0.12 \\
    P(X_1 = 0)P(X_2 = 1) &= 0.4 \cdot 0.7 = 0.28 \\
    P(X_1 = 1)P(X_2 = 0) &= 0.6 \cdot 0.3 = 0.18 \\
    P(X_1 = 1)P(X_2 = 1) &= 0.6 \cdot 0.7 = 0.42
\end{align}

Vi ser at $X_1$ er uafhængig af $X_2$.

\textbf{Del 2) Undersøg uafhængighed når den simultane fordeling af $X$ er:}

\begin{table}[ht]
\centering
\caption{Simultan fordeling af $X$}
\begin{tabular}{|l|l|l|}
\hline
          & $X_1=0$ & $X_1 = 1$ \\ \hline
$X_2 = 0$ & 0.15    & 0.15      \\ \hline
$X_2 = 1$ & 0.25    & 0.45      \\ \hline
\end{tabular}
\end{table}

Til klassen: Er \textit{dette overhovedet muligt - givet ovenstående resultat?} 

\textbf{Del 3) gør rede for at begge simulatane fordelinger er i overensstemmelse med de angivne marginale fordelinger}

\begin{align}
    P(X_1 = 0) &= P((0,0)) + P((0,1)) = 0.4 \\ 
    P(X_1 = 1) &= P((1,0))+P((1,1)) = 0.6 \\
    P(X_2 = 0) &= P((0,0)) + P((1,0)) = 0.3 \\
    P(X_2 = 1) &= P((0,1)) + P((1,1)) = 0.7
\end{align}

\subsubsection{Opgave 2.9}

Note brug min() og maks() som funktioner istedet for bogens notation.

\begin{itemize}
    \item 2 terninger, $T_1, T_2$
    \item $T_1, T_2$ er ligefordelt på $\{1, 2, 3, 4, 5, 6\}$
    \item $Y = min(T_1, T_2)$
    \item $Z = max(T_1, T_2)$
\end{itemize}

\textbf{Hvad er den simultane fordeling?}

$Y$ er vandret, $Z$ lodret: Vi ved at det må være en øvre trekantsmatrice.

Til diagonalen: Vi ved at der er kun måde at min og maks kan være ens $min(T_1,T_2) = max(T_1, T_2) \implies T_1 = T_2$.

Til den øvre trekant: $Y = 1, Z_2 \implies T_1 = 1, T_2 = 2 \lor T_1=2, T_2 = 1$. Dette kan gøres for alle elementer af den øvre trekant

\begin{table}[ht]
\centering
\caption{Simultan fordeling}
\begin{tabular}{|l|l|l|l|l|l|l|}
\hline
      & $Y = 1$ & $Y=2$ & $Y=3$ & $Y=4$ & $Y=5$ & $Y=6$ \\ \hline
$Z=1$ & 1/36    & 2/36  & 2/36  & 2/36  & 2/36  & 2/36  \\ \hline
$Z=2$ & 0       & 1/36  & 2/36  & 2/36  & 2/36  & 2/36  \\ \hline
$Z=3$ & 0       & 0     & 1/36  & 2/36  & 2/36  & 2/36  \\ \hline
$Z=4$ & 0       & 0     & 0     & 1/36  & 2/36  & 2/36  \\ \hline
$Z=5$ & 0       & 0     & 0     & 0     & 1/36  & 2/36  \\ \hline
$Z=6$ & 0       & 0     & 0     & 0     & 0     & 1/36  \\ \hline
\end{tabular}
\end{table}


\textbf{Er $Y, Z$ uafhængige}

Husk:

\begin{equation}
    P(Y = A, Z = B) =P(Y=A)P(Z=B) \quad \forall A, B \in \{1, 2, 3, 4, 5, 6 \}
\end{equation}

Vi skal bare have et modeksempel.
Eftersom: $P(Y=1, Z=2)=0$ kan vi konkluderer ikke uafhægighed. \textit{Overvej dette !}


\subsection{Øvelse 5}

\textbf{24/09/2018 - C.1, C.2, C.3 \&  3.20, 3.24, 3.27 (optional 3.2) sørensen}

\subsubsection{Opgave C.1}

\begin{itemize}
    \item Basketball player
    \item 10 skud
    \item ssh for at ramme 0.5
\end{itemize}

Binomial fordeling

\textbf{Hvad er SSh for at ramme 8 skud med ssh 0.5}

\begin{equation}
    p(x) = \begin{pmatrix}
    10 \\ 8 \end{pmatrix} 0.5^8 (1-0.5)^{10-8} = 0.04394
\end{equation}

\textbf{Hvad er SSh for at ramme med ssh 0.6}

\begin{equation}
    p(x) = \begin{pmatrix}
    10 \\ 8 \end{pmatrix} 0.6^8 (1-0.6)^{10-8} = 0.1209
\end{equation}

\textbf{Ssh på 0.5 - hvad er varians of middelværdi}

\begin{equation}
    \E(X) = n\cdot p = 0.5 \cdot 10 = 5
\end{equation}

fra wikipedia

\begin{equation}
    \Var(X) = n \cdot p \cdot (1 - p) = 2.5
\end{equation}

\subsubsection{Opgave C.2}

\begin{itemize}
    \item $X$ er stokastisk variabel
    \item diskret pdf $f(x) = \frac{x}{8}$
    \item $x\in \{1, 2, 5 \}$
\end{itemize}

\textbf{Hvad er E(X)}

\begin{equation}
    \E(X) = \sum_{i=1}^n p_i \cdot x_i = 1\cdot \frac{1}{8} + 2 \cdot \frac{2}{8} + 5 \cdot \frac{5}{8} = \frac{1 + 4 + 25}{8} = 3.75
\end{equation}

\textbf{Hvad er Var(X)}

\begin{equation}
    \Var(X) = E(X^2) - (E(X))^2
\end{equation}

\begin{equation}
    E(X^2) = 1^2 \cdot \frac{1}{8} + 2^2 \cdot \frac{2}{8} + 5^2 \cdot \frac{5}{8} = \frac{1 + 8  + 125}{8} = 16.75
\end{equation}

\begin{equation}
    \Var(X) = 16.75 - 3.75^2 = 16.75 - 14.0625 = 2.6875
\end{equation}

\textbf{Hvad er $E(2X + 3)$}

Vi bruger:

\begin{equation}
    \E(a + bX) = a + b \E(X)
\end{equation}

Husk $\E(X) = 3.75$

\begin{equation}
    2\cdot 3.75 + 3 = 7.5 + 3 = 10.5
\end{equation}

\subsubsection{Opgave C.3}

\begin{itemize}
    \item Efterspørgsel for software er $X$
    \item købspris 10
    \item salgspris 35
    \item Ved årets ende er softwaren intet værd
    \item køber 4 kopier af software
\end{itemize}

\textbf{Find $\E(X)$}

\begin{equation}
    \E(X) = 0.1 \cdot 0 + 0.3 \cdot 1 + 0.3 \cdot 2 + 0.2 \cdot 3 + 0.1\cdot4 = 0.3 + 0.6  + 0.6 + 0.4 = 1.9
\end{equation}

\textbf{Find $\Var(X)$}

\begin{equation}
    \Var(X) = E(X^2) - (E(X))^2
\end{equation}

\begin{equation}
    \E(X^2) = 0.1 \cdot 0 + 0.3 \cdot 1 + 0.3 \cdot 4 + 0.2 \cdot 9 + 0.1\cdot 16 = 0.3 + 1.2  + 1.8 + 1.6 = 4.9
\end{equation}

\begin{equation}
    \Var(X) = 4.9 - 1.9^2 = 4.9 - 3.61 = 1.29
\end{equation}

\textbf{Efterspørgselsfunktion $Y$, samt $\E(Y)$ og $\Var(Y)$}

man køber 4 stykker software $4\times 10$. og sælger $x$ af dem som er en realisation af $X$.

\begin{equation}
    Y := 35X - 40
\end{equation}

husk 
\begin{equation}
    \E(a + bX) = a + b \E(X)
\end{equation}

\begin{equation}
    \E(Y) = \E(35X - 40) = 35\cdot \E(X) - 40 = 3.5 \cdot 1.9 - 40 = 26.5 
\end{equation}

Normalt ville vi sige:

\begin{equation}
    \Var(X) = E(X^2) - (E(X))^2
\end{equation}

Vi gør noget smartere her (kig bog s. 93):

\begin{equation}
    \Var(aX + b) = b^2 \Var(X)
\end{equation}

\begin{equation}
    \Var(Y) = \Var(35X - 40) = 35^2 \cdot \Var(X) = 35^2\cdot 1.29 = 1580.25
\end{equation}

\subsubsection{Opgave 3.20}

\begin{itemize}
    \item en stokastisk variabel $X$ er ligefordelt på $\{1, 2, 3, 4, 5, 6\}$ (en terning)
    \item stokastisk variabel $Y:= R + H$, hvor er og $R, H$ er terninger
    \item $Z$ er stokastisk variabel som er for uniform på $\{1, 2, 3\cdots, n$.
\end{itemize}

\textbf{Find middelværdi og varians for $X$}

Man siger at $X:= unif\{a,b\} = unif\{1, 6\}$

Middelværdi 
\begin{equation}
    \E(X) = \sum_{i=1}^6\frac{1}{6} i = 3.5
\end{equation}

Fra wikipedia om diskrete uniform fordeling \begin{verbatim}
    https://en.wikipedia.org/wiki/Discrete_uniform_distribution
Varians
\end{verbatim}
Generelt er der gode informationer om distributioner på wiki!

\begin{equation}
    \Var(X) = \frac{(b-a+1)^2-1}{12}
\end{equation}

\begin{equation}
    \Var(X) = \frac{(6 - 1 + 1)^2-1}{12} = \frac{35}{12} = 2.92
\end{equation}

\textbf{For Y}

$R, H := unif\{1,6\}$. $Y=R+H$

Vi ved at $R \independent H$

brug Sætning 3.7.7 (s. 91) - (uafhængighed er ikke nødvendig)

\begin{equation}
   \E(Y) = \E(R+H)=\E(R) + \E(H) = 3.5 + 3.5 = 7 
\end{equation}

Grundet uafhængighed kan vi nu bruge sætning 3.8.8 (s. 101)

\begin{equation}
    \Var(X_1 + X_2 + \cdots X_n) = \Var(X_1) + \Var(X_2) \cdots \Var(X_n)
\end{equation}

\begin{equation}
    \Var(Y) = \Var(R) + \Var(H) = 2.92 + 2.92 = 5.84
\end{equation}

\textbf{Middelværdi og varians for $Z$}

Vi kan definere den stokastiske variabel $Z := unif(1,n)$
\begin{equation}
    \E(Z) = \sum_{i=1}^n \frac{1}{n}i = \frac{1}{n}\sum_{i=1}^n i
\end{equation}

summen er $\frac{n(n+1)}{2}$. Vis gaus beviset: vi har $n/2$ gange (1 + n). 1 + 50 = 51, 2 + 49 = 51 osv det kan vi gøre 25 gange.

\begin{equation}
    \E(Z) = \frac{1}{n} \frac{n(n+1)}{2} = \frac{n+1}{2}
\end{equation}

Nu skal variansen udregnes!

\begin{equation}
    \Var(Z) = \E(Z^2) - (E(Z))^2
\end{equation}

I bogen har vi opgivet at:
\begin{equation}
    \sum_{i=1}^n i^2 = \frac{1}{6} n (2n+1)(n+1)
\end{equation}

Vi ved derfor at:

\begin{equation}
    E(Z^2) = \sum_{i=1}^n \frac{1}{n} i^2 = \frac{1}{n} \sum_{i=1}^n i^2 = \frac{1}{n}\frac{1}{6} n (2n+1)(n+1) = \frac{1}{6} (2n+1)(n+1)
\end{equation}

(Andel af udtrykket er $E(Z)^2$)

\begin{equation}
    \Var(Z) = \frac{1}{6} (2n+1)(n+1) - \frac{n+1}{2}\frac{n+1}{2}
\end{equation}

Vi ser udtrykket kan forkortes:

\begin{equation}
    \Var(Z) = \left(\frac{1}{6} (2n+1) - \frac{n+1}{2^2 }\right)(n+1)
\end{equation}

\subsubsection{Opgave 3.24}

\begin{itemize}
    \item en stokastisk variabel $X$
    \item $\E(X) = 5$
    \item $\Var(X) = 2$
\end{itemize}

\textbf{Find $\E(7 + 8X + X^2)$}

\begin{equation}
    \E(7 + 8X + X^2) = \E(7) + \E(8X) + \E(X^2)
\end{equation}

Først ved vi at $\E(7) = 7$.

Dernæst

\begin{equation}
    \E(8X) = 8 \cdot \E(X) = 8\cdot5 = 40
\end{equation}

Til sidst

\begin{equation}
    \Var(X) = \E(X^2) - \E(X)^2 
\end{equation}

Vi kender variansen og $\E(X)$:

\begin{equation}
    2 = \E(X^2) - 5^2 \implies \E(X^2) = 2 + 5^2 = 27
\end{equation}

\begin{equation}
    \E(7 + 8X + X^2) = 7 + 40 + 27 = 74
\end{equation}

\subsubsection{Opgave 3.27}
\begin{itemize}
    \item 3 stokastiske variable
    \item $X_1$, $X_2$, $X_3$
    \item identiske og uafhængige
\end{itemize}

Vis at

\begin{equation}
    \Corr(X_1 + X_2, X_2 + X_3 ) = \frac{1}{2}
\end{equation}

\begin{equation}
    \Corr(X,Y) = \frac{\Cov(X,Y)}{\sqrt{\Var(X)\Var(Y)}}
\end{equation}

\begin{equation}
    \Cov(X,Y) = (X - \E(X))(Y- \E(Y))
\end{equation}

Indsæt vores stokastiske variable $X_1 + X_2$ og $X_2 + X_3$. 

\begin{align}
    &\Cov(X_1 + X_2,X_2 + X_3) = \\ &(X_1 + X_2 - \E(X_1) + \E(X_2))(X_2 + X_3 - \E(X_2) + \E(X_3)) = \\
    &([X_1 - \E(X_1)]+[X_2 - \E(X_2)])([X_2 - \E(X_2)] + [X_3 - \E(X_3)]) = \\
    &\Cov(X_1, X_2) + \Cov(X_1, X_3) + \Cov(X_2, X_3) + \Var(X_2) 
\end{align}

Vi ved at uafhængighed implicerer ar covariancen er lig 0. Det betyder:

\begin{equation}
    \Cov(X_1 + X_2,X_2 + X_3) = \Var(X_2) = \sigma^2
\end{equation}

Brug \textbf{sætning 3.8.8} (s. 101). Man kan splitte variansen op af ukorrelerede stokastiske variabler til en sum

\begin{align}
    &\Var(X_1 + X_2)\Var(X_2 + X_3) = \\
    &(\Var(X_1) + \Var(X_2))(\Var(X_2) + \Var(X_3)) = \\
\end{align}

Vi ved variansen er ens for alle stokastiske variable sådan at: $Var(X_1) = \Var(X_2) = \Var(X_3) = \sigma^2$

\begin{equation}
    2\sigma^2 \cdot 2\sigma^2
\end{equation}

\begin{equation}
    \sqrt{2\sigma^2 \cdot 2\sigma^2} = 2\sigma^2
\end{equation}

Vi har herved fundet det ønskede resultat!

\begin{equation}
    \Corr(X_1 + X_2, X_2 + X_3) = \frac{\sigma^2}{2\sigma^2} =\frac{1}{2}
\end{equation}

\subsubsection{Opgave 3.2}

\begin{itemize}
    \item 5 Cola-smagere
    \item 2 Cola-mærker $\{C, P\}$
    \item med sandsynlighed $p$ gætter de rigtigt
    \item 4 ud af 5 gætte på cola $P$. 1 gættede $C$
\end{itemize}

\textbf{Hvad er den betingede ssh for at det var cola C der blev serveret}

Definér to stokastiske variable:

$S := \{\textbf{Hvilke cola der blev serveret}\}$

$C:= \{\textbf{hvilken cola der blev serveret}\}$

$P(D) =  \frac{1}{2}$

$P(S\mid D) \sim Bin(5,p)$

\begin{equation}
    P(S = 4\mid D=C)\begin{pmatrix} 5 \\ 4\end{pmatrix} p (1-p)^4
\end{equation}

\horizline

\subsection{Øvelse 6}

\textbf{28/09/2018 - C.4 \& Opgave 1}

\subsubsection{Opgave C.4}

\begin{itemize}
    \item Poisson distribution
    \item Antal opkald kan modelleres med en stokastisk variabel kaldet $X:= Poisson(\lambda)$.
\end{itemize}

Om Poisson fordelingen:
En ventetidsfordeling! Citat wikipedia:

\begin{displayquote}
"[Poisson fordelingen] is a discrete probability distribution that expresses the probability of a given number of events occurring in a fixed interval of time or space if these events occur with a known constant rate and independently of the time since the last event." - Wikipedia
\end{displayquote}

Den har den egenskab at:
$\E(X) = \Var(X) = \lambda$

Om denne fordeling kan vi sige at sandsynligheden for et givent udfald er (pdf):

\begin{equation}
    p(x) = \frac{\lambda^x}{x!}e^{-\lambda}
\end{equation}

Lad dem regne selv

\textbf{Ssh for præcis 7}

\begin{equation}
    p(7) = \frac{7^{10}}{7!}e^{-10} = 0.090079 
\end{equation}

\textbf{Ssh for max 7 opkald}

\begin{equation}
    P(X \leq 7) = \sum_{i=0}^7 \frac{i^{10}}{i!}e^{-10} = 0.22022
\end{equation}

\begin{equation}
    P(3 \leq X \leq 7) = \sum_{i=3}^7 \frac{i^{10}}{i!}e^{-10} = \sum_{i=0}^7 \frac{i^{10}}{i!}e^{-10} - \sum_{i=0}^2 \frac{i^{10}}{i!}e^{-10} 
\end{equation}

Indsæt værdier udregnet i python

\begin{equation}
    0.22022 - 0.002769 = 0.217451
\end{equation}

\subsubsection{Opgave 1}

\begin{itemize}
    \item Værdi af cykel 4000 kr
    \item ssh for den bliver stjålet 5 \%
    \item man kan tegne en cykel så den bliver erstattet for hele dens værdi
\end{itemize}

\textbf{Del 1) Hvor meget er man villig til at betale for en sådan forsikring?}

Spørg klassen - Intet rigtigt svar?

\textbf{del 2) Udregn værdi af cykel (på et år)}

Vi definerer X stokastiske variable:

$X:= \textbf{Cykel værdi}$

$P(X=0) = 0.05$ og $P(X=4000) = 0.95$


Så ganger vi værdien på $X$ bagefter.

\begin{equation}
    \E(X) = 0.95 \cdot 4000 = 3800
\end{equation}

\textbf{del 3) Cykel forsikring!}

\begin{equation}
    Y:= \textbf{Værdi af cykel minus forsikring 1}
\end{equation}

\begin{align}
    (Y\mid X=0) &= 0 - 400 + 4000 = 3600\\ (Y \mid X=4000) &= 4000 - 400 = 3600
\end{align}

\begin{equation}
    \E(Y) = 0.95 \cdot 3600 + 0.05 \cdot 3600 = 3600
\end{equation}

\textbf{Del 4) Forsikring med selvrisiko på 1000 kr!}

pris = 150 årligt, selvrisiko = 1000.

$Z := \textbf{Værdi af cykel minus forsikring 2}$

\begin{align}
    (Z \mid X = 0) &= 0 - 150 - 1000 + 4000 = 2850 \\
    (Z \mid X = 4000) &= 4000 - 150 = 3850
\end{align}

\begin{equation}
    \E(Z) = 0.05\cdot 2850 + \cdot 0.95 \cdot 3850 = 3800
\end{equation}

\textbf{Del 5) Sammenlign middel værdier}

Klassediskussion

\textbf{Del 6) Nytte af af $X$, $Y$, $Z$}

nyttefunktion:
\begin{equation}
    u(v) = 10 v - 0.001 v^2, \qquad v \in \{0, 1, \cdots 4000 \}
\end{equation}

Transformér de enkelte stokastiske variable først! X:

\begin{align}
    u(X \mid X = 0) &= 0  \\
    u(X \mid X = 1) &= 10\cdot 4000 - 0.001 \cdot 4000^2 = 24000
\end{align}

transformation af Y:
\begin{align}
    u(Y \mid Y = 3600) &= 10 \cdot 3600 - 0.001 \cdot 3600^2 = 23040\\
\end{align}
    
Transformation af Z:
\begin{align}
    u(Z \mid Z = 3850) = 10 \cdot 3850 - 0.001 \cdot 3850^2 = 23677.5\\
    u(Z \mid Z = 2850) = 10 \cdot 2850 - 0.001 \cdot 2850^2 = 20377.5  
\end{align}

\begin{equation}
    \E(u(X)) = 0.95 \cdot 24000 + 0.05 \cdot 0 = 22800
\end{equation}

\begin{equation}
    \E(u(Y)) = 0.95 \cdot 23040 +  0.05 \cdot 23040 = 23040
\end{equation}

\begin{equation}
    \E(u(Z)) = 0.95 \cdot 23677.5 + 0.05 \cdot  20377.5  = 23512.5
\end{equation}

\textbf{Del 7) Vis generelt udtryk for den forventede værdi af $u(W)$}

\begin{equation}
    u(v) = 10 v - 0.001 v^2, \qquad v \in \{0, 1, \cdots 4000 \}
\end{equation}

lad $W$ være koncentreret på mængden $T$:
\begin{equation}
    \E(u(W)) = \sum_{w \in T} (10\cdot w - 0.001 w^2)p(w)
\end{equation}

\begin{equation}
    \E(u(W)) = \sum_{w \in T} (10\cdot w)p(w) - \sum_{w \in T} (0.001 w^2)p(w)
\end{equation}


\begin{equation}
    \E(u(W)) = 10\cdot\sum_{w \in T} ( w)p(w) - 0.001\cdot \sum_{w \in T} ( w^2)p(w)
\end{equation}

\begin{equation}
    \E(u(W)) = 10\E(W) - 0.001\cdot \E(W^2)
\end{equation}

Vi ved at:
\begin{equation}
    \Var(X) = \E(X^2) - (\E(X))^2 \implies \Var(X) + (\E(X))^2 = \E(X^2)
\end{equation}

Vi bruger dette:

\begin{equation}
    \E(u(W)) = 10\cdot\E(W) - 0.001\cdot (\E(W))^2 + \Var(W)
\end{equation}

Som var det ønskede udtryk

\textbf{Del 8) Udregn variansen af $X$, $Y$, $Z$}

Vi bruger formlen for den varians:

\begin{equation}
    \sum_{x \in T}(x - \E(X))^2p(x)
\end{equation}

Varians af X
\begin{equation}
    0.95 \cdot (3800 - 4000)^2 + 0.05 \cdot (0 - 4000)^2 = 760000
\end{equation}

Varians af Y: Den er $\Var(Y)=0$. Vi får altid udbetalt det samme! \textbf{Definition 3.7.13}

Varians af Z
\begin{equation}
    0.95 \cdot (3850 -3800)^2 + 0.05 \cdot (2850 - 3800)^2 = 47500 
\end{equation}
\horizline

\subsection{Øvelse 7}

\textbf{Opgaver: 3.4, 3.13, 3.14, 4.5, 4.6, (4.14)}

\subsubsection{Opgave 3.4}

\begin{itemize}
    \item 5 terninger kastes
\end{itemize}

\textbf{SSH for 3 seksere}

Man kan bruge både binomial fordelingen og Polynomialfordelingen.

Vi bruger binomialfordelingen $X:=Binom(n = 5, p = 1/6)$


\begin{equation}
    p = \frac{1}{6}
\end{equation}

VI har antalsparameter $n=5$, og antal succeser $x=3$
\begin{equation}
    P(X = 3) = \begin{pmatrix}
    5 \\ 3 \end{pmatrix} \left(\frac{1}{6}\right)^3\left(1 - \frac{1}{6} \right)^{5-3} = 0.0321
\end{equation}

\textbf{SSH for mindst 3 seksere}

\begin{equation}
    P(X \geq 3) =  \sum_{i=3}^n \begin{pmatrix}
    5 \\ i \end{pmatrix} \left(\frac{1}{6}\right)^i\left(1 - \frac{1}{6} \right)^{5-i} = 0.03549
\end{equation}

\textbf{SSh for præcis 3 ens}


Brug hvad vi har udregnet tidligere. SSH for præcis 3 seksere, kan vi gange med 6 for at finde det for alle!
\begin{equation}
    P(Z=3) = 6 \cdot 0.0321 = 0.1929
\end{equation}


\textbf{SSH for mindst 3 ens}


Brug hvad vi regnede ud tidligere for mindst 3 seksere
\begin{equation}
    P(Z=3) = 6 \cdot 0.03549 = 0.2129
\end{equation}

\subsubsection{Opgave 3.13}

\begin{itemize}
    \item $X, Y  \sim Uni(0, N)$
    \item $X \independent Y$
\end{itemize}

\textbf{Find $P(X>Y)$}

Find middelværdien for $X, Y$. 

Vi ser: $P(X>Y \mid Y = 0) = P(X>0)$, $P(X>Y \mid Y = 1) = P(X>1)$. Vi ved at $Y, X$ er ligefordelt sådan at alle ting er lige sandsynlige. Dette implicerer $P(Y=y) = \frac{1}{N + 1}, \forall y \in Y$.

Vi kender CDF af den diskrete uniforme fordeling:

\begin{equation}
    P(Y\geq k) = \frac{k- a + 1}{n}
\end{equation}

Sæt det hele sammen:

\begin{equation}
    P(X \geq Y) = \frac{1}{N + 1} \sum_{i=0}^N \frac{i - 0 + 1}{N + 1} = \frac{1}{N + 1} \frac{1}{N + 1}\sum_{i=0}^N i + 1
\end{equation}

Husk at summen fra 1 til N kan skrives som = $(n+1)n / 2$. I vores tilfælde $(n + 1 + 1)(n+1)/2$ ,grundet vi har $i+1$ i vores sum.

\begin{equation}
    P(X\geq Y) = \frac{1}{(N + 1)^2 } \frac{(N+1)(N+1+1)}{2} = \frac{(N+2)}{2(N+1) }
\end{equation}

\textbf{Find $P(X=Y)$}

Der er $N+1$ udfald.

\begin{equation}
    P(X = Y, Y=y) = \frac{1}{(1 + N)^2}
\end{equation}

Dette er klart tænk på terninger ssh for 1 dobbelt sekser $1/6^2$.

Vi har $1+N$ måder at dette kan ske på:
\begin{equation}
    P(X=Y) (N+1)\frac{1}{(N+ 1)^2}= \frac{1}{N+1}
\end{equation}

\textbf{Find $P(Z)$ hvor $Z \sim max(X,Y)$}

Vi ser at: 
$P(Z = 0) =  P(X = 0, Y = 0)$

Og at:
$P(Z=1) = P(X = 1, Y = 1) + P(Y = 1, X= 0) + P(X = 0, Y = 1)$.

Vi prøver at generaliserer observationen: 


\textbf{Find $P(V)$ hvor $V \sim min(X,Y)$}

DROP AT LAVE

\textbf{Find $P(W)$ hvor $W \sim
\lvert X - Y \rvert$}

DROP AT LAVE
\subsubsection{Opgave 3.14}

LAV I KLASSEN

\begin{itemize}
    \item $(X_1, X_2)$ er en stokastisk vektor
    \item SE OPLÆG for den simultane fordeling
\end{itemize}

\textbf{SSH $X_1$ er et lige tal}

Vi husker relationen mellem marginale, betingede og simultane fordelinger!

\begin{equation}
    P(X_1 = k) = \sum_{i=1}^n P(X_1 = k, X_2 = x_i) 
\end{equation}

Vi ser at $X_1$ skal være et lige tal:

\begin{equation}
    P(X_1 \in \textbf{Lige tal}) = P(X_1 = 0) + P(X_1 = 2) + P(X_1 = 6) = 1 - P(X_{1}=-1)
\end{equation}


\begin{align}
   P(X_{1}=-1) = &P(X_1 = -1, X_2= 3)\\ + &P(X_1 = -1, X_2 = 1) \\ +&P(X_1 = -1, X_2 = -2)
\end{align}

\begin{equation}
       P(X_{1}=-1) = 0 + \frac{2}{9} + \frac{1}{9} = \frac{3}{9} 
\end{equation}

Vi finder den sandsynlighed vi ønskede fra start:

\begin{equation}
    P(X_1 \in \textbf{Lige tal}) = 1 - P(X_{1}=-1) = 1 - \frac{3}{9} = \frac{6}{9}
\end{equation}


\textbf{SSH, $X_{1}X_{2}$ er et ulige tal}

Kravet er at produktet af de to stokastiske variable skal være et ulige tal. Dette vil implicere at $X_1 \in \{\textbf{ulige tal}\}, X_2 \in \{\textbf{ulige tal}\}$. 

\begin{align}
    P(X_1X_2 \in \{\textbf{Ulige tal}\}) &= P(X_1 = -1, X_2 = 3) \\
    &+ P(X_1 = -1, X_2 = 1) \\
    &= \frac{2}{9} 
\end{align}

\textbf{SSH for $X_2>0$ og $X_1 \geq 0$}

\begin{align}
    P(X_2>0, X_1\geq 0) &= 
    P(X_2 = 3, X_1 = 2) \\
    &+ P(X_2 = 3, X_1 = 6) \\
    &+ P(X_2 = 1, X_1 = 2) \\
    &+ P(X_2 = 1, X_1 = 6) \\
    &= \frac{1}{9} + \frac{1}{9} + \frac{1}{9} + \frac{4}{27} = \frac{13}{27}
\end{align}

\subsubsection{Opgave 4.5}

\textit{Lav i klassen}

\begin{itemize}
    \item shh for sikring defekt 0.03
    \item køber pakke med 100 sikringer
\end{itemize}

\textbf{SSH for at i en pakke med 100 sikringer maks 2 er er defekte
}

Brug sætning 4.1.2

VI lader altså vores antal parameter gå mod uendelig. Vi bruger nu en poisson fordeling!

Vi ser at $n\cdot p = \lambda = 100 \cdot 0.03 = 3$

Vi definerer vores stokastiske variabel $X \sim Poisson(\lambda = 3)$

\begin{equation}
    P(X \leq 2) = \sum_{i=0}^2 \frac{\lambda^i}{i!}e^{-\lambda} = \sum_{i=0}^2 \frac{3^i}{i!}e^{-3} \approx 0.42
\end{equation}

\subsubsection{Opgave 4.6}

\begin{itemize}
    \item En terning kastes indtil den første sekser opnås
\end{itemize}

\textbf{Hvad er ssh for at en sekser opnås inden 6 kast.}

%Dette må kunne gøres med en distribution
\begin{equation}
    P(X<6) = 1 - P(X\geq 5) = 1 - (1-1/6)^{5+1} = 0.665 
\end{equation}

5 + 1 fordi 0 skal tælles med

\textbf{Hvad er den største værdi af $i \in \N$ hvor $P(X>i) \geq \frac{1}{2}$ }

\begin{equation}
    P(X > 0) = ( 1 - 1/6)^1 =  0.8333
\end{equation}

\begin{equation}
    P(X > 1) = ( 1 - 1/6)^2 = 0.6944
\end{equation}

\begin{equation}
    P(X > 2) = ( 1 - 1/6)^3 = 0.5787
\end{equation}

\begin{equation}
     P(X > 3) = ( 1 - 1/6)^3 = 0.4822
\end{equation}

VI ser at $i=2$ er det største!!

\subsubsection{Optional (4.14)}

\begin{itemize}
    \item En stokastisk variabel $X$
    \item $X \sim Poisson(\lambda) $
\end{itemize}

\textbf{Hvad er $E \left( 2^X \right)$}

$Z = 2^X$. VI har så at
\begin{equation}
    p(z) = \frac{\lambda^{2^x}}{2^x!}e^{-\lambda}
\end{equation}

\begin{equation}
    \E(Z) = \sum_{i=0}^\infty 2^{i} \frac{\lambda^{2^i}}{2^i!}e^{-\lambda}
\end{equation}

Vi kan trække en fra i nævneren da den bliver ganget på!

\begin{equation}
    = \sum_{i=0}^\infty \frac{\lambda^{2^i}}{(2^i - 1)!}e^{-\lambda}
\end{equation}

Man trækker et lambda fra tælleren ud foran sumtegnet!

\begin{equation}
    = \lambda \sum_{i=0}^\infty \frac{\lambda^{2^i - 1}}{(2^i - 1)!}e^{-\lambda}
\end{equation}
\textbf{Hvad er $E \left((1+X)^{-1} \right)$}


\horizline

\subsection{Øvelse 8}

\textbf{Opgaver: 4.4, opgave A, (Opgave H)}


\subsubsection{Opgave 4.4}

\begin{itemize}
    \item A står ved en lidet trafikkeret vej
    \item Antal taxaer pr. minut, er poisson fordelt med $\lambda = \frac{1}{30}$
\end{itemize}

\textbf{Del 1) Hvad er ssh for A må vente mere end en halv time}

Altså poisson fordelingen måler "antal observationer" som vores x. og vores $\lambda$ som vores parameter. Vi bliver nødt til at gange lambda (det er på minut basis, og vi skal have det på halv time basis) $t$.

$Y \sim Poisson( \lambda = 1/30 t)$


$\lambda = 1/3 * 30$
\begin{equation}
    P(Y = 0) =\frac{\lambda^x}{x!}e^{-1} = \frac{1^0}{0!}e^{-1} = 0.36787
\end{equation}

\textbf{Del 2) Hvad er ssh for at vente 1 1/2 time.}


$\lambda = 1/30 * 90 = 3$
\begin{equation}
    P(Y = 0) =\frac{3^x}{x!}e^{-3} = \frac{3^0}{0!}e^{-3} = 0.04978
\end{equation}


\textbf{Del 3) SSh for $Y>0$ Taxa er der før 10 minutter}

$\lambda = 1/30 * 10 = 1/3$
\begin{equation}
    P(Y > 0) =1 - P(Y=0) = 1 -  \frac{(1/3)^0}{0!}e^{-(1/3)} = 0.28346
\end{equation}


\textbf{Del 4)} Vis at ventetiden, afrundet nedad til helt minuttal, er geometrisk fordelt med $p=1 - e^{1/30}$

Den geometriske fordeling:

Antal forsøg inden succes

\begin{equation}
    pdf = (1-p)^k p
\end{equation}

Først ser vi at:
\begin{equation}
    P(Y = y) = P(X_y > 0, X_{y-1} = 0)
\end{equation}

Altså ventetiden må være sådan at man ikke har fået taxa i sidste minut, men har i dette minut.

Brug nu at en simultan fordeling kan skrives som en betinget fordeling
\begin{align}
    &P(X_y > 0, X_{y-1} = 0) \\
    = &P(X_y > 0 \mid X_{y - 1} = 0)P(X_{y - 1}=0) \\
    \overset{(*)}{=} &(1 - P(X_{y=1}=0))P(X_{y-1}=0)
\end{align}

Vi har i (*) brugt at $P(X_{y} > 0 \mid X_{y - 1}= 0)$ Svarer til $ P(X_{1} > 0)$ som svarer til $1 - P(X_{1} = 0)$

Indsæt nødvendige tal:

\begin{align}
    \left(1 -  \frac{(1/30)^0}{0!}e^{-1/30}\right)\left(\frac{((t - 1)/30)^0}{0!}e^{-(t - 1)/30} \right)    
\end{align}

Vi ser at: $ \frac{(t - 1/30)^0}{0!} = \frac{1}{1} =  1$

Hvilket betyder:

\begin{align}
    &\left(1 -  \frac{(1/30)^0}{0!}e^{-1/30}\right)\left(\frac{((t - 1)/30)^0}{0!}e^{-(t - 1)/30} \right)    \\
    = &\left(1 -  e^{-1/30}\right)\left(e^{-(t-1)/30} \right) \\
    \approx & \left(1 -  e^{-1/30}\right)\left(e^{-1/30\cdot t} \right)
\end{align}

Vi skulle have i den geometriske fordeling: $p = 1 - e^{-1/30}$

\begin{equation}
    (1-p)^k p = (1 - (1 - e^{-1/30}))^t (1 - e^{-1/30})
\end{equation}

Vi forkorter

\begin{equation}
    (1-p)^k p = e^{-1/30\cdot t} (1 - e^{-1/30})
\end{equation}

Vi har vist udtrykket!

\subsubsection{Opgave A}

Lav i klassen!!!

Cykelforsikring fortsat!

\begin{itemize}
    \item udbetaling ved mistet cykel 4000
    \item ssh for cykel stjålet pr. år: $5 \%$
    \item Maks en cykel stjålet om året
    \item forsikring pris 400
\end{itemize}

\textbf{Del 1)}

10 cyklister tegner forsikring:

\begin{equation}
    Y \sim Binomial(n = 10, p=0.05)
\end{equation}

\textbf{Del 2) Udregn Forventet antal stjålne cykler, samt forventet udgift}

\begin{equation}
    \E(Y) = n \cdot p = 10 \cdot 0.05 = 0.5
\end{equation}

Forventet udgift:

\begin{equation}
    \E(Y) \cdot 4000 = 2000
\end{equation}

\textbf{Del 3) SSh for mere end en cykel bliver stjålet}

\textit{Få folk til at opskrive binomial koefficienter osv.}

\begin{equation}
    P(Y>1) = 1 - P(Y =  1) - P(Y = 0) = 0.08613
\end{equation}

\textbf{Del 4) Antag nu  100 cyklister}

\begin{equation}
    Z \sim Binomial(n = 100, p=0.05)
\end{equation}

\begin{equation}
    \E(Z) = 100 \cdot 0.05 = 5
\end{equation}

Forventede indtægter:

\begin{equation}
    400 \cdot 100 = 40000
\end{equation}

Forventede udgifter:

\begin{equation}
    4000 \cdot \E(Z) = 4000 \cdot 5 = 20000
\end{equation}

\textbf{Del 5)Ssh for man udgifter overstiger indtægter}

Udgifer overstiger indtægter når der er 11, som får stjålet sin cykel:

Med binomial (udregnet på com):

\begin{equation}
    P(Z > 10) = 1 - \sum_{i=0}^{10}P(Z=i) = 0.01147
\end{equation}

Med poisson:

$lambda = 100 \cdot 0.05 = 5$

\begin{equation}
    P(Z > 10) = 1 - P(Z\leq 10) = 0.013695
\end{equation}

\textbf{Del 6) Antag nu nu n=200, Ssh udgifter over indtægter}

Dette sker når der er 21 som får stjålet cyklen

$lambda = 200 \cdot 0.05 = 10$

$W \sim Poisson(10)$

\begin{equation}
    P(W>20) = 0.0015882
\end{equation}

\textbf{Del 7)}

Klasse diskussion!!!

\subsubsection{(Optional) Opgave H}


\horizline

\subsection{Øvelse 9}

\textbf{Opgaver: 5.2, 5.3, 5.7, U41.1, U41.2 }

\subsubsection{Opgave 5.2}

\begin{itemize}
    \item X er en kontinuær stok var
    \item $p(x) = \alpha x^{- (\alpha + 1)}$ for $x>1,\alpha > 0 $
\end{itemize}

\textbf{Find fordelingsfunktionen for X}

Vi ved at $p(x) = F'(x)$ Hvis vi skulle finde sandsynligheden for et udfald ville vi bruge tætheden $p(x)$ lad os sige vi ville finde ssh for at $X$ er i intevallet $a$ til $b$: da

\begin{equation}
    \int_{a}^{b}p(x) dx
\end{equation}

Fordelingsfunktionen er kendetegnen ved for intervallet $(-\infty, \infty)$:

\begin{equation}
    \int_{-\infty}^{x} p(x) dx
\end{equation}

Vi har dog intervallet $(1, \infty)$

Vi opskriver integralet:

\begin{equation}
    \int_{1}^{x} \alpha x^{-(\alpha + 1)}    
\end{equation}

\begin{equation}
   \left[\frac{\alpha}{-\alpha + 1 - 1}  x^{-\alpha + 1 - 1} \right]_{1}^{x} = [-x^{-\alpha}]_{1}^{x}=  - x^{-\alpha}  +1
\end{equation}

\subsubsection{Opgave 5.3}

LAV I KLASSEN

fordelingsfunktionen fotr $X$ er givet ved:
\[   
F(x)= 
     \begin{cases}
       0\qquad \for x \leq 0\\
       x/3 \qquad \for 0 < x \leq 1 \\
       (2x -1)/3 \qquad \for 1 < x \leq2\\
       1 \qquad \for x> 2 
     \end{cases}
\] 

\textbf{Find de følgende sandsynligheder}

\begin{equation}
    P(0.5 < X < 1) = F(1) - F(0.5) = \frac{1 - 0.5}{3}=  \frac{1}{6}
\end{equation}

Vi kan ignorere punktsandsynligheden da denne er 0 (i forhold til $\leq$ udtryk i oplæg).

\begin{equation}
    P(1 \leq X <  1.5) = F(1.5) - F(1) = \frac{3 - 1 }{3} - \frac{1}{3} = \frac{1}{3}
\end{equation}

\begin{equation}
    P(2/3 < X < 4/3) = F(4/3) - F(2/3) = \frac{2(4/3) - 1}{3} - \frac{2/3}{3} 
\end{equation}

\begin{equation}
   = \frac{8/3 - \frac{3}{3} - 2/3}{3} = \frac{3/3}{3} = \frac{1}{3}
\end{equation}

\textbf{Redegør for kontinuitet}

Vi viser kontinuæritet via et lille $\delta>0$

Først se om: $F(0 + \delta) \ra 0$ og $ F(0 - \delta )\ra 0$ for $\delta \ra 0$. Man ser at for $x/3$ går mod 0, hvis $x$ er tæt på 0. (Trivielt at se 0 går mod 0 for lille $x$).

Undersøg i en omegn af punktet $x=1$: $F(x \pm \delta) \ra \frac{1}{3}$ for $\delta \ra 0$. Det er klart da: $x/3 \ra \frac{1}{3}, \for x = 1 - \delta$ og $(2x - 1)/3 \ra \frac{1}{3}, \for x = 1 + \delta$

Undersøg i en omegn af punktet $x=2$: 
$F(x \pm \delta) \ra 1$ for $\delta \ra 0$. man ser at $(2(2-\delta)  - 3)/3 \ra 1$ for $\delta \ra 0$. (trivilt at 1 går mod 1)

Kontinuitet er vist. Vi noterer at fordelingsfunktionen overholder at $F: \R \mapsto [0,1]$ og at $F(x) \leq F(x + h), \quad h>0$. Altså den er defineret på hele den reelle akse, samt at den er monotont voksende!

\textbf{Find tæthedsfunktionen for $X$}

Vi differentiere de enkelte udtryk og får:

\begin{equation}
p(x) =
    \begin{cases}
        0 \qquad x \leq 0\\
        \frac{1}{3} \qquad 0 < x \leq 1 \\
        \frac{2}{3} \qquad 1 < x \leq 2 \\
        0 \qquad x > 2
    \end{cases}
\end{equation}

\subsubsection{Opgave 5.7}

LAV I KLASSEN!

\begin{itemize}
    \item 5.1.5 i bogen:
    \begin{equation}
    p(x) = \beta x^{\beta -1}
\end{equation}
    \item $x \in [0,1]$
\end{itemize}



\textbf{Vis at 5.1.5 (i bogen) har middelværdi $\beta / (\beta +1 )$}

Vi behøver ikke at teste om middelværdien eksisterer!

Definition på middelværdi!
\begin{equation}
    E(X) = \int_{-\infty}^{\infty} x p(x) dx < \infty
\end{equation}

\begin{align}
    E(X) = \indefint x \beta x^{\beta -1}  = \indefint \beta x^\beta
\end{align}

Vi ved at $ x$ er koncentrerer på intervallet 0 til 1: $x \in (0, 1)$

\begin{align}
    E(X) = \indefint x \beta x^{\beta -1}  = \indefint \beta x^\beta
\end{align}


Vi har her et uendeligt integrale, men $x$ er koncentreret på en mindre mængde. Vi bruger at $P(\emptyset)=0$ og at vi må splitte integralerne op (indskudssætningen):

\begin{equation}
    \int_a^c f(x) dx = \int_a^b f(x) dx + \int_b^c f(x)dx,\qquad  \text{(Indskudssætningen)}
\end{equation}

Vi ser at integralerne i intervallet $(-\infty, 0[$ og $]1, \infty)$ er lig $0$.

\begin{equation}
    E(X) \int_{0}^{1} \beta x^\beta = \left[ \frac{\beta}{\beta + 1} x^{\beta + 1}  \right]_{0}^{1} = \frac{\beta}{\beta + 1}
\end{equation}


\textbf{Vi finder variansen}

\begin{equation}
    \Var(X) = \E(X^2) - \E(X)^2
\end{equation}

\begin{equation}
    E(X^2) =\indefint x^2 \beta x^{\beta -1}  = \indefint \beta x^{\beta + 1}
\end{equation}

Analogt med før

\begin{equation}
    \E(X^2) = \left[ \frac{\beta}{\beta + 2} x^{\beta + 2}  \right]_{0}^{1} = \frac{\beta}{\beta + 2}
\end{equation}

Variansen findex:

\begin{equation}
    \Var(X) = \frac{\beta}{\beta + 2} - \left( \frac{\beta}{\beta + 1} \right)^2
\end{equation}

kan evt. forkortes

\subsubsection{Opgave U41.1}

\begin{itemize}
    \item $X, Y \sim Uni(0,1)$
    \item den uniforme fordeling er kontinuær
\end{itemize}

\begin{equation}
    \E(X) = \E(Y) = \frac{1}{2}(a+b) = \frac{1}{2}(1 + 0) = \frac{1}{2}    
\end{equation}

Brug sætning 6.4.2 - man kan splitte forventinger op.


\textbf{find $\E(6X + 32Y)$}

\begin{equation}
    \E(6X + 32Y) = \frac{6 + 32}{2} = 19
\end{equation}

\textbf{Find $\E(X^3)$ og $\E(X^3 + Y^3)$}

\begin{equation}
    \E(X^3) = \int_0^1 x^3 p(x) = \left[ \frac{1}{4} x^4\right]_0^1 = \frac{1}{4}
\end{equation}

Vi har derfor selvfølgelig $\E(X^3 + Y^3) = 2 \cdot \frac{1}{4} = \frac{1}{2}$

\textbf{Find $\Var(X) = \E(X^2) - [\E(X)]^2$}

Vi ved at $\E(X)^2 = \lp\frac{1}{2}\rp^2 = \frac{1}{4}$

\begin{equation}
    \E(X^2) = \int_{0}^{1} x^2 p(X) = \lsp \frac{1}{3} x^3\rsp_0^1 = \frac{1}{3} 
\end{equation}

Varians:

\begin{equation}
    \Var(X) = \frac{1}{3} - \frac{1}{4} = \frac{1}{12}
\end{equation}

\textbf{Find tæthed for $Z = X - \frac{1}{2}$}

\begin{equation}
    p(z) = 1, \quad z \in[-0.5 , 0.5]
\end{equation}

\textbf{Find $\E(Z)$}

Brug sætning 5.2.5. lineær transformation.
\begin{equation}
    \E(Z) = \E\lp X - \frac{1}{2}\rp =   \E(X) - \frac{1}{2} = 0
\end{equation}

\textbf{Find $F(Z)$}

\begin{equation}
    F(Z) = z - \frac{1}{2}, \qquad z\in[-0.5 , 0.5]
\end{equation}

\subsubsection{Opgave U41.2}

\begin{itemize}
    \item stokastisk variabel $X$
    \item $p(x) = \lambda \exp (-\lambda x)$
\end{itemize}

\textbf{Del 1 - A) Opskriv fordelingsfunktionen for $X$ og vis at $Y=F(X)$ er ligefordelt på $[0,1]$}

\begin{equation}
    F(x) = 1 - \exp(- \lambda x)
\end{equation}

Vis at $Y=F(X)$ er ligefordelt på $[0,1]$

\begin{equation}
    P(Y \leq y) = P(F(X) \leq y) = P(X \leq F^{-1}(y)) = P(X \leq x) = F(X) = y
\end{equation}

\begin{equation}
    x = F^{-1} (y) =  \ln \lp \frac{1}{1- \lambda } \rp / \lambda
\end{equation}

$t(X) = F(X) = y$ bruges i sidste led af ligningen!

Vi ser at $P(Y \leq y) = y$ Hvor vi ved at $y$ er fordelingsfunktionen for en uniform fordeling! 

\textbf{Del 2)}


\horizline

\subsection{Øvelse 10}

\textbf{12/10/2018, Opgaver: 5.1, 5.5, 5.13, 5.15,  U41.3 og U41.4}

\subsubsection{Opgave 5.1}

\begin{itemize}
    \item $X \sim exponential(\lambda)$
    \item pdf: $\lambda e^{-\lambda x}$
\end{itemize}

\textbf{Find $P(X>x)$, for alle $x>0$}

Vi ved at fordelingsfunktion $F(x)$ svarer til $P(X<x)$ hvilket betyder at $P(X>x) = 1 - F(x)$.

\textit{Kommentar: vi bruger lille $x$ i fordelingsfunktionen. hvorfor? fordi det er en funktion der tager et tal (en realisation) af $X$}

Vi kan se på wikipedia at exponential fordelingens fordelingsfunktionen CDF er:

\begin{equation}
    F(x) =  1 - e^{\lambda x}
\end{equation}

Så vi har at:

\begin{equation}
    P(X>x) = 1 - (1 - e^{\lambda x}) = e^{\lambda x}
\end{equation}

\textbf{SSH $P(1 < X < 2)$ ,hvor $\lambda = 1$}

brug (hvor lambda er 1):
\begin{equation}
    F(x) =  1 - e^{1 x}
\end{equation}

\begin{equation}
    P(1 < X < 2) = F(2) - F(1) = (1 - e^2) - ( 1- e^1) = 0.2325
\end{equation}

\subsubsection{Opgave 5.5}

Lav i klassen!

\begin{itemize}
    \item Laplace-fordelingen
    \item defineret på hele $\R$
    \item funktionsforskrift:
    \begin{equation}
        f(x) = \frac{1}{2}e^{-\lvert x \rvert}, \qquad x \in \R
    \end{equation}
\end{itemize}

\textbf{Find fordelingsfunktionen $F$}

Fordelingsfunktionen er: $F(k) = \int_{-\infty}^{k} f(x) dx $

Vi ser, vi må skære integralet op i to dele på grund af normerings operatoren på $x$.

Først $x < 0$
\begin{equation}
    F(a) = \int_{-\infty}^a \frac{1}{2}e^ {x} = \lsp \frac{1}{2} e^{x} + k \rsp_{-\infty}^{a} = \lp \frac{1}{2} e^{a} + k \rp - \lp \frac{1}{2} e^{-\infty} + k\rp = \frac{1}{2} e^{a}
\end{equation}

Nu $x\geq 0$
\begin{equation}
    F(a) = \int_{-\infty}^0 \frac{1}{2}e^ {x} +  \int_{0}^a \frac{1}{2} e^{-x} = \frac{1}{2} + \lsp \frac{1}{-1} \frac{1}{2} e^{- x}  \rsp_{0}^{a}  = \frac{1}{2} + \lsp -\frac{1}{2} e^{- x}  \rsp_{0}^{a} 
\end{equation}

\begin{equation}
    F(a) = \frac{1}{2} +  \lp - \frac{1}{2} e^{-a} \rp - \lp - \frac{1}{2} e^{0} \rp = \frac{1}{2} + \frac{1}{2} - \frac{1}{2} e^{-a} = 1 - \frac{1}{2} e^{-a} 
\end{equation}

Vi kan opskrive fordelingsfunktionen!

\begin{equation}
    F(x) =
    \begin{cases}
    &\frac{1}{2} e^{x}, \qquad x<0 \\
   &1 - \frac{1}{2} e^{-x}, \qquad  x\geq 0 \\
    \end{cases}
\end{equation}

\textbf{Del 2) Find middelværdi}

Vi behøver ikke at vise middelværdi og varians eksisterer!

\begin{equation}
    \E(X) = \int_{-\infty}^{\infty} x p(x) dx
\end{equation}

Vi splitter intergralet op i intervallerne $(-\infty,0)$ og $[0, \infty)$:


(man har her brugt reglen for partiel integration - kig Thomas note/formelsamling) $f(x) = exp(x), g(x) = x$:

for integralet i intervallet $(-\infty,0)$:

\begin{equation}
    \int \frac{1}{2} x e^{x} dx = \frac{1}{2}  (x - 1) e^{x}
\end{equation}

for integralet i intervallet $[0, \infty)$

\begin{equation}
   \int \frac{1}{2} x e^{-x} dx=  - \frac{1}{2}  (x + 1) e^{-x} 
\end{equation}

vi ved at:

\begin{equation}
    \int_{-\infty}^{\infty}  x \frac{1}{2} e^{- \lvert x \rvert} dx = \int_{-\infty}^{0} \frac{1}{2} x e^{x} dx + \int_{0}^{\infty} \frac{1}{2} x e^{-x} dx
\end{equation}

Vi sætter integralernes grænser ind i stamfunktioner udledt ovenfor:


\begin{align}
    \int_{-\infty}^{0} \frac{1}{2} x e^{x} dx = \lp \frac{1}{2}  (0 - 1) e^{0} \rp - \lp \frac{1}{2}  (- \infty - 1) e^{- \infty} \rp = -\frac{1}{2} - 0 = -\frac{1}{2} 
\end{align}

\begin{equation}
    \int_{0}^{\infty} \frac{1}{2} x e^{-x} dx = 
    \lp - \frac{1}{2}  (\infty + 1) e^{-\infty} \rp - \lp  - \frac{1}{2}  (0 + 1) e^{0}\rp = 0 + \frac{1}{2} = \frac{1}{2}
\end{equation}

Så vi har at:

\begin{equation}
    \E(X) = - \frac{1}{2} + \frac{1}{2} = 0
\end{equation}

\textbf{Find variansen $\Var(X) $}

VI ved at $\E(X) = 0$ det betyder at $Var(X) = \E(X^2)$. \textit{Husk på formlen for varians.}


\begin{equation}
    \Var(X) = \E(X^2) - \E(X)^2 = \E(X^2) =  \int_{-\infty}^{\infty} x^2 p(x) dx
\end{equation}

vi deler igen integralet op. og bruger reglerne for partiel integration. \textit{Vi ender med at få integralet fra før som et del element}.

I intervallet $(-\infty, 0)$:

\begin{equation}
    \frac{1}{2}\int_{-\infty}^{0}  x^2 e^x dx =  \lsp \lp \frac{1}{2}x^2 - x + 1\rp e^{x} \rsp_{-\infty}^{0} = 1 - 0 = 1
\end{equation}

I intervallet $[0, \infty)$:

\begin{equation}
    \frac{1}{2} \int_{0}^{\infty}  x^2 e^{-x} dx = \lsp - \lp \frac{1}{2} x^2 + x + 1\rp e^{-x}\rsp_{0}^{\infty} = 0 - (-1) = 1
\end{equation}

Vi har at:

\begin{equation}
    \Var(X) = \frac{1}{2}\int_{-\infty}^{0}  x^2 e^x dx +  \frac{1}{2} \int_{0}^{\infty}  x^2 e^{-x} dx =  1 + 1 = 2
\end{equation}

\subsubsection{Opgave 5.13}

LAV I KLASSEN

\begin{itemize}
    \item $X$ er en kontinuær stokastisk variabel i intervallet $(a, b)$
    \item $X$ har en kontinuer sandsynlighedstæthed $p$ på $(a,b)$
\end{itemize}

Vi bruger sætning $5.4.1$

\begin{equation}
    q(y) = 
    \begin{cases}
        p(t^{-1}(y)) \lvert \frac{d}{dy} t^{-1}(y) \rvert , \qquad &y \in (v,h) \\
        0, \qquad &y \notin (v,h)
    \end{cases}
\end{equation}

hvor $v = \inf t(I), h = \sup t(I)$ og $I$ er intervallet $(a,b)$

Til de kommende opgaver kan der siges generalt at: $x = t^{-1} (y) $

Og der skippes ofte $(y)$ fra notation, således at: $\frac{d}{dy}t^{-1}(y)$ bliver til  $\frac{d}{dy}t^{-1}$

\textbf{Del 1) Find tætheden for $exp(X)$}

vi har vores transformation givet som $t = exp(\cdot)$ som implicerer at $t^{-1} = ln(\cdot)$.

Vi finder den afledte af vores inverse transformation

\begin{equation}
    \frac{d}{dy} t^{-1} (y) = \frac{d}{dy} ln (y) = \frac{1}{y}
\end{equation}

Vi opskriver:

\begin{equation}
    q(y) = 
    \begin{cases}
    p( ln(y) ) \cdot \lvert \frac{1}{y} \rvert, \qquad &y \in (e^{a}, e^{b})\\
    0, \qquad &\ellers
    \end{cases}
\end{equation}

Man ser at faktisk $y \in \R_{+} \forall y \in Y$, hvilket betyder, man ikke ville behøve at lave normeringstegnet

\textbf{Antag resten af opgaven at $a>0$}

\textbf{Del 2) Find tætheden for $\sqrt X$}

Vi finder transformationens inverse $t^{-1} = y^2$. og herfra den afledte: $\frac{d}{dy} t^{-1} = 2 y$. 

\begin{equation}
    q(y) = 
    \begin{cases}
        p(y^2) \cdot 2y ,&y \in (\sqrt{a}, \sqrt{b}) \\
        0, &\ellers
    \end{cases}
\end{equation}

Vi bemærker at y ikke kan antage værdier under 0, grundet $a>0$.

\textbf{Del 3) Find tætheden for $\frac{1}{X}$}

Vi finder transformationens inverse $t^{-1} = \frac{1}{y} $

den inverse transformations afledte:
$\frac{d}{dy}t^{-1} =-\frac{1}{y^2}$. \textit{Det huskes at man tager den absolutte værdi $\implies$ man fjerner minuset} 

\begin{equation}
    q(y) = 
    \begin{cases}
    p \lp \frac{1}{y} \rp \frac{1}{y^{2}} &y \in \lp \frac{1}{a}, \frac{1}{b}  \rp \\
    0, &\ellers
    \end{cases}
\end{equation}

\textbf{Del 4) Find tætheden for $X^2$}

Vi finder den inverse transformation: $t^{-1} = \sqrt{y}$

Den afledte af den inverse transformation:
$\frac{d}{dy} t^{-1} = \frac{1}{2} y^{-1/2}$

\begin{equation}
    q(y) = 
    \begin{cases}
    p(\sqrt{y})\frac{1}{2} y^{-1/2}, &(a^2, b^2) \\
    0, &\ellers
    \end{cases}
\end{equation}

\subsubsection{Opgave 5.15}

\begin{itemize}
    \item $X \sim N(\mu, \sigma^2)$
    \item $Y = \exp(X) $
\end{itemize}

\textbf{Del 1) Find sandsynlighedstætheden for $Y$}

tæthedsfunktionen for normal fordlingen:
\begin{equation}
    p(x) = \frac{1}{\sqrt{2 \pi \sigma^2}} exp \lp - \frac{(x-\mu)^{2}}{2\sigma^2} \rp
\end{equation}

Vi finder den inverse transformation: $x = t^{-1}(y) = ln(y)$

Den inverse transformations afledte mht y: $\frac{d}{dy} t^{-1}(y) = \frac{1}{y}$

læg mærke til $\ln(y)$ ind i udtrykket

\begin{equation}
    q(y) =
    \begin{cases}
         \frac{1}{\sqrt{2 \pi \sigma^2}} exp \lp - \frac{(\ln (y)-\mu)^{2}}{2\sigma^2} \rp  \cdot \frac{1}{y}, &y \in (0, \infty) \\
        0, & \ellers
    \end{cases}
\end{equation}

\textbf{Del 2) Vis at $Y = \beta X$ er scala invariant}

Vi finder den inverse transformation $x = t^{-1} (\beta y) = ln(\beta y)$. Vi husker at: $ln(\beta y) = ln(\beta) + \ln(y) $


Den inverse transformations afledte mht y: 

\begin{equation}
    \frac{d}{dy} t^{-1} (\beta y) = \frac{d}{dy} ln(y) + ln(\beta) = \frac{1}{y}
\end{equation}

Vi indsætter de fundne værdier

\begin{equation}
    q(y) =
    \begin{cases}
         \frac{1}{\sqrt{2 \pi \sigma^2}} exp \lp - \frac{(\ln (\beta) + \ln(y)-\mu)^{2}}{2\sigma^2} \rp  \cdot \frac{1}{y}, &y \in (0, \infty) \\
        0, & \ellers
    \end{cases}
\end{equation}

Vi ser den transformerede fordeling stadig er logaritmisk normalfordelt!

\textbf{Del 3}

Vi husker en detalje: $\int_{-\infty}^{\infty} p(x) dx = 1$. Dette betyder, at hvis vi kan skabe det ovenstående integrale, og få det resterende ud foran integralet, så har vi fundet resultatet!

Husk $q(y)$ er 0 når ikke $y\in(0, \infty)$
\begin{equation}
    \int_{-\infty}^{\infty} q(y) dy = \int_{0}^{\infty} q(y) dy  
\end{equation}


\begin{equation}
    \E(Y) = \int_{0}^{\infty} y q(y) dy  
\end{equation}

\begin{equation}
    \E(Y) = \int_{0}^{\infty} y \frac{1}{\sqrt{2 \pi \sigma^2}} exp \lp - \frac{(\ln (y)-\mu)^{2}}{2\sigma^2} \rp  \cdot \frac{1}{y}
\end{equation}

Vi ser at $y$ går ud med $\frac{1}{y}$ Vi indsætter $\mu = 0, \sigma = 1$ som angivet i opgaveteksten.

\begin{equation}
    \E(Y) = \int_{0}^{\infty}  \frac{1}{\sqrt{2 \pi}} exp \lp - \frac{(\ln (y))^{2}}{2} \rp
\end{equation}

Det bagerste udtryk manipuleres:

\begin{equation}
    exp \lp -\frac{ln(y)^2}{2}\rp =    exp \lp -\frac{ \ln(y) \ln(y) }{2}\rp = exp \lp -\frac{1}{2} \rp exp \lp \ln(y) \ln(y) \rp 
\end{equation}

Går i stå her!

\subsubsection{Opgave U41.3}

\begin{itemize}
    \item $X$ er ligefordelt på $(0,1)$.
\end{itemize}

\textbf{Del 1) $S = \1_{(0,0.25)}$ Find $P(S=1)$}

\begin{equation}
    P(X \in (0, 0.25)) = F(0.25) = \frac{1}{4}
\end{equation}

\textbf{Del 2) $S=  \1_{(0, p)}$. Find $P(S=1)$}

\begin{equation}
    P(X \in (0, p)) = F(p) = p
\end{equation}

\textbf{Del 3) Beskriv hvordan du kan simulere en trækning fra en stokastisk variabel $Y$}

$P(Y = 1) = \frac{1}{9}$ og $P(Y=2) = \frac{8}{9}$

Vi ved at fordelingsfunktionen $F: \R \mapsto [0,1]$. Det betyder at den inverse $F^{-1}: [0,1] \mapsto \R$. Overvej dette.

Vi kan altså sample fra intervallet $[0,1]$ og mappe det til en real værdi gennem den inverse fordelingsfunktion:

Vi har implicit givet fordelingsfunktionen ovenfor:

\begin{equation}
    F(y) = 
    \begin{cases}
        0, &y < 1 \\
        \frac{1}{9}, & 1 \leq y < 2 \\
        1, &2 \leq y
    \end{cases}
\end{equation}

\textit{Tegn fordelingsfunktionen og den inverse fordelingsfunktion}

Det betyder at vi kunne sample således:

$Y = 1$ når $x \in \lp 0, \frac{1}{9} \rp$.

$Y =2$ når $x \in \lp \frac{1}{9}, 1 \rp $

\subsubsection{Opgave U41.4}

\begin{itemize}
    \item $X \sim N(\mu, \sigma^2)$
\end{itemize}

\textbf{Del 1) Hvad er fordelingen af $Y = (X -\mu) / \sigma$}

Denne er let, da dette bare er en tilbage skalering af normalfordelingen! Dvs. en standard normalfordeling:

\begin{equation}
    Y \sim N(0,1)
\end{equation}

\textbf{Del 2) Hvad er fordelingen af $Z = (X -\mu)^{2} / \sigma^2 $}

Vi ser dette er:

\begin{equation}
     Z = \frac{(X -\mu)^2}{\sigma^2} = \lp \frac{X - \mu }{\sigma} \rp^2
\end{equation}

Dette svarer altså til den kvadrerede standard normalfordeling: $\chi^2$-fordelingen.
\horizline

\subsection{Øvelse 11}

\textbf{22/10/2018, opgaver: U43.1.1, U43.1.2, U43.1.3 U43.1.4}

\subsubsection{Opgave U43.1.1}

\begin{itemize}
    \item $X, Y$ er ligefordelt på $A$
    \item $A = [0,1] \times [0,1]$
    \item $p(x,y) = \1_A(x,y)$
\end{itemize}

Tegn 2-D sketch af definitionsmængden

\textbf{Del 1) Udregn $P(X < 0.1, Y < 0.6)$}

\begin{align}
        P(X < 0.1, Y < 0.6) &= \int_0^{0.6} \int_0^{0.1} \1_A(x, y) dx dy \\
        &= \int_{0}^{0.6} [x]_{0}^{0.1} \1_A(y) dy \\
        &= [x]_{0}^{0.1} [y]_{0}^{0.6} \\
        &=(0.1 - 0) \cdot (0.6 - 0) \\
        &= 0.1 \cdot 0.6 = 0.06
\end{align}

\textbf{Del 2) Udregn $P(0.25< X < 0.75, 0.4 < Y < 0.6$}

Analogt med før - opskrivningen er ikke nødvendig:

\begin{equation}
    P(0.25 < X < 0.75, 0.4 < Y < 0.6) = 0.5 \cdot 0.2 = 0.1
\end{equation}

\textbf{Del 3) Udregn $P(X < 0.1)$}

Her bruges at man kan integrere irrelevante variable ud: \textbf{sætning 6.1.3}

\begin{equation}
    q(x) = \int_{\R} p(x,y) dy
\end{equation}

dvs:

\begin{equation}
    q(x) = \1_{[1,0]}(x)
\end{equation}

Vi finder nu det ønskede udtryk
\begin{equation}
    P(X < 0.1) = \int_{0}^{0.1} \1_{[0,1]}(x) = [x]_{0}^{0.1} = 0.1
\end{equation}

\textbf{Del 4) Find den marginale fordeling for $X$}

Igen bruges sætning $6.1.3$

\begin{equation}
    q(x) = \int_{\R} p(x,y) dy
\end{equation}

dvs:

\begin{equation}
    q(x) = \1_{[0,1]}(x)
\end{equation}

Altså vi svarede indirekte på det problem før!

$p_{x}(x) = \1_{[0,1]}(x)$ og lige så $p_y (y) = \1_{[0,1]}(y)$ Vi ser altså nu at $p(x,y) = p_x (x) \cdot p_y (y)$

\subsubsection{Opgave U43.1.2}

\begin{itemize}
    \item $X,Y$ er uafhængige
    \item $X, Y$ er ligfordelte på intervallet $[0,1]$
    \item $Y* = 2Y$
\end{itemize}

\textbf{Find $E(Y*), V(Y*)$}



Brug sætning \textbf{6.3.2} som viser at hvis $X \independent  Y \implies X \independent \phi(Y)$

Vi har uafhængighed hvilket implicerer:

\begin{equation}
    p(x,y*) = p(x) p(y*)
\end{equation}

Nu integreres $X$ ud:

\begin{equation}
    p(y*) = p(y*) \int_{\R}p(x) dx = p(y*)
\end{equation}

Vi finder den forventede værdi:

$2$ er den øvre grænse, $0$ er den nedre grænse for $Y$. 

\begin{equation}
    \E(Y*) = 2 \cdot \E(Y) = 2\cdot 0.5 = 1
\end{equation}

Variansen findes ved: $\Var(aX) = a^2 \Var(X)$.

\begin{equation}
    \Var(Y) = \frac{1}{12}(0 - 1)^2 = \frac{1}{12}
\end{equation}

\begin{equation}
    \Var(Y*) = 2^2 \Var(Y) = 4 \cdot \frac{1}{12} = \frac{1}{3}
\end{equation}

\textbf{Del 2) Tætheden for $Y*$}

Tætheden er:

tætheden for en uniform (kontinuær) distribution er: $p(x) = \frac{1}{b - a}\1_{x \in [a, b]}(x)$

Vi bruger dette:

\begin{equation}
    p(y*) = \frac{1}{2-0}\1_{x \in [0,2]}(y*)
\end{equation}

\textbf{Del 3) $Z = X + Y*$ Find tætheden for $Z$, $q(z)$}

Vi bruger \textbf{korollar 6.3.2} (få en studerende til at læse op).

\begin{equation}
    q(z) = \int_{-\infty}^{\infty} p_{1}(x)p_{2}(z - x) dx
\end{equation}

\begin{equation}
    p_x(x)p_{y*}(z - x) = \1_{[0,1] \times [0,2]}\frac{1}{2}(x)(z-x) = \frac{1}{2}(xz - x^2)
\end{equation}

Nu integreres denne:

\begin{align}
    q(z) &= \int_{\R} \frac{1}{2}(xz - x^2) dx \\
    &= \frac{1}{2} \int_{\R} (xz - x^2) dx \\ 
    &= \lsp \frac{1}{2}\frac{1}{2} x^2 z - \frac{1}{2}\frac{1}{3}x^3\rsp_{0} ^{1} = \frac{1}{4}z - \frac{1}{6} 
\end{align}

NOGET ER GALT

\subsubsection{Opgave U43.1.3}

\begin{itemize}
    \item $X, Y \in [5,10] \times [3,7]$
    \item $p(x,y) = \frac{1}{20}\1_{[5,10] \times [3,7]} (x,y)$
\end{itemize}

Skitser definition mængden.

\textbf{Del 1) Forklar hvorfor $p(x,y)$ er en tæthedsfunktion}

notér at $(10 - 5) \times (7 - 3) = 20$, således at den samlede areal under kurven er 1.

\textbf{Find $P(6 \leq X \leq 10, 4 \leq Y \leq 6)$}

\begin{align}
    P(6 \leq X \leq 10, 4 \leq Y \leq 6) &= \int_{6}^{10}\int_{4}^{6} \frac{1}{20}\1_{[5,10] \times [3,7]} (x,y) dy dx \\
    &= \frac{1}{20} \int_{6}^{10}\int_{4}^{6} \1_{[5,10] \times [3,7]} (x,y) dy dx \\
    &= \frac{1}{20} \int_{6}^{10} \1_{[5,10]}(x) \lsp y \rsp_4^6 dx \\
    &= \frac{1}{20} \int_{6}^{10} \1_{[5,10]}(x) (6 - 4) dx \\
    &= \frac{2}{20}\int_{6}^{10} \1_{[5,10]}(x) dx \\
    &= \frac{2}{20} \lsp x \rsp_{6}^{10} \\ 
    &=  \frac{2}{20} (10-6) = \frac{8}{20}
\end{align}

\textbf{Del 3) Find de marginale fordelinger}

\begin{equation}
    p(x) = \frac{1}{20}\int_{3}^{7} \1_{[5,10] \times [3, 7]} (x,y) dy = \frac{4}{20} \1_{[5,10]} (x,y) 
\end{equation}

Omvendt for $Y$:

\begin{equation}
    p(y) = \frac{5}{20} \1_{[3,7]} (x,y)
\end{equation}

\textbf{Del 4) Find $\E(X)$}

For en ligefordeling har man middelværdi ved (a og b er enderne):
\begin{equation}
    \E(X) = \frac{a + b}{2}
\end{equation}

Vi bruger dette

\begin{equation}
    \E(X) = \frac{5 + 10}{2} =7.5
\end{equation}

\subsubsection{Opgave U43.1.4 }

\begin{itemize}
    \item $X,Y  \in [0,\infty)$
    \item $p(x,y) = 6 \exp(-2x -3y)$
\end{itemize}

Praktisk at vide:

\begin{equation}
    \int \exp(-b x) dx = -\frac{\exp(-b x)}{b}
\end{equation}

\textbf{Del 1 - a) find $P(X \leq 2, Y \leq 4)$}

\begin{align}
    P(X \leq 2, Y \leq 4) &= \int_0^2 \int_0^{4} 6 \exp(-2x -3y) dy dx \\
    &= \int_0^2 \int_0^{4} 6 \exp(-2x)\exp(-3y) dy dx \\
    &= 6 \int_0^2  \exp(-2x) \lp \int_0^{4}  \exp(-3y) dy \rp dx \\
    &=   6 \int_0^2  \exp(-2x) \lp \lsp -\frac{\exp(-3y)}{3}\rsp_0^4 \rp dx 
\end{align}


Vi løser det indre problem:

\begin{align}
    \lsp -\frac{\exp(-3y)}{3}\rsp_0^4 &= \lp-\frac{\exp(-12)}{3}\rp - \lp -\frac{1}{3} \rp \\
    &= \frac{1}{3} + \frac{\exp(-12)}{3} \\ 
    &= \frac{1 - \exp(-12)}{3}
\end{align}

Vi indsætter dette!

\begin{align}
    6 \int_0^2  \exp(-2x) \lp \frac{1 - \exp(-12)}{3} \rp dx &= 6 \lp\frac{1 - \exp(-12)}{3} \rp \int_0^2  \exp(-2x) dx \\ 
    &= 6 \lp \frac{1 - \exp(-12)}{3} \rp \lsp -\frac{\exp(-2x)}{2}\rsp_0^2
\end{align}

Vi udregner det inderste:

\begin{align}
    \lsp -\frac{\exp(-2x)}{2}\rsp_0^2 &= \lp - \frac{\exp(- 2\cdot2 )}{2}\rp - \lp -\frac{1}{2} \rp \\
    &=  \frac{1}{2} - \frac{\exp(- 4 )}{2} \\ 
    &= \frac{1 - \exp(-4)}{2}
\end{align}

Dette indsættes:

\begin{align}
    6 \lp \frac{1 - \exp(-12)}{3} \rp \lp \frac{1 - \exp(-4)}{2} \rp = \lp 1 - \exp(-12) \rp \lp 1 - \exp(-4) \rp
\end{align}

\textbf{Del 1 - b) find $P(X > 1, Y \leq 3)$}

\textbf{Lav i klassen!} Efter samme opskrift som ovenfor:

Resultat:

\begin{equation}
    P(X>1, Y\leq 3) = \exp(-2) \lp 1 - \exp(9) \rp
\end{equation}

\textbf{Find de marginale fordelinger $p_y(y), p_x(x)$}

Man integrere den ene variabel ud: dvs, integrer $y$ ud, hvis man ønsker at finde $p_x (x)$, og vice versa.

\begin{align}
    p_y(y) = \int_{\R} p(x,y) dx &=  \int_{\R} 6 \exp(-2x -3y) dx \\
    &= 6 \exp(-3y) \int_{\R} \exp(-2x) dx \\ &= 6 \exp(-3y) \cdot \frac{1}{2} \\
    &= 3 \exp(-3y)
\end{align}

Hvor man har udnyttet at $\int_{\R} \exp(-2x) dx = \frac{1}{2}$

\textbf{Lad klassen lave anden halvdel!}

Resultatet er analogt for $Y$, bare hvor

\begin{equation}
    p_x (x) = \int_{\R} p(x,y) dy = \frac{1}{3} \cdot 6 \exp(-2x) = 2 \exp(-2x)
\end{equation}

\textbf{Del 3) Find fordelingsfunktionen for $X$}

Jeg udskifter $x$ med $a$ for ikke at gøre notationen forvirrende!

\begin{align}
    F(a) \int_{0}^{a} p(x) dx &= \int_{0}^{a} 2 \exp(-2x) dx \\
    &= 2 \lsp -\frac{-\exp(-2x)}{2}\rsp_{0}^{a} \\
    &= 2 \lp 1 \rp - 2\lp- \frac{\exp(-2a)}{2} \rp \\
    &= 1 - \exp(-2a)
\end{align}

Dette indsættes:

\begin{equation}
    F(x) =  1 - \exp(-2x)
\end{equation}

Medianen findes

\begin{align}
    0.5 = 1 - \exp(-2x)  
    &\lra 0.5 = \exp(-2x) \\
    &\lra ln(0.5) = -2x \\
    &\lra -\frac{ln(0.5)}{2} = x
\end{align}

\textbf{Del 4) Vis uafhængighed}

Vi ser at $p(x)p(y) = p(x,y)$ -  Dette er sætning \textbf{6.2.1}

\begin{equation}
    \lp 2 \exp(-2x) \rp \lp 3 \exp(-3y )\rp = 6 \exp(-2x - 3y)
\end{equation}

\horizline

\subsection{Øvelse 12}

\textbf{26/10/2018, opgaver: U43.2.1, U43.2.2, U43.2.3, U43.2.4, U43.2.5 fra bogen: 6.4, 6.21}

\subsubsection{Opgave U43.2.1}

\begin{itemize}
    \item $A= \{x,y \mid x^2 + y^2 \leq 1$
    \item $p(x,y) = \frac{1}{\pi}\1_{A} (x,y)$
\end{itemize}

\textbf{Del 1) Tegn $p(x,y)$}

- Tegn på tavlen en tredimensionel enhedscirkel. Højden: $\frac{1}{\pi} \approx \frac{1}{3}$

\textbf{Del 2) Find den marginale tæthed for $X$}

For at finde den marginale tæthed skal man integrere $Y$ ud af udtrykket $p(x,y)$

Først noteres at:

\begin{equation}
    X^2 + y^2 \leq 1 \lra y \leq \sqrt{1 - x^2} 
\end{equation}

\begin{align}
    p_X (x) &= \int_{\R} \frac{1}{\pi} \1_A (x,y) dy \\
    &=  2 \frac{1}{\pi} \int_0^{\sqrt{1 - x^2}} \1_A (x,y) dy \\
    &= 2 \frac{1}{\pi} \lp\lp\sqrt{1- x^2}\rp - \lp0\rp\rp \\
    &= \frac{2\sqrt{1 - x^2}}{\pi} 
\end{align}


2-tallet kommer fra at y både kunne have været positivt og negativt!!!

\textbf{Del 2) Find den marginale tæthed}

Kig på github!

\subsubsection{Opgave U43.2.2}

\begin{itemize}
    \item $A = \lcp x,y \mid x\in[1,2], y\in[1,2] \rcp$
    \item $p(x,y) = \1_A p(x,y)$
\end{itemize}

\textbf{Del 1) Tegn $p(x,y)$}

Gør på tavlen. 3-dimensionel tegning.

\textbf{Del 2) Find de marginale tætheder $p_Y, p_X$}

\begin{equation}
    p_X (x) = \int_{\R} \1_A(x,y) dy = \int_{1}^{2} \1_A(x,y) dy = \1_{[1,2]}(x) \int_{1}^{2} \1_A(y) dy = \1_{[1,2]}(x)
\end{equation}

Analogt for $p_Y(y)$ =

\begin{equation}
    p_Y(y) = \1_{[1,2]}(y)
\end{equation}


\textbf{Del 3) Definér $Z= X + Y$. Find $\E(Z), \Var(Z)$}

Vi ser at $X \independent Y$

Det implicerer at:

\begin{equation}
    \E(Z) = \E(X) + \E(Y) = \frac{3}{2} + \frac{3}{2} = 3     
\end{equation}

Hvor man har udnyttet at $\E(X) = \E(Y) = \frac{a + b}{2} = \frac{1 +2 }{2} = \frac{3}{2}$. Man husker at $\frac{a + b}{2}$ er middelværdien for den uniforme fordeling!

Variansen findes:

VI husker de er uafhængige hvilket gør vi kan sige - Fundet på wikipedia - generelt er wikipedia bedre til egenskaber end bogen - bogen er meget rodet opbygget:

\begin{equation}
    \Var(Z) = \Var(X) + \Var(Y) = \frac{1}{12} + \frac{1}{12} = \frac{1}{6}
\end{equation}

Vi finder variansen af $X$:
\begin{equation}
    \Var(X) = \frac{1}{12}(a-b)^2 =\frac{1}{12}
\end{equation}

\textbf{Find tætheden $q(z)$ for $Z$}

Vi bruger Korollar 6.3.2
\begin{align}
    p(x, x - z) &= \1(1 \leq x \leq 2) \1(1 \leq z- x \leq 2) \\
    &= \1(2 \leq z \leq 3) \1(1 \leq x \leq z-1) \\
    &+ \1(3 \leq z \leq 4) \1(z-2 \leq x  \leq 2)
\end{align}

Kig github for illustration!

Vi bruger dette:

\begin{align}
    q(z) &= \int_\R \1(2 \leq z \leq 3) \1(1 \leq x \leq z-1) +  \1(3 \leq z \leq 4) \1(z-2 \leq x  \leq 2) dx \\
    &= \int_\R \1(2 \leq z \leq 3) \1(1 \leq x \leq z-1) dx + \int_{\R} \1(3 \leq z \leq 4) \1(z-2 \leq x  \leq 2) dx \\
    &=  \1(2 \leq z \leq 3) \int_{1}^{z-1} \1 dx + \1(3 \leq z \leq 4) \int_{z-2}^{2} \1 dx 
\end{align}

Indsætter i stamfunktionen giver:

\begin{equation}
   q(z) = \1(2 \leq z \leq 3)(z-2) + \1(3 \leq z \leq 4)(4 - z) 
\end{equation}

\textbf{del 4) Benyt $q(z)$ til at udregne $\E(z)$}

\begin{align}
    \int_\R z q(z) dz &= \int_{\R}\1(2 \leq z \leq 3)(z^2-2z) + \1(3 \leq z \leq 4)(4z - z^2) dz \\
    &= \int_2^3 (z^2-2z) dz + \int_3^4 (4z - z^2) dz \\
    &=\lsp \frac{1}{3}z^3 - z^2\rsp_2^3 + \lsp \frac{1}{2}z^2 - \frac{1}{3}z^3\rsp_3^4 \\
    &= 3
\end{align}

\textbf{Del 5) Udregn $\Cov(X, Z)$}

\begin{align}
    \Cov(X,Z) &= \Cov(X , X +Y) \\
    &=\Cov(X,X) + \Cov(X,Y) \\
    &= \Var(X) = \frac{1}{12}
\end{align}

Vi husker at $X, Y$ er uafhængige

Vi husker at variansen af $X$ er fundet tidligere


\subsubsection{Opgave U43.2.3}

\begin{itemize}
    \item $p_X(x) = \exp(-x)$
    \item $p_Y(y) = \exp(-y)$
    \item $X \independent Y$
    \item X, Y er defineret på $\R_+$
\end{itemize}

\textbf{Del 1) Find tætheden $p(x,y)$}

Grundet uafhængighed mellem $X,Y$ ved vi at: $p(x,y) = p_X(x)p_Y(y)$

Vi bruger dette:

\begin{equation}
    p(x,y) = \exp(-x)\exp(-y)
\end{equation}

\textbf{Del 2) find tætheden for $Z = X + Y$}

Vi gør som tidligere:

\begin{align}
     p(x, z-x) &= \1(0< x < z-x) \exp(-x)\exp(- (z-x))  \\ &=
     \1(0< x < z-x)\exp(-x) \exp(x) \exp(-z) \\
     &= \1(0< x < z-x)\exp(-z) 
\end{align}
   

udtrykket $\cdot 1$ er kun for at understrege der altid står 1.

\begin{equation}
    q(z) = \int_{0}^{z}\exp(-z) dx = \exp(-z) \int_{0}^{z} \1 dx = z\exp(-z)
\end{equation}

\textbf{Del 3) Find tætheden for $Z= X - Y$}

Vi bruger korollar 6.3.2 og indser at:
$Z = X - Y \implies Y = X - Z$

\begin{align}
    p(x,x-z) = \1(0 < z < x) \exp(-x) \exp( - (x - z)) \\
    &= \1(0 < z < x) \exp(-2x)\exp(-z)
\end{align}

Vi finder tætheden $q(z)$

\begin{align}
    q(z) &= \int_\R  \1(0 < z < x) \exp(-2x) \exp(-z) dx \\
    &= \exp(-z) \int_{0}^{\infty} \exp(-2x) dx \\
    &= \exp(-z)  \lsp-\frac{\exp(-2x)}{2}\rsp_{0}^{\infty} \\
    &= \frac{1}{2} \exp(-z) 
\end{align}

VI husker at i anden nederste ligning skal kun $x$ indsættes i square brackets.

\subsubsection{Opgave U43.2.4}

\begin{itemize}
    \item $X_1, X_2, X_3, X_4$ er identiske og uafhængige
    \item $\E(X_i) = 5, \Var(X_i) =9$
    \item $Y = X_1 + 2X_2 - X_4$
\end{itemize}

\begin{equation}
    \E(Y) = 5 + 2\cdot5 - 5 = 10
\end{equation}

Man husker de stokastiske variable er uafhængige

\begin{equation}
    \Var(Y) = \Var(X_1 + 2X_2 - X_4) = \Var(X_1) + 2^2 \Var(X_2) + \Var(X_4) = 6 \cdot 9 = 54
\end{equation}

\subsubsection{Opgave U43.2.5}

Drop denne opgave! Tidspres gør det umuligt at nå!

\subsubsection{Opgave 6.4}

\begin{itemize}
    \item Man har $p(x,y)$ givet ved:
    \begin{equation}
    p(x,y) = \begin{cases}
        3xy^{-2}, &x \in (0,1), y \in (1,3) \\
        0 &\text{ellers}
    \end{cases}
    \end{equation}
\end{itemize}

Find de marginale fordelinger for $X, Y$ og vis uafhængighed!

Lad $A = \lcp x,y \mid x\in(0,1), y\in(1,3)\rcp$

\begin{align}
    p_X(x) &= \int_\R \1_A(x,y) 3xy^{-2} dy \\
    &= \1_{[0,1]}(x) x\int_{1}^{3} 3y^{-2} dy \\
    &= \1_{[0,1]}(x) x\lsp 3 \cdot \frac{1}{-1} y^{-1} \rsp_{1}^{3} \\
    &= \1_{[0,1]}(x) x\lsp \frac{-3}{y} \rsp_1^3 \\
    &= \1_{[0,1]}(x) x\lp -1 + 3 \rp\\ 
    &=  \1_{[0,1]}(x) x\cdot 2 \\
    &= 2x, \quad x\in(0,1)
\end{align}


Det samme gøres for y

\begin{align}
    p_Y(y) &= \int_\R \1_A (x,y) 3xy^{-2}dy \\
    &= \1_{[1, 3]}(y) y^{-2} \cdot 3\int_{0}^{1} x dx \\
    &= \1_{[1, 3]}(y)y^{-2} \cdot 3 \lp \frac{1}{2} \rp \\
    &= \frac{3}{2}y^{-2}, \qquad y\in(1,3)
\end{align}

Vi tester for uafhængighed:

\begin{align}
    p_X(x) \cdot p_Y(y) &= \frac{3}{2}y^{-2}2x \\
    &= 3 y^{-2}x \\
    &= p(x,y)
\end{align}


Hvilket viser uafhængighed.
 
\subsubsection{Opgave 6.21}

\begin{itemize}
    \item $X \sim Uni(-1, 1)$
    \item $Y=X^2$
\end{itemize}

Vis at $\Corr(X, Y) = 0$

\begin{equation}
    \Corr = \frac{\Cov(X,Y)}{\sqrt{\Var(X)\Var(Y)}}        
\end{equation}

Vi finder Covariansen:

\begin{equation}
    \Cov(X,Y) = \E(X - \E(X))\E(Y - \E(Y)) = \E(X Y) - \E(X)\E(Y)
\end{equation}

Vi ser hurtigt at  $\E(X) = \E(Y) = 0$.
(evt - tegn for at overbevise klasse).

\begin{equation}
    \E(X \cdot Y) = \int_{-1}^{-1} x \cdot x^2 dx = \lp \frac{1}{4} \rp - \lp \frac{1}{4} \rp = 0
\end{equation}

Herfra ser vi let at:

\begin{align}
    \Cov(X,Y) = \E(X Y) - \E(X)\E(Y) = 0 - 0 = 0
\end{align}

De er ikke uafhængige! Kan vises formelt, men bedre med intuition ved at tegne!

Vis github!

\horizline

\subsection{Øvelse 13}

\textbf{28/10/2018, opgaver: 44.1.1, 44.1.2, 44.1.3 44.1.4}

\subsubsection{Opgave 44.1.1}

\begin{itemize}
    \item to terninger (stokastiske variable) $X_1, X_2$ 
    \item $(X_1, X_2) \in \lcp 1, 2, 3, 4, 5, 6\rcp^2 = \{x_{1,i}\} \times \{x_{2,j}\}, \quad i,j \in \lcp 1, 2, 3, 4, 5, 6\rcp$
    \item $Z = X_1 + X_2$
\end{itemize}

Den vigtige regel:

\begin{equation}
    p_x(x) =  \int_{y} p_{x, y}(x,y) dy= \int_{y} p_{x\mid y}(x \mid y)p_y(y) dy 
\end{equation}

\textbf{Find $P(X_1 = i \mid Z \geq 4)$}

Vi noterer først vi ikke har kontinuerte stokastiske variable!

Man får en god idé

\begin{equation}
    P(X_1 = i \mid Z \geq 4) = \frac{P(X_1 = i , Z \geq 4)}{P(Z\geq 4)}
\end{equation}

(skits summen af to terninger på tavlen) 

Vi indser hurtigt at $P(Z \geq 4) = \frac{33}{36}$

Vi indser også at:

\begin{equation}
    P(X_1 = i \mid Z \geq 4) = \frac{P(X_1 = i , X_1 + X_2 \geq 4)}{33/36}  = \frac{P(X_1 = i ,  X_2 \geq 4 - i)}{33/36}
\end{equation}

Vi indser at $i$ er en konstant og vi nu har uafhængighed i den simultane sandsynlighed således at:

\begin{equation}
    P(X_1 = i)P(X_2 \geq 4 -i )
\end{equation}

Husk at $P(X_1 = i) = \frac{1}{6}$

Vi kan opskrive det hele i et samlet udtryk:

\begin{equation}
        P(X_1 = i)P(X_2 \geq 4 -i ) = \frac{1}{6}P(X_2 \geq 4 -i )
\end{equation}

\begin{equation}
    P(X_1 = i)P(X_2 \geq 4 -i ) = \frac{1}{6}\cdot \frac{4 - 1 + i}{6} \quad i<4
\end{equation}
    
over i siger man bare $1/6 \times 1/6$
    
\begin{equation}
    P(X_1 = i)P(X_2 \geq 4 -i ) = \frac{3+ i}{36} \quad i <4
\end{equation}

Vi husker at dele med $33/36$ 


\begin{equation}
    P(X_1 = i \mid Z \geq 4)=
        \begin{cases}
            4/33 &i=1 \\
            5/33 &i=2 \\
            6/33 &i\geq3
        \end{cases}
\end{equation}

\textbf{Find $\E(X_1 = i  \mid Z \geq 4)$}

vi ved at $P(X_1 =i )= \frac{1}{36}$. VI kan derfor sige:

\begin{equation}
    \sum_{i=1}^6 i \cdot \min\lp \frac{3+i}{33}, \frac{6}{33} \rp = \frac{4\cdot 1 + 5 \cdot 2 + (3 + 4 + 5 +6)\cdot 6}{33}
\end{equation}

\subsubsection{Opgave 44.1.2}

\begin{itemize}
    \item $Z \in \{1, 2\}$ angiver kommune
    \item $V \in \{0, 1\}$ angiver om man er velhavende
    \item $P(V=1 \mid Z=1) = 0.8 = 1- P(V=0 \mid Z = 0)$
    \item $P(V=1 \mid Z=2) = 0.1$
\end{itemize}

\textbf{Udregn $\E(V \mid Z=1)$ og $\E(V \mid Z =2$}

Udtrykkene er udtryk for sandsynligheden for at være velhavende betinget på hvilken kommune man kommer fra.

\begin{equation}
    \E(V \mid Z=1) =  0 \cdot P(V=0 \mid Z = 0) + 1 \cdot (V=1 \mid Z=1) = 0.2 \cdot 0 + 0.8 \cdot 1 = 0.8
\end{equation}

For kommune 2:

\begin{equation}
    \E(V \mid Z = 2) = 1 \cdot P(V=1 \mid Z=2) +  0\cdot  P(V=0 \mid Z=2) = 0.1 
\end{equation}

\textbf{Vis udtrykket:}

\begin{equation}
    \E(V \mid Z=z) = f(z) = 0.8 \cdot \1(z=1) + 0.2 \cdot \1(Z=2)
\end{equation}

Man ser at hvis $z=1 \implies E(V \mid Z=1) = 0.8$

og omvendt: $z=2 \implies E(V \mid Z=2) = 0.1$

\textbf{Hvad udtrykker $\E(V \mid Z=z)$}

Det betyder at vores forventning er afhængig af realization af $z$.

\textbf{Del 4) }

Man definerer nu den stokastiske variabel \textit{Den betingede middelværdi af V givet Z}.

\begin{equation}
    \E(V \mid Z) = f(z)
\end{equation}

Vis at:

\begin{equation}
    \E(f(z)) = \E(\E(V \mid Z)) = 0.8 P(Z=1) + 0.1 P(Z=2)
\end{equation}

Det følger næsten naturligt:

\begin{equation}
    \E(f(z)) = \E(0.8 \cdot \1(z=1) + 0.1 \cdot \1(z=2))
\end{equation}

Herfra følger det da $V\in \{0,1\}$

\begin{equation}
    \E(f(z)) = 0.8 \cdot P(Z=1) + 0.1 \cdot P(Z=2)
\end{equation}

\subsubsection{Opgave 44.1.3}


\begin{itemize}
    \item $X$ er ligefordelt på $A = [0,10]$
\end{itemize}

\textbf{Del 1) Opskriv tætheden $p(x)$ for $X$ og vis $P(X) > 5 = \frac{1}{2}$}

Tegn tæthedsfunktionen.

Man ved at $F(x) \rightarrow 1$ for $x\rightarrow \infty$. nærmere bestemt ved man at $F(10) = 1$.
Man ved at $\int \1_{A}(x)$ vil være $x$, så man skal gange en konstant på for at få $F(10) = 1$. Hel konkret $10\cdot c = 1 \implies c = 1/10$

\begin{equation}
    p(x) = \frac{1}{10}\1_{A}(x)
\end{equation}

\begin{equation}
    P(X > 5) = \int_{0}^{5} \frac{1}{10}\1_{A}(x) = \frac{1}{10}\int \1_{A}(x) = \frac{1}{10} [x]_{0}^{5}= \frac{1}{10}(5 - 0) = 0.5
\end{equation}

\textbf{Del 2) Find $\E(X)$}

\begin{equation}
    \E(x) = \int_{-\infty}^{\infty} p(x)x
\end{equation}

Vi ved at indikator funktionen kun er defineret i intervallet $[0,10]$. så vi kan skrive:

\begin{align}
    \E(X) &= \int_{0}^{10} \frac{1}{10} x \cdot \1_{A}(x)  \\
    &= \frac{1}{0} \int_{0}^{10} x \\
    &= \frac{1}{10}\lsp\frac{1}{2}x^2\rsp_0^{10} \\
    &= \frac{1}{10}\cdot\frac{1}{2}\cdot10^2 = 5
\end{align}

\textbf{Vis at tætheden for $X\mid X>5$  kan skrive som:}

Skitser det givne på en tegning!

\begin{equation}
    q(x) = \frac{2}{10}\1(5 < x < 10)
\end{equation}

Man indser hurtigt at: $X \in [0, 5] \cap X \in (5,10] = \emptyset$. Vi kan altså herfra konkludere at $X \mid X>5$ kun er defineret på intervallet $(5, 10]$. 

$X \mid X>5$ er stadig uniformt fordelt, og vi kan derfor sige at: $q(x) = c \cdot \1(5 < x \leq 10)$. Igen ved vi også at $Q(10) = 1$. Vi kan hurtige udlede at $c=\frac{1}{5}$. hvormed det ønskede resultat er vist.

\textbf{Del 4) Er $\E(X \mid X>5) = 7.5?$}

Først se på tegningen. Herfra burde det fremgår tydeligt. Mere formelt:

\begin{align}
    \E(X \mid X>5) &= \int_{-\infty}^{\infty}x \cdot \frac{1}{5}\1(5 < x <10) \\
    &= \frac{1}{5}\int_{5}^{10} x \cdot \1(5 < x <10) \\
    &= \frac{1}{5}\lsp \frac{1}{2}x^2 \rsp_{5}^{10} \\
    &= \frac{1}{5}\frac{1}{2}(10^2 - 5^2) \\
    &= \frac{1}{5}\frac{1}{2} \cdot 75 \\
    &= 7.5
\end{align}


\subsubsection{Opgave 44.1.4}

\begin{itemize}
    \item $X$ angiver ratingen fra 0 til 1
    \item $Y$ angiver værdipapirets værdi i 1000 \$
    \item $X, Y$ er ligefordelt på mængden $B$
    \item $B = \{(x,y) \in \R^2 \mid 0 < x < 1, 0.5 + 2x \leq y \leq 2.5 + 2x \}$
\end{itemize}

Lad os starte med at tegne $B$. Kig github!


\textbf{Find tæthedsfunktionen $f_{X, Y}(x,y)$ for den simultane fordeling for $(X,Y)$}

Vi hurtigt indser at de marginale fordelinge må blive 1. Den hurtigste måde at konstanten $c$ på (tænk simultan fordeling $f(x,y) = c \1_{B} (x, y)$). er at finde arealet af $B$.

\begin{equation}
    \frac{1}{c}= h \cdot l = 1 \cdot 2 = 2 \implies c = \frac{1}{2}
\end{equation}

tæthedsfunktionen er:

\begin{equation}
    f_{x,y}(x,y) = \frac{1}{2}\1_{B}(x,y)
\end{equation}

\textbf{Del 2) Find $P(Y>2)$}

Tegn på tegningen hvad det egentlig medfører. Altså på mængden $B$.

Først og fremmest ved vi at vi må integrere $X$ ud af tætheden.

\begin{equation}
    p_y(y) = \int_{\R} \frac{1}{2}\1_{B}(x,y) dx
\end{equation}

Vi lavet et trick og skærer mængden $B$ ud i to mængder $M_1$, $M_2$.

\begin{equation}
    M_1 = \{x, y \mid 0 < x <1, 0.5 + 2x < y < 2.5\}
\end{equation}

\begin{equation}
    M_2 = \{x,y \mid 0 < x < 1, 2.5 < y < 2.5 + 2x\} 
\end{equation}

\begin{equation}
    p_Y (y) = \frac{1}{2}\int_\R \1_{M_1} (x,y) dx + \frac{1}{2}\int_\R \1_{M_2} (x,y) dx 
\end{equation}

Vi håndterer først $M_1$:

Vi ser at vi skal differentiere $x$ ud. Mængden er er altså defineret i $y$-intervallet [0.5, 2.5]. Vi isolerer $x$ som en funktion af $y$:

NOTE: Tegn diagrammet på tavlen og forklar intuitionen!

\begin{equation}
    y = 0.5 + 2x \implies \frac{1}{2}(y - 0.5) = x
\end{equation}

Hvor vi husker at: $y \in [0.5, 2.5]$

Vi kan nu finde at arealet for $M_1:$

\begin{equation}
    \int_{0}^{\frac{1}{2}(y-0.5)} 1 dx = [x]_{0}^{\frac{1}{2}(y-0.5)} = \frac{1}{2}y - 0.25
\end{equation}

Analogt for $M_2$:

(KIG PÅ TAVLESKITSE)
\begin{equation}
    y = 2.5 + 2x \implies \frac{1}{2}(y - 2.5) 
\end{equation}

\begin{equation}
    \int^{1}_{\frac{1}{2}(y - 2.5)} 1 dx = [x]_{\frac{1}{2}(y - 2.5)}^{1} = 1 - \lp \frac{1}{2} y - 1.25\rp = 2.25 - \frac{1}{2}y 
\end{equation}

hvor vi husker at $y \in (2.5 , 4.5]$

Vi opskriver $p_Y (y)$. Man husker at gange konstanten $\frac{1}{2}$ på.

\begin{equation}
    p_Y(y) = 
    \begin{cases}
        \frac{1}{2}\lp 2.25 - \frac{1}{2}y \rp &,y \in (2.5 , 4.5] \\
        \frac{1}{2}\lp \frac{1}{2}y - 0.25 \rp & ,y\in [0.5, 2.5]
    \end{cases}
\end{equation}

Vi kan opskrive $P(Y>2) = 1- P(Y\leq2) = 1- \int_{0.5}^{2} \frac{1}{2}\lp\frac{1}{2}y - 0.25\rp dy$

\begin{align}
   1- \int_{0.5}^{2} \frac{1}{2}\lp\frac{1}{2}y - 0.25\rp dy &= 1 - \frac{1}{4}\int_{0.5}^{2} y - 0.5 dy \\  
   &= 1 - \frac{1}{4} \lsp \frac{1}{2}y^2 - 0.5y \rsp \\
   &= 1 - \frac{1}{4}\lp \lp\frac{1}{2}2^2 - \frac{1}{2} \cdot 2 \rp - \lp \frac{1}{2}0.5^2 - 0.5 \cdot 0.5\rp \rp \\
   &= 1 - \frac{1}{4}( 2 - 1) + \frac{1}{4}\lp \frac{1}{8} - \frac{1}{4} \rp \\
   &= 1 - \frac{1}{4} - \frac{1}{4}\frac{1}{8} \\
   &= 0.71875
\end{align}

\textbf{Del 4) Angiv den betingede fordeling af $X$ givet $Y=1$}

Vi skal finde $p_{X \mid Y=1}(x)$

Vi kan altså bruge vores regel:

\begin{equation}
    p_{X \mid Y} (x) p_{Y}(y) = p(x , y) \implies p_{X \mid Y} = \frac{p(x,y)}{p_Y(y)}
\end{equation}

Vi ved at $Y=1$. Vi bruger dette:

\begin{equation}
    p_{Y}(1) = \frac{1}{2}\lp \frac{1}{2}(1) - 0.25\rp = \frac{1}{4} - \frac{1}{8} = \frac{1}{8}
\end{equation}

Vi indsætter $Y=1$ i den øverste del af brøken. Vi ved vi er i den nederste mængde $M_1$. Dette implicerer:

\begin{equation}
    0.5 + 2x < y \land y = 1 \implies  0.5 +2x < 1 \implies x < \frac{1}{4}
\end{equation}

Vi kan herfra konkludere at:

\begin{equation}
    p_{X\mid Y=1}(x) = \frac{1}{2}\frac{\1_{[0, 0.25]}(x)}{1/8} = 4\cdot \1_{[0, 0.25]}(x)
\end{equation}

\textbf{Del 5) udregn forventede rating når $Y=1$ og når $Y=2$}

Vi kender formlen for forventningen af en ligefordeling: $\E(x) = \frac{a + b}{2}$

Vi har svaret for $E(X \mid Y=1) = \frac{0 + 0.25}{2} = \frac{1}{8}$

Vi skal nu analogt finde den betingede tæthed når $Y=2$

\begin{equation}
    p_Y(2) =  \frac{1}{2}\lp \frac{1}{2}(2) - 0.25\rp = \frac{1}{2} - \frac{1}{8} = \frac{3}{8} 
\end{equation}

Vi ser igen på mængden $M_2:$

\begin{equation}
    0.5 + 2x < y \land y=2 \implies 0.5 + 2x < 2 \implies x < \frac{3}{4}
\end{equation}

Vi kan herfor konkluderer at den betingede fordeling for $X \mid Y=2$ må være:

\begin{equation}
    p_{X \mid Y=2} (x) = \frac{1}{2}\frac{\1_{\lsp0,\frac{3}{4}\rsp}(x)}{\frac{3}{8}} = \frac{4}{3} \cdot \1_{\lsp0,\frac{3}{4}\rsp} 
\end{equation}

Vi finder forventningen som må være:

\begin{equation}
    \E(X\mid Y=2) = \frac{1}{2}\frac{3}{4}=\frac{3}{8}
\end{equation}

\textbf{Del 6) Find variansen $\Var(X \mid Y = 1$ og $\Var(X\mid Y=2)$}

I stedet for at bruge hintet kigger vi på distributionen og bruger regnereglen for ligefordelinger:

\begin{equation}
    Var(X) = \frac{1}{12}(a - b)^2
\end{equation}

\begin{equation}
    \Var( X \mid Y =1) = \frac{1}{12}\lp 0 - \frac{1}{4} \rp^2 = \frac{1}{16}\frac{1}{12} = \frac{1}{192}
\end{equation}

\begin{equation}
    \Var(X \mid Y = 2) = \frac{1}{12} \lp 0 - \frac{3}{4} \rp^2 = \frac{1}{12}\frac{9}{16} = \frac{9}{192}
\end{equation}

\textbf{Hvornår er den betingede varians størst - dvs. variansen af ratingen betinget på prisen}

Kig på tegningen: Det rigtige svar må være $Y=2500$.

Man overvejer følgende:

\begin{equation}
    Var(X) = \frac{1}{12}(a - b)^2
\end{equation}

I intervallet $y \in [0.5, 2.5]$ ved vi at:

\begin{equation}
    Var(X \mid Y= y) = \frac{1}{12}(0 - a)^2
\end{equation}

Hvor at er øvre grænse:

\begin{equation}
    0.5 + 2x < y \implies x = \frac{1}{2}(y - 0.5)
\end{equation}

er monotont stigende med højere i $y$ i intervallet $[0.5, 2.5)$.

Vi kan derfor sige at:

I intervallet $[0.5, 2.5)$ finder vi den højeste varians ved $Y=2.5$. 

Analogt kan man den højeste varians i i intervallet $[2.5 , 4.5]$ til at være $Y=2.5$ 

Illustrer på tavle!



\horizline

\subsection{Øvelse 14}

\textbf{01/11/2018, opgaver: 44.2.1, 44.2.2, 44.2.3}

\subsubsection{Opgave 44.2.1}

\begin{itemize}
    \item $U, V \sim N(0,1)$
    \item $U \independent V$
    \item derudover er følgende variable defineret:
    \begin{align}
        X &= \alpha_1 + \beta_1 U \\
        Y &= \alpha_2 + \beta_2 U + \delta_2  V
    \end{align}
\end{itemize}

\textbf{Del 1) Udregn $P(0.1 < U <0.5)$}

Kig github! Gjort i python, så det er let tilgængeligt for alle online.

\begin{equation}
    P(0.1 < U <0.5) = 1 - F_U(0.1) - (1- F_U(0.5)) = - F_U(0.1) + F_U(0.5) = 0.15163
\end{equation}

\textbf{Del 2) Udregn $E(X), E(Y)$ samt $V(X), V(Y)$}

\begin{equation}
    \E(X) = \E(\alpha_1 + \beta_1 U) = \E(\alpha_1) + \beta_1 \E(U) = \alpha_1 
\end{equation}

\begin{equation}
    \E(Y) = \E(\alpha_2 + \beta_2 U + \delta_2 V) = \alpha_2
\end{equation}

Igen fordi at $\E(U) = \E(V) = 0$ eftersom vi har standard normalt fordelte stokastiske variable $U, V$

Variansen kan nu findes:

Man husker at standard normalt fordelte stokastiske variable har egenskaben: $\sigma^2 = \sigma = 1$.

\begin{equation}
    \Var(X) = \Var(\alpha_1 + \beta_1 U) = \beta_1^2 \Var(U) = \beta_1^2 
\end{equation}

\begin{equation}
    \Var(Y) = \Var(\alpha_2 + \beta_2 U + \delta_2 V) = \Var(\beta_2 U + \delta_2 V)
\end{equation}

Vi husker at når $V \independent U \implies \Var( V + U) = \Var(V) + \Var(U)$

\begin{align}
    \Var(\beta_2 U + \delta_2 V) &= \Var(\beta_2 U) + \Var(\delta_2 V) \\
    &= \beta_2^2\Var(U) + \delta^2\Var(V)  \\ 
    &= \beta_2^2 + \delta_2^2
\end{align}

\textbf{Del 3) Udregn $\E(X \cdot Y)$ og $\Cov(X,Y)$}

\begin{align}
    \E(X \cdot Y) &= \E\lsp (\alpha_1 + \beta_1 U) (\alpha_2 + \beta_2 U + \delta_2  V) \rsp  \\
    &= \alpha_1 \alpha_2 + \beta_1 \beta_2 \E(U^2) \\
    &= \alpha_1 \alpha_2 + \beta_1 \beta_2
\end{align}

Dette er klart da  $\E(V) = \E(U) = 0 \land V \independent U \implies \E(V \cdot U) = 0$

Covariansen findes:

\begin{equation}
    \Cov(X,Y) = \E(X\cdot Y) \E(X)\E(Y) = (\alpha_1 \alpha_2 + \beta_1 \beta_2) - (\alpha_1\alpha_2) = \beta_1\beta_2
\end{equation}

\textbf{Del 4) Hvad skal $\alpha_1, \beta_1$ sættes til for at $\E(X) = 10 , \Var(X) = 4$}

Vi har tidligere fundet: $\E(X) = \alpha_1$, $\Var(X) = \beta_1^2$.

Vi kan derfor hurtigt konkluderer at: $\alpha_1 = 10 \implies \E(X) = 10$ og at $\beta_1^2 = 2 \implies \Var(X) = 4$

\textbf{Del 5)}

Find $\alpha_1, \alpha_2 + \beta_1\beta_2, \delta_2$ for at $\Cov(X, Y) = 4$

Vi kan hurtigt konkludere at: $\Cov(X,Y) = \beta_1 \beta_2$, betyder at vi kan sige $\beta_1 = \beta_2 = 2 \implies \Cov(X,Y) = 4$. De resterende parametre kan sættes tilfældigt.

\subsubsection{Opgave 44.2.2}

\begin{itemize}
    \item $Y_1$ Er timeløn i periode 1
    \item $Y_2$ Er timeløn i periode 2
    \item $Y_1 \sim N(\mu, \sigma^2)$
    \item $Y_2 \sim \alpha + \beta Y_1 + U$
    \item $U \sim N(0,v^2)$
    \item $Y \independent U$
\end{itemize}

\textbf{Del 1) lad $\mu = 350$ og $\sigma^2  = 12365$}

\begin{itemize}
    \item \textbf{Del A)} Udregn Ssh for timelønnen i periode 1 er højst 275
    \item \textbf{Del B)} Udregn Ssh for timelønnen i periode 1 er mindst 425
\end{itemize}

Kig i Github!

\begin{equation}
    P(Y_1 < 275) = F_{Y_1}(275) =0.25 
\end{equation}

\begin{equation}
    P(Y_1 > 425) = 1 - F_{Y_1}(425) 
\end{equation}

\textbf{Del 2) Udregn middelværdi og varians af $Y_2$}

\begin{equation}
    \E(Y_2) = \E(\alpha + \beta Y_1 + U) = \E(\alpha) + \beta \E(Y_1) + \E(U)
\end{equation}

Vi husker $\E(U) = 0$ ($U$ var fordelt omkring 0). Derudover husker vi at $\E(Y_1) = \mu$

\begin{equation}
    \E(Y_2) = \alpha + \beta \mu
\end{equation}

Vi finder variansen.

\begin{align}
    \Var(Y_2) &= \Var(\alpha + \beta Y_1 + U) \\
    &= \Var(\beta Y_1 + U) \\
    &= \beta^2 \Var(Y_1) + \Var(U) \\
    &= \beta^2 \sigma^2 + v^2
\end{align}

Hvor vi har udnyttet at man kan splitte variansen op til en sum af to uafhængige stokastisk variable.

\textbf{Del 3) er $Y_1 \independent Y_2$}

Nej! Dette kan ses uden yderligere udregninger, da $Y_2$ er en transformation af $Y_1$. 

Note: Dette afhænger af den Strukturalle kausale model er \textit{faithful}, hvilket vi kan se i dette tilfælde den er. (Læs mere i Jonas Peters bog om causality hvis man er interesseret)

\textbf{Del 4) Angiv den betingede middelværdi og varians af $Y_2$ betinget på $Y_1= y_1$}

\begin{equation}
    p_{Y_2\mid Y_1 = y_1}(y_2) = \alpha + \beta y_1 + U
\end{equation}

Vi kan hermed sige at:

\begin{equation}
    \E(Y_2 \mid Y_1 = y_1) = \E(\alpha + \beta y_1 + U) = \alpha + \beta y_1
\end{equation}

Inden variansen findes, da overvej: $\Var(\beta y_1) = 0$ . Da $y_1 $ er en realisation of derfor ikke har noget stokastisk element.  

Variansen er:

\begin{equation}
    \Var(Y_2 \mid Y_1 = y_1) = \Var(\alpha + \beta y_1 + U) = \Var(U) = v^2
\end{equation}

\textbf{Del 5)}

Antagelser
\begin{itemize}
    \item $\mu =350$
    \item $\sigma^2 = 12365$
    \item $\alpha = 350(1- \beta)$
    \item $v^2 =  12365(1 - \beta^2)$
    \item (ikke i opgaven - men nødvendigt) $\lvert \beta \rvert < 1$
\end{itemize}

\textbf{Angiv de marginale fordelinger af $Y_1$ og $Y_2$}

Man ser hurtigt at:

\begin{equation}
    Y_1 := N(350, 12365)    
\end{equation}

Den marginale fordeling for $Y_2$:

\begin{align}
    Y_2 &:= \alpha + \beta Y_1 + U \\
        &= 350(1-\beta) + \beta Y_1 + U \\
        &= 350(1-\beta) + \beta N(350, 12365) + N(0, 12365(1 -\beta^2)) \\
\end{align}

Vi kan se at $\E(350(1-\beta) + \beta N(350, 12365) + N(0, 12365(1 -\beta^2))) = 350$. Fra $\beta 350 + (1-\beta)350 = 350$

For variansen ser vi følgende: Husk $k^2 \Var(X) = \Var(kX)$. Det vil sige: $\beta N(350, 12365) = N(350, \beta^2 12365)$. Variansen for 2 ukorrelerede stokastiske variable: $\Var(X) + \Var(Y) = \Var(X + Y)$. Hvilker betyder at variansen i dette tilfælde er:

\begin{equation}
    \Var(N(350, \beta^2)) + \Var(N(0, 12365(1-\beta^2)) =12365( \beta^2 + (1 -\beta^2) = 12365
\end{equation}

\textbf{Antag nu $\beta > 0$. Vis at$\E(Y_2 \mid Y_1 = 275) <\E(Y_2)$}

Vi kan se at $\E(Y_2) = 350$. læs ovenfra:

Vi kan se at:

\begin{equation}
    \E(Y_2 \mid Y_1 = 275) = \E(350(1 - \beta) + \beta275 + N(0,12365(1-\beta^2))
\end{equation}

Hvilket betyder:

\begin{equation}
    \E(Y_2 \mid Y_1 = 275) = 350(1-\beta) + \beta 275 < 350
\end{equation}

\textbf{Del C)}

Lad nu \textbf{$\beta = 0.90$}
 
 NOGET GÅR GALT
 
\begin{itemize}
    \item $P(Y_1 \leq 275, Y_2 \leq 275) \approx 0.193$
    \item $P(Y_1 \leq 275, Y_2 \leq 425) \approx 0.250$
\end{itemize}

Vi ved at $P(A , B) = P( A \mid B) P(B) $

Så vi kan sige:

\begin{equation}
    P(Y_2 \leq 275 \mid Y_1 \leq 275) =  \frac{P(Y_1 \leq 275, Y_2 \leq 275) }{P(Y_1\leq 275)} = \frac{0.193}{0.25} = 0.772
\end{equation}

\subsubsection{Opgave 44.2.3}

\begin{itemize}
    \item $(X,Y)$ er en todimensionel stokastiske vektor
    \item $f_{X,Y} = 6 \exp(-2x-3y) \quad x,y \in [0,\infty)$
    \item $A = \{(x,y) : 0 \leq x + y< 1, x>0, y >0\}$
\end{itemize}

\textbf{Del 1) Hvorfor er $X \independent Y$}

Lavet i en tidligere ugeseddel. Kan splittes op i to tæthedsfunktioner:

\begin{equation}
    f_X(x) = 2 \exp(-2x), \quad f_Y (y) = 3 \exp(-3y)
\end{equation}

Disse kan ganges sammen til $f_{X,Y}$

\textbf{Del 2) Udregn $P((X,Y)\in A)$}

Først tegn $A$: Kig github!

Note: $\int e^{bx}dx = \frac{e^{bx}}{b}$ 

Vi ser$ x+y = 1 \implies y-1 = x$. Og $y = x- 1$

\begin{align}
    P((X,Y) \in A) &= \int_0^1\int_0^{1-y} 6\exp(-2x-3y) dx dy\\
    &= \int_0^1 \int_0^{1-y} 6\exp(-2x)\exp(-3y) dx dy \\
    &=6 \int_0^{1}\exp(-3y)\int_0^{1-y}\exp(-2x) dx dy \\
    &\cdots \\
    &= [- \exp(-2x)]_{0}^{1} - 2\exp(-3)[\exp(x)]_{0}^{1} \\
    &= [1-\exp(-2)] - 2 \exp(-3)[\exp(1)-1] \\
    &\approx 0.6935
\end{align}

\textbf{Find tætheden for $(X,Y)$ givet $(X,Y) \in A$}

\begin{align}
    p(x,y \mid (X,Y) \in A) &= \frac{p(x,y, (X,Y) \in A)}{P((X,Y) \in A)} \\
    &= \frac{p(x,y) \1_A(x,y)}{0.6935} \\
    &= 6\exp(-2x -3y) / 0.6935 \qquad 0< x+ y <1
\end{align}

Kig Thomas rettevejledning for integrale show!

\textbf{Del 4) Er $X\independent Y \mid A$}

Nej! $A$ er ikke en produkt mængde!

\textbf{Del 5) Find $\E(X \mid (X,Y) \in A$ og ligeledes for $Y$}


\horizline

\subsection{Øvelse 15}

\textbf{5/11/2018, opgaver: U45.1, U45.2, U45.3, U45.4}

\subsubsection{Opgave U45.1}

\begin{itemize}
    \item $X \sim N(\mu, \sigma^2)$
\end{itemize}

\textbf{Vis at $Y=\frac{1}{\sqrt{\sigma^2}}(X -\mu)$ er standard normal fordelt $N(0,1)$}

Vi kan starte med at opskrive den parametriske form op for tætheden af en normalfordeling med $mu$ og $sigma^2$ som henholdsvis middelværdi og varians

\begin{equation}
    p(x) = \frac{1}{\sqrt{2 \pi \sigma^2}} \exp \lp-\frac{(x- \mu)^2}{2\sigma^2} \rp 
\end{equation}

Vi husker et par regneregler:

Vi ved at $\E(X) = \mu$.

Det vil sige at:

\begin{equation}
    \E[Y] = \E \lsp \frac{1}{\sqrt{\sigma^2}}(X -\mu) \rsp = \frac{1}{\sqrt{\sigma^2}} \E \lsp (X -\mu)  \rsp
\end{equation}

Som sagt: $\E(X) = \mu$.

\begin{equation}
    \frac{1}{\sqrt{\sigma^2}} \E \lsp (X -\mu)  \rsp =\frac{1}{\sqrt{\sigma^2}}( \E[X] - \E[\mu]) =\frac{1}{\sqrt{\sigma^2}}  \lp \mu - \mu \rp = 0 = \E[Y] 
\end{equation}

Vi finder variansen. Man husker at $\Var(aX) = a^2 \Var(X) $

Vi ved at $\Var(X) = \sigma^2$

\begin{equation}
    \Var(Y) = \Var\lp \frac{1}{\sqrt{\sigma^2}}X\rp
\end{equation}

Vi ignorerer $\mu$ da dette er en konstant
\begin{equation}
    \Var\lp \frac{1}{\sqrt{\sigma^2}}X\rp = \frac{1}{\sigma^2}\Var(X) = \frac{1}{\sigma^2}\sigma^2 = 1 = \Var(Y)
\end{equation}

Og vi har nu vidst at $Y\sim N(0,1)$

\textbf{Del 2) $(X,Y)$ er en 2-dimensionel stokastisk vektor med middelværdi $mu$ og covarians matrice = $\Omega$}

\textbf{Vis at $Z = \frac{1}{\sqrt{\sigma_X^2}}(X - \mu_X)$ er standard normalfordelt}

\begin{equation}
    \mu = \lp
    \begin{array}{cc}
         \mu_X  \\
         \mu_Y 
    \end{array} \rp \qquad
    \Omega = \lp
    \begin{array}{cc}
    \sigma^2_Y & \sigma_{XY} \\
    \sigma_{XY} & \sigma^2_X
    \end{array}
    \rp
\end{equation}

Samme argument som før!

\textbf{Lad $(X,Y)$ være som i spørgsmål 2, men med $\mu = (0,0)^T$. Vis at $Z = Y-\beta X$ er normaltfordelt med $N(0,\sigma^2)$}

Hvor $\sigma^2$ er 

\begin{equation}
    \sigma^2 = \sigma^2_Y - \beta \sigma_{XY}
\end{equation} 

og $\beta$ er:

\begin{equation}
    \beta = \sigma_{X,Y} / \sigma^2_X
\end{equation}

Vi husker at summen af to normalfordelte stokastiske variable er normalfordelt. Dvs: $Y - \beta X$ nødvendigvis må være normalfordelt!

Vi finder middelværdien først:

\begin{equation}
    \E[Z] = \E[Y - \beta X] = \E[Y] - \beta \E[X] = 0 - \beta\cdot 0 = 0
\end{equation}

Nu finder vi variansen:

\begin{equation}
    \Var(Z) = \Var(Y - \beta Z)
\end{equation}

Vi husker der er covarians mellem $X$ og $Y$.

Det vil sige:

\begin{align}
    \Var(Z) 
    &= \Var(Y - \beta x) \\
    &= \Var(Y) + \beta^2 \Var(X)  + 2\Cov(Y, - \beta X) \\
    &= \Var(Y) + \beta^2 \Var(X)  - 2\beta \Cov(Y, X) \\
    &= \sigma^2_Y + \beta^2 \sigma^2_X - 2\beta \sigma_{X,Y} \\
    &= \sigma^2_Y + (\sigma_{X,Y} / \sigma^2_X)^2 \sigma^2_X - 2(\sigma_{X,Y} / \sigma^2_X) \sigma_{X,Y}\\
    &= \sigma^2_Y + \frac{\sigma_{X,Y}^2}{\sigma^2_X} - 2\frac{\sigma_{X,Y}^2}{\sigma^2_X} \\
    &= \sigma^2_Y - \frac{\sigma_{X,Y}^2}{\sigma^2_X}
\end{align}

Hvilket var det vi ønskede at vise:

\begin{equation}
    Z \sim N(0, \sigma^2) = N \lp 0,\sigma^2_Y - \frac{\sigma_{X,Y}^2}{\sigma^2_X} \rp
\end{equation}

\textbf{Del 4)}

Vis at:

\begin{equation}
    \E((Y - \beta X)X) = 0
\end{equation}

Hvilket betyder at $Z$ og $X$ er uafhængige!

\begin{align}
    \E((Y - \beta X)X) &=
    \E(YX - \beta X^2) \\
    &= \E(YX) - \beta\E(X^2) \\
\end{align}

Her ser man at $\E(X) = \E(Y) =0 \implies \E(XY) = \sigma_{XY}$ og at $\E(X) = 0 \implies \E(X^2) = \sigma^2_X$

\begin{align}
    \E(YX) - \beta\E(X^2) &= \sigma_XY - \beta \sigma_X^2 \\
    &= \sigma_{XY} - (\sigma_{X,Y} / \sigma^2_X)\sigma_X^2 \\
    &= 0
\end{align}

For at afgøre om de er uafhængige definerer vi først fejlleddet $\epsilon$:

\begin{equation}
    \epsilon = Y - \beta X
\end{equation}

\begin{align}
    \Cov(\epsilon, X) = \E(\epsilon X ) - \E(\epsilon) \E(X) = \E(\epsilon X ) = 0
\end{align}

Det betyder at $\epsilon$ er ukorreleret med $X$. Og da både $X$ og $\epsilon$ er normalfordelte kan vi konkludere de er uafhængige!

\subsubsection{Opgave U45.2}

Laves i klassen!

\begin{itemize}
    \item stokastisk vektor $(X,Y)$
    \item distribueret med $N(m, \Omega)$
\end{itemize}

\begin{equation}
    m = \lp 
    \begin{array}{cc}
         1  \\
         0 
    \end{array} \rp
    \qquad \Omega = \lp
    \begin{array}{cc}
         1 & \rho  \\
         \rho & 1
    \end{array} \rp
\end{equation}

\textbf{Del 1) Hvad er $\E(X)$ og $\Var(X)$}

Vi kan direkte aflæse svarene $\E(X) = 1$ og $\Var(X) = 1$.

\textbf{Del 2) Hvordan er $Y$ fordelt:}

Aflæses i $m$ og $\Omega$   

\begin{equation}
    Y \sim N(0,1)
\end{equation}

\textbf{Del 3) Hvad er $\Cov(X,Y)$}

Dette kan aflæses i kovarians-matricen off-diagonal elementer: $\sigma_{X Y} =  \rho$

\textbf{Del 4) Hvad er $\E(Y \mid X = x)$}

Fra Rahbeks note (property G.3) finder vi formlen:

\begin{equation}
    \E(Y \mid X =x ) = \mu_{Y \mid X} = \mu_Y + \omega (x - \mu_X)
\end{equation}

Hvor $\omega = \sigma_{YX}/\sigma_X^2$

Hvilket implicerer at:

\begin{equation}
    \E(Y \mid X =x ) = \mu_{Y \mid X} = \mu_Y + (\sigma_{YX}/\sigma_X^2)(x - \mu_X) 
\end{equation}

Vi finder de passende værdier i kovarians-matricen: $\mu_X = 1$, $\mu_Y = 0$, $\sigma_X^2=1$ og $\sigma_{XY} = \rho$

\begin{equation}
    \E(Y \mid X = x) = 0 + \lp  \frac{\rho}{1}\rp(x - 1) = \rho(x -1)
\end{equation}

\textbf{Del 5) Hvad er $\E(X \mid Y = y)$}

Vi bruger samme formel som før, bare hvor: $\omega = \sigma_{YX}/\sigma_Y^2$

\begin{equation}
    \E(X \mid Y = y) = \mu_x + \omega(y - \mu_Y) = 1 + \rho y
\end{equation}

\textbf{Del 6) Hvad er $\Var(Y\mid X= x)$}

Vi kigger igen i Rahbeks note (property G.3).

Finder formlen:

\begin{equation}
    \Var(Y\mid X = x) = \sigma_Y^2 - \omega \sigma_{XY} = \sigma_Y^2 - \frac{\sigma_{XY}^2}{\sigma_X^2}
\end{equation}

Og vi har stadig $\omega = \sigma_{YX}/\sigma_Y^2$. Altså samme $\omega$ som i del 5.

Vi har de passende værdier:
$\sigma_X^2 = 1$, $\sigma_Y^2 =1$ og $\sigma_{XY} = \rho$

\begin{equation}
    \Var(Y\mid X = x) = \sigma_Y^2 - \frac{\sigma_{XY}^2}{\sigma_X^2} = 1 - \frac{\rho^2}{1} = 1 -\rho^2
\end{equation}

\textbf{Del 7) Hvad gælder for $(X,Y)$ hvis $\rho = 0.9$ og $\rho=0$}

Hvis $\rho=0.9$ er $X,Y$ stærkt positivt korrelerede. Det vil sige en høj realisering af $Y$ implicerer en høj realisering $X$.

Omvendt $\rho=0$ implicerer $X$ og $Y$ er ukorrelerede! Da de begge er normalt fordelte er de nødvendigvis også uafhængige!

\subsubsection{Opgave U45.3}

\begin{itemize}
    \item $(X, Y)$ er stokastisk vektor med som er normalfordelt med $N(\mu, \Omega)$
\end{itemize}

\begin{equation}
    \mu = \lp
    \begin{array}{cc}
         0  \\
         2 
    \end{array}\rp \qquad
    \Omega = \lp
    \begin{array}{cc}
        1 & 0.5 \\
        0.5 & 1
    \end{array} \rp
\end{equation}

\textbf{Opskriv tætheden $p(x,y)$}

Kig formlen på s.237 Sørensen (ligning 8.3.6)

\begin{equation}
    p(x,y) = \frac{1}{2\pi}\frac{1}{\sqrt{\det(\Omega)}}\exp\lp-\frac{1}{2}(x - \mu_X, y - \mu_Y)\Omega^{-1}
    \lp
    \begin{array}{cc}
         x - \mu_X  \\
         y - \mu_Y 
    \end{array} 
    \rp
    \rp
\end{equation}

Hvis det ser uklart ud, da kan vi hurtigt lige definerer vektoren $K = (x - \mu_X, y - \mu_Y)$

Hvilket forsimpler utrykket til (Gøres for klargøre at det er en vektor $K$ man "opløfter i 2"): 

\begin{equation}
    p(x,y) = \frac{1}{2\pi}\frac{1}{\sqrt{\det(\Omega)}}\exp\lp-\frac{1}{2}K \Omega^{-1} K^T    \rp
\end{equation}

Vi finder de to centrale ting:

\begin{equation}
    \det(\Omega) = 1^2 - 0.5^2 = 0.75
\end{equation}


Kig i Sørensen s.237
\begin{equation}
    \Omega^{-1} = \frac{1}{1 -\rho^2} \lp
    \begin{array}{cc}
        1 & -0.5  \\
        -0.5 & 1
    \end{array}    
    \rp = 
    \lp
    \begin{array}{cc}
         \frac{4}{3} & -\frac{2}{3} \\
         -\frac{2}{3} & \frac{4}{3}
    \end{array} \rp
\end{equation}

Og vi har fundet tætheden!

\subsubsection{Opgave U45.4}

\begin{itemize}
    \item $Y$ er afkast på amerikansk aktie (Microsoft)
    \item $X$ er et aktie-indeks (SP500)
    \item $Y := \beta X + \epsilon$
    \item $\epsilon \sim N(0,\sigma^2)$
\end{itemize}

\textbf{Del 1) Fortolkning af $\beta$}

$\beta$ er relateret til kovariansen mellem $X$ og $Y$ og angiver altså samvariansen mellem en given aktie og hvordan hele markedet bevæger sig. En aktie med negativ $\beta$ vil altså kunne reducerer volatiliteten i en portfølje da den er modsat korreleret med de andre aktier.

\begin{equation}
    \beta = \frac{\sigma_{XY}}{\sigma^2_X}
\end{equation}

Står beskrevet i Rahbeks Note s. 10

\textbf{Del 2) En anden model blev repræsenteret}

\begin{equation}
    Y := \epsilon_Y, \qquad X := \epsilon_X
\end{equation}

hvor $\epsilon_Y \sim N(0, \sigma_Y^2)$ og $\epsilon_X \sim N(0, \sigma_X^2)$

\textbf{Forklar hvordan denne kan stemme overens med modellen præsenteret ovenfor}

\begin{equation}
    \E(Y) = 0
\end{equation}

Modellen ovenfor var implicit en betinget model for $Y$:

\begin{align}
    \E(Y \mid X) &= \E( \beta X + \epsilon \mid X) \\
    &= \beta X
\end{align}

Da $\E(\epsilon \mid X) = 0$

Altså så før så vi på en betinget model, men nu er det to marginale modeller, som er opstillet!

Man bruger altså i den betingede model information om hvordan en aktie samvarierer med markedet


\horizline

\subsection{Øvelse 16}

\textbf{9/11/2018, opgaver: 45.5, 45.6, 45.7}

\subsubsection{Opgave U45.5}

Lav i klassen!

\begin{itemize}
    \item $(Y, X)$ er en 2-dimensionel stokastisk vektor
    \item $(X,Y)$ er fordelt med $N(\mu, \Omega)$
    \item Vi ved at:
    \begin{equation}
        \E(Y \mid X) = X\qquad \Var(Y \mid X) = 1
    \end{equation}
    \item derudover ved vi:
    \begin{equation}
        \E(X) = 0\qquad \Var(X) = 1
    \end{equation}
\end{itemize}

Vi kan først konkludere at:

husk at $\omega = \sigma_{YX}/\sigma_{X}^2$

\begin{align}
    \E(Y\mid X) &= \mu_Y + \omega(X - \mu_X) \\
    &= \E(\mu_Y + \omega X) \\
    &= \E(\mu_Y + (\sigma_{XY}/ \sigma^2_{X}) X) \\
    &= \E(\mu_Y + \sigma_{XY} X) \\
    &= \E(\mu_Y) + \sigma_{XY}\E(X)
\end{align}

Fra det kan vi konkludere at: $\E(\mu_Y) + \sigma_{XY}\E(X) = X \implies$ $\mu_Y = 0$ og $\sigma_{XY} = 1$.

Vi udnytter dette:

\begin{align}
    \Var(Y \mid X) &= \sigma_Y^2 - \omega \sigma_{XY} \\
    &= \sigma_Y^2 - \frac{\sigma_{YX}}{\sigma_X^2}\sigma_{XY} \\
    &= \sigma_Y^2 - \frac{1^2}{1^2} \\
    &= \sigma_Y^2 - 1 = 1
\end{align}

Hvorfra vi kan konkludere at $\sigma_Y^2 = 2$

Vi kan herfra opskrive den funktionelle form:

\begin{equation}
    \mu = \lp 
    \begin{array}{cc}
         0 \\
         0 
    \end{array}\rp \qquad
    \Omega = \lp
    \begin{array}{cc}
        2 & 1 \\
        1 & 1
    \end{array} \rp
\end{equation}

\subsubsection{Opgave U45.6}

\begin{itemize}
    \item $Z_1 \independent Z_2$
    \item $Z_1, Z_2 \sim N(0,1)$
    \item VI har følgende to stokastiske variable:
    \begin{align}
        Y &:= 2Z_1 +Z_2 \\
        X &:= 3Z_1
    \end{align}
\end{itemize}

\textbf{Del 1) Hvordan er $(Y, X)$ fordelt}

Denne opgave følger eksemplet i Sørensen 8.3.3 meget tæt

Husk $\E(Z_1) = \E(Z_2) = 0$ og $\Var(Z_1) = \Var(Z_2) = 1$

Vi ser hurtigt:

\begin{equation}
    \E(Y) = \E(2Z_1 +Z_2) = 2\E(Z_1) + \E(Z_2) = 0 + 2\cdot0 = 0
\end{equation}

\begin{equation}
    \E(X) = \E(3Z_1) = 3\E(Z_1) = 0
\end{equation}

Variansen af de to er:

\begin{equation}
    \Var(Y) = \Var(2Z_1 +Z_2) = 2^2 \Var(Z_1) + \Var(Z_2) = 4 + 1 = 5
\end{equation}

\begin{equation}
    \Var(X) = \Var(3Z_1) = 3^2 \Var(Z_1) = 9
\end{equation}

Nu skal kovariansen findes mellem $X$ og $Y$:

Kig formlen på s. 236 Sørensen. Her ser vi at:

\begin{equation}
    \Cov(X, Y) = ab
\end{equation}

hvor $a$ og $b$ er de konstanter der ganget på den stokastiske variabel som er går igen i udtrykket for henholdsvis $X$ og $Y$. Altså i vores tildælde $2$ for $Y$, og $3$ for $X$.

\begin{equation}
    \Cov(X,Y) = 2 \cdot 3 = 6
\end{equation}

Vi har med andre ord $(Y,X)$ er distribueret med $N(\mu, \Omega)$, hvor:

\begin{equation}
    \mu = \lp
    \begin{array}{cc}
         0\\
         0 
    \end{array}
    \rp \qquad
    \Omega = \lp
    \begin{array}{cc}
        5 & 6 \\
        6 & 9
    \end{array} \rp
\end{equation}

\textbf{Del 2) Find $\E(Y \mid Z_1)$}


\begin{align}
    \E(Y \mid Z_1) &= \E(2Z_1 +Z_2 \mid Z_1) \\
    &= 2\E(Z_1 \mid Z_1) + \E(Z_2\mid Z_1) \\
    &= 2Z_1
\end{align}

\textbf{Del 3) Find $\E(X \mid Z_2)$}

\begin{align}
    \E(X \mid Z_2) &= \E(3Z_1 \mid Z_2) \\
    &= 0
\end{align}

\textbf{Del 4) Find $\E(Y \mid X)$}

Husk $\omega = \sigma_{XY} / \sigma_X^2$

\begin{align}
    \E(Y \mid X) &= \mu_Y + \omega(X - \mu_X) \\
    &= 0 + \frac{\sigma_{XY}}{\sigma_{X}^2}(X - 0) \\
    &= \frac{6}{9}X = \frac{2}{3}X
\end{align}

\subsubsection{Opgave U45.7}

\begin{itemize}
    \item $X$ er diskret ligefordelt på $\{-1, 0, 1\}$
    \item $Y$ er kontinuært ligefordelt på intervallet $(-1,1)$.
\end{itemize}

\textbf{Del 1) Find $\E(X)$, $\Var(X)$, $P(X>0)$}

Find relevante formler på wiki

\begin{equation}
    \E(X) = (a + b)/2 = (-1 + 1)/2 = 0
\end{equation}

\begin{equation}
    \Var(X) = \frac{(b - a + 1)^2 - 1}{12} = \frac{(1 - (-1) +1)^2 -1}{12} = \frac{8}{12} = \frac{3}{4}
\end{equation}

\begin{equation}
    P(X > 0) = \frac{\text{\# Gunstige}}{\text{\# Mulige}} = \frac{1}{3} 
\end{equation}

\textbf{Del 2) Find $\E(X \mid X >0)$}

\begin{equation}
    \E( X \mid X > 0) = 1
\end{equation}

klart da $X \mid X > 0$ kun kan antage værdien 1.

\textbf{Del 3) Find $\E(Y)$ og $\Var(Y)$ samt $P(Y>0)$}

\begin{equation}
    \E(Y) = (a + b)/2 = (-1 + 1)/2 = 0
\end{equation}

\begin{equation}
    \Var(Y) = \frac{(b - a)^2}{12} = \frac{((1 - (-1) )^2}{12} = \frac{4}{12} = \frac{1}{3}
\end{equation}

\begin{equation}
    P(Y > 0) = \int_{0}^{1} \frac{1}{2}\1_{(-1,1)}(x) dx = \frac{1}{2}\lp (1) - (0) \rp = \frac{1}{2} 
\end{equation}

\textbf{Del 4) Find $\E(Y \mid Y >0)$}

husk $P(X,Y) = P(X \mid Y) P(Y)$ - hvor $X,Y$ er arbitrære navne for at illustrere matematikken.

\begin{align}
    \E(Y \mid Y>0) &= \int_{-\infty}^{\infty} y \frac{P(Y, Y>0)}{P(Y > 0)} dy\\&= \int_{-\infty}^{\infty} y 
    \frac{\frac{1}{2}\1_{(-1 , 1)}(y) \1_{(0,1)}(y)}{\frac{1}{2}} dy \\
    &= \int_0^1
    y \1_{(0,1)}(y)\\
    &= \lsp \frac{1}{2} y^2 \rsp_0^{1} = \frac{1}{2}
\end{align}

\horizline

\subsection{Øvelse 17}

\textbf{22/10/2018, opgaver: 1, 2, 3}

\subsubsection{Opgave 1}

Kig do file do\_1\_17

\subsubsection{Opgave 2}

\textbf{Del 1}

Figur A: Histogram (og kernel density estimation). Viser tæthedsfunktionen (eller en approksimation).

Figur B: Viser den empiriske CDF.

Figur C: Q-Q plot er et plot der viser der modholder den empiriske distribution med en parametrisk - i dette tilfælde den gaussiske distribution. Dette er gjort ved at sammenligner quantiler.

Figur D: Boxplot - giver indblik i antal outliers samt hvordan de kvartiler, og median er fordelt.

\textbf{Del 2}

A) Medianen er den observation som er svarer til det punkt hvor $F(x) = 0.5$. Altså med andre ord $F^{-1}(0.5) =\text{median} $

B) Kig boxplot. Ja det synes der at være. Det er dog altid svært at vurdere outliers.

C) Ja, vi ser at fordelingen er centreret omkring en middelværdi og er stort set symmetrisk og unimodal.

D) Kig CDF. Omkring halvdelen af alle firmaerne.

C) 10\% fraktilen (hvilket kaldes 10\%  percentilen). Angiver det punkt hvor $F(x)= 0.1 $. I vores konkrete tilfælde cirka $- 500$

E) Vi kan se på Q-Q plottet at distribution er lidt lang i halerne, men det er ikke klart om den er venstre eller højre skæv. Derudover er det ikke klart om disse afvigelser i halerne er nok, til at antage den skulle være venstre skæv eller højre skæv.

\subsubsection{Opgave 3}

\textbf{Del 1}

$gns\_gym$: Man finder at gennemsnittet fra gymnasiet er kontinuær.

$studietimer$: Man finder at antal timer brugt på studie er tælle data.





\horizline

\subsection{Øvelse 19}

\textbf{19/11/2018, opgaver: 1, 2}

\subsubsection{Opgave 1}

Lav i klassen!

\textbf{Del 1) Opskriv udfaldsrummet $\Y$}

Vi ser hurtigt at en $Y_i$ må være defineret på de positive naturlige tal: $\N_+$ inkusiv 0.

Vi ved parameter rummet $\Theta$ må have følgende restriktioner: i poisson fordelingen er middelværdi og varians begge $\lambda$. Da vi ved at variansen er positive må $\lambda$ være defineret på den reello postive akse: dvs $\Theta =\R_+ $

\textbf{Del 2)}

Kig side 62. eksempel

sandsynlighedsfunktion

\begin{equation}
    l(\lambda \mid y_i) = f_{Y_i}(y_i \mid \lambda) = \frac{\exp(-\lambda) \lambda^{y_i}}{y_i !}, \qquad y_i \in \N
\end{equation}

Hvis man har mere information om branchen, ville man skulle betinge på det.

\textbf{Del 3)}

Man husker at uafhængighed har implikationen:

\begin{equation}
    f(x,y) = f(x) f(y)
\end{equation}

Vi opskriver den samlede sandsynlighedsfunktion!

\begin{equation}
    f_{Y_1, Y_2 \cdots Y_n} (y_1, y_2, \cdots, y_n \mid \lambda) = \prodn \frac{\exp(-\lambda)\lambda^{y_i}}{y_i !}
\end{equation}


\textbf{Del 4)}

Igen kig side 62

\begin{equation}
    L(\lambda \mid y_1 ,y_2 \cdots y_{n})  =\prodn \frac{\exp(-\lambda)\lambda^{y_i}}{y_i !}
\end{equation}


Nu opskrives log-likelihood funktionen:

\begin{equation}
    \log L (\lambda \mid y_1 ,y_2 \cdots y_{n}) = \sumn y_i \log(\lambda) - \lambda - \log(y_i !)
\end{equation}

Dette kan igen om skrives til


\begin{equation}
     = \log(\lambda) \sumn y_i - \lambda n - \sumn log(y_i !)
\end{equation}

Forskellen mellem $y_i$ og $Y_i$ er om vi overvejer problem som værende realisationer eller stokastiske variable. DVS. vores endelig estimat $\hat{\theta}$ er enten et tal ( realisation) eller en stokastisk variabel (stokastiske variable).

\textbf{Del 5) Find første ordens betingelsen}

\begin{equation}
    \frac{\partial }{\partial \lambda} L (\lambda \mid y_1, y_2, \cdots, y_n) = \frac{1}{\lambda} \sumn y_i -n
\end{equation}

\textbf{Del 6) Løs of find $\hat{\lambda} $}

Kig side 69.

Vi ved (eller antager) at vi har et har at gøre med et konvekst optimerings problem. Vi finder maksimum ved at sætte:

\begin{align}
    \frac{1}{\hat{\lambda}} \sumn y_i -n &= 0 \\
    \implies \quad \hat{\lambda} &= \frac{1}{n}\sumn y_i
\end{align}

\textbf{Del 7) Find 2. ordens betingelsen og argumenter for det er et maksimum}

\begin{equation}\label{eq:opg1}
    H (\lambda) = \frac{- \sumn y_i}{\lambda^2}
\end{equation}

som er negativt, hvilket betyder vi har vist at et hvert ekstremum må være et unikt ekstremum.

\textbf{Del 8)}

 Vi har i opgave teksten oplyst at $\sumn y_i = 63$

Vi bruger \ref{eq:opg1} og indsætter summen af $y_i$:

\begin{equation}
    \hat{\lambda} = \frac{1}{n} \sumn y_i = \frac{1}{21} 63 = 3
\end{equation}

Vi har altså fundet estimatet!

\textbf{Del 9) Hvad hvis summen var 0?}

Så ville $\lambda = 0$, hvilket ikke ville være i overensstemelse med vores parameter rum $\Theta$, som kræver at $\lambda > 0$.

\subsubsection{Opgave 2}

\begin{itemize}
    \item $Y_i \in \Y = \lcp y \in \R \mid y > 0\rcp$
    \item $Y_i \sim Exponential(\theta)$
    \item $f_{Y_i} (y \mid \theta) = \theta \exp (- \theta y)$
\end{itemize}

\textbf{Del 1) Hvad er Y}

$Y_i$ er en stokastisk variabel hvor $y_i$ er en realisation. Table 3.1 på side 55, beskriver forskellen. Det betyder også om vi ender med et estimat (et tal). Eller en Estimator som er en stokastisk variabel.

$y$ indgår i tæthedsfunktionen og relaterer til hvordan sandsynlighedsmassen er ved punktet $y$. Man kan ikke sige at det er sandsynligheden da punktsandsynligheden er 0 - dvs: $P(Y=y) = 0$.

\textbf{Del 2)}

Man kan se at $\theta$ ikke er på individ niveau. Altså vi antager at $\theta_i = \theta$. Og altså at alle stokastiske variable er identisk distribueret.

Dette er vigtigt, da ellers ville man ikke kunne lave maksimerering. Vi antager nemlig at vi kan finde et parameter $\theta$. Hvis de ikke var identisk fordelt skulle vi finde et parameter for hvert enkelt stokastisk variabel.

\textbf{Del 3) Opskrive sample likelihood funktionen}

Vi opskriver likelihood contribution:

\begin{equation}
    l(\theta \mid y_i) = f_{Y_i} = (y_i \mid \theta) = \theta \exp (-\theta y_i)
\end{equation}

Vi kan nu opskrive sample likelihood funktionen:

\begin{equation}
    L(\theta \mid y_1,  \cdots , y_n ) = \prodn l(\theta \mid y_i) = \prodn \theta \exp (-\theta y_i)
\end{equation}

Vi opskriver nu log-likelihood funktionen:

\begin{equation}
    \log L (\theta \mid y_1, \cdots, y_n) = \sumn \log(\theta) - \theta y_i = n \log(\theta) - \theta \sumn y_i
\end{equation}

OPSKRIV OGSÅ MED $Y_i$ altså så vi modellerer med stokastiske variable i stedet for realisationer.

Dette vil have implikationen at vi fik en estimator til sidst i stedet for et estimat.

\textbf{Del 4) HVorfor er uafhængighed en vigtig antagelse}.

Uden uafhængighed kunne man ikke skrive den simultane tæthed op som produktet af marginale tætheder. Dette er kun muligt under antagelse af uafhængighed. Derfor meget vigtigt!

\textbf{Del 5) Find maximum likelihood estimatoren}

Spørgsmål til klassen: `` Skal vi nu bruge store $Y$ eller lille $y$''

Svar: store $Y$

\begin{equation}
    S(\theta) = \frac{\partial}{\partial \theta} = n \frac{1}{\theta} - \sumn Y_i
\end{equation}

Vi finder nu estimatoren:

\begin{equation}
    n \frac{1}{\hat{\theta}} - \sumn Y_i = 0 \implies \frac{n}{\sumn Y_i}  =  \hat{\theta}
\end{equation}

\textbf{Del 6) Find maximum likelihood estimatet}

Vi ved at $\sumn y_i = 252$. $n=120$

Nu finder vi et estimat $\implies$ vi går fra lille $y$ til store $Y$.

\begin{equation}
    \hat{\theta} = \frac{120}{252} = 0,476190
\end{equation}

\textbf{Del 7) Forskellen på estimat og estimator}

Estimatet er et tal! Estimator er en stokastiske variabel

Hvis man læser om et estimat på $\theta = 0.2$ så snakkes der om et estimat.

En estimator kan have lille varians.


\horizline

\subsection{Øvelse 20}

\textbf{19/11/2018, opgaver: 3, 4, 5}

\subsubsection{Opgave 3}

\begin{itemize}
    \item Pærerne fra opgave 2 er ikke identiske fordelt! Deres distribution er rigtigt:
    \begin{equation}
    Y_i =
    \begin{cases}
        exponential(\theta_1) & i \in \{ 1, 2, 3, \ldots, 75\} \\
        exponential(\theta_2) & \in \{ 76, 77, \ldots, 120\}
    \end{cases}
    \end{equation}
    \item $Y_i \independent Y_j$, for $i \neq j$
\end{itemize}

Parameterrummet:

\begin{equation}
\theta = (\theta_1 , \theta_2) \in \Theta
\end{equation}

Hvor at $\Theta = (0, \infty)^2$

Vi har likelood sample funktionen:

\begin{equation}
    L_n (\theta \mid y_i) = \prod_{i=1}^{75} \theta_1 \exp(- \theta_1 y_i) \prod_{i=76}^{120}\theta_2 \exp(- \theta_2 y_i)
\end{equation}

\textbf{Opskriv log-likelihood funktionen for datasættet $\{ y_i \}^{n}_{i=1}$}

Vi tager logaritmen til udtrykket ovenfor:

\begin{equation}
    \log L_n (\theta \mid y_i) = 75 \log(\theta_1) - \theta_1 \sum_{i=1}^{75}  y_i + (120 - 75)\log(\theta_2) - \theta_2 \sum_{i=76}^{120}  y_i
\end{equation}

\textbf{Del 3) Find maximum likelihood estimatoren $\hat{\theta}(Y_1 , \ldots, 120)$}

Vi tager udgangspunkt i store Y nu, da vi skal finde estimatoren. Vi finder første ordens betingelserne:. $\theta_1, \theta_2$

\begin{equation}
    S(\hat{\theta_1}) = \frac{\partial}{\partial \theta_1} \log L_n (\theta \mid Y_i) = \frac{75}{\hat{\theta_1}} - \sum_{i=1}^{75} Y_i = 0
\end{equation}

Hvilket løst for $\theta_1$ bliver:

\begin{equation}
\hat{\theta_1} = \frac{75}{\sum_{i=1}^{75} Y_i }
\end{equation}

Ligeledes ville det for $\theta_2$ være:
\begin{equation}
\hat{\theta_2} = \frac{(120 - 75)}{\sum_{i=76}^{120} Y_i }
\end{equation}

\textbf{Del 4) Vi får nu fortalt at: $\sum_{i=1}^{75}  y_i = 160$ og $\sum_{i=76}^{120}  y_i = 92$}


Vi indsætter i estimatoren og finder estimatet (estimaterne):

\begin{equation}
    \theta = ( \theta_1, \theta_2) = (0.468, 0.489)
\end{equation}

Spørg i klassen: Er de nye pærer bedre end de gamle?

\textbf{Del 5) Forklar hvordan man fortolke dette som en betinget model}

Man kan nu betinge levetiden på om det er en ny eller gammel pære! Hvis man har information om dette, kan man sige mere præcist noget om forventet levetid af pæreren.

\subsubsection{Opgave 4}

\begin{itemize}
    \item Fan chart for inflation fra bank of England.
\end{itemize}

\textbf{Del 1) Er fordelingen skewed. Hvad kan skyldes dette?}

Man kan forestille sig at historisk har man udregnet momenterne og fundet fordelingen skewed opad. Det vil sige at  engang imellem har man historisk oplevet meget høj inflation, men sjældent meget lav inflation.

Uddyb mere i klassen? Spørg klassen.

\textbf{Del 2) Næste periode (dvs kvartal 4. 2004) er fordelt $N(\mu = 2.6, \sigma^2 = 0.56)$}

Kig github.
\subsubsection{Opgave 5}

\begin{itemize}
    \item $\Y = \{0, 1, 2, 3\}$
    \item
    \begin{equation}
    P(Y_i = y_i) = \begin{cases}
        \frac{2\theta}{3} & y_i=0\\
        \frac{\theta}{3} & y_i=1 \\
        \frac{2(1-\theta)}{3} & y_i=2 \\
        \frac{1- \theta}{3} & y_i= 3
    \end{cases}
    \end{equation}
    \item $\theta$ er mellem 0 og 1
\end{itemize}

\textbf{Del 1) Vis sandsynlighedsfunktionen kan skrives som nedenfor}

\begin{equation}
    f_{Y_i}(y \mid \theta) = \lp \frac{2}{\theta}\rp^{\1(y_i = 0)} \lp \frac{\theta}{3} \rp^{\1(y_i = 1)} \lp \frac{2(1-\theta)}{3} \rp^{\1(y_i = 2)} \lp  \frac{1- \theta}{3} \rp^{\1(y_i = 3)}
\end{equation}

Dette er klar hvis man har et udtryk:

\begin{equation}
    \lp \cdot \rp^{\1(\text{condition})}
\end{equation}

Hvis betingelsen er sand får man udtrykket i parantesen ellers får man 1, da ethvert udtryk opløftet i 0 er 1.

Uddyb eksempel på tavlen.

\textbf{Opskriv sample likelihood funktionen og loglikelihood funktionen}

VI ser at likelikehood bidraget er:

\begin{equation}
    l(\theta \mid y_i) = f_{Y_i}(y_i \mid \theta)
\end{equation}

Og vi har at sample likelihood funktionen er:

\begin{equation}
    L (\theta \mid y_i) = \prodn l(\theta \mid y_i)
\end{equation}

\begin{equation}
    L (\theta \mid y_i) = \prodn \lp \frac{2}{\theta}\rp^{\1(y_i = 0)} \lp \frac{\theta}{3} \rp^{\1(y_i = 1)} \lp \frac{2(1-\theta)}{3} \rp^{\1(y_i = 2)} \lp  \frac{1- \theta}{3} \rp^{\1(y_i = 3)}
\end{equation}

vi tager logaritmen:

\begin{align}
    \log L(\theta \mid y_i) &=\sumn \1(y_i = 0) \log \lp \frac{2}{\theta}  \rp + \1(y_i = 1) \log \lp \frac{\theta}{3} \rp \\
    &+ \1(y_i = 2) \log \lp \frac{2(1-\theta)}{3} \rp +  \1(y_i = 3) \log \lp  \frac{1- \theta}{3} \rp
\end{align}

Vi kan transformere summen, da vi indser vi har:

\begin{itemize}
    \item 2 observationer = 0
    \item 3 observationer = 1
    \item 3 observationer = 2
    \item 2 observationer = 3
\end{itemize}

\begin{align}
    \log L(\theta \mid y_i) &=2 \log \lp \frac{2}{\theta}  \rp + 3 \log \lp \frac{\theta}{3} \rp \\
    &+ 3 \log \lp \frac{2(1-\theta)}{3} \rp +  2 \log \lp  \frac{1- \theta}{3} \rp
\end{align}

Dette kan omskrives til:

\begin{equation}
    \log L(\theta \mid y_1, y_2 \cdots y_n) = 5 \lsp \log \lp \frac{2}{9}\rp  + \log(\theta - \theta^2) \rsp
\end{equation}

(Folk kan lave omskrivningen i klassen) - Hint dem til Frederiks rettevejledning.


\textbf{Del 3) Tegn $\log L_n$ som funktion af $\theta$}

Kig github.

\textbf{Find maximum likelihood estimatet}

\begin{equation}
    S(\hat{\theta}) = \frac{\partial}{\partial \hat{\theta}} \log L_n (\hat{\theta } \mid y_i) =5 \frac{1}{\theta - \theta^2} (1 - 2\theta) = 0 \implies \hat{\theta} = 0.5
\end{equation}

Vi har med givne $\hat{\theta} = 5$:



\begin{equation}
    P(Y_i = y_i) = \begin{cases}
        \frac{2\cdot 0.5}{3} = 1/3 & y_i=0\\
        \frac{0.5}{3} = 1/6 & y_i=1 \\
        \frac{2(1-0.5)}{3} = 1/3 & y_i=2 \\
        \frac{1- 0.5}{3} = 1/6 & y_i= 3
    \end{cases}
\end{equation}


Observeret:

2/10, 3/10, 3/10, 2/10

Men vi har et småt sample. Det er meget svært at lave konklusioner!

%\fi

\horizline

\subsection{Øvelse 21}

\textbf{26/11/2018, Opgaver: 1, 2, 3}



\horizline

\subsection{Øvelse 22}

\textbf{30/11/2018, Opgaver: 4, 5}

\subsubsection{Opgave 4}

side 65 i Heino's bog

\begin{itemize}
    \item $\{Y_i\}_{i=1}^{n}$ er i.i.d. normalt fordelte stokastiske variable
    \item altså:
    \begin{equation}
        Y_i \sim N(\mu, \sigma^2)
    \end{equation}
    \item tæthedsfunktionen:
    \begin{equation}
        f_{Y_{i}} (y \mid \mu, \sigma^2) = \frac{1}{\sqrt{2\pi \sigma^2}} \exp \lp \frac{-(y- \mu)^2}{2\sigma^2}\rp
    \end{equation}
\end{itemize}

\textbf{Del 1.a) Opskriv likelihood funktionen $L(\mu, \sigma^2 \mid Y_1, Y_2 \cdots Y_n)$}

likelihood contribution:

\begin{equation}
    l (\mu, \sigma^2 \mid Y_i) =  \frac{1}{\sqrt{2\pi \sigma^2}} \exp \lp \frac{-(Y_i - \mu)^2}{2\sigma^2}\rp
\end{equation}

Likelihood funktionen:

\begin{equation}
    L (\mu, \sigma^2 \mid Y_i) = \prodn \frac{1}{\sqrt{2\pi \sigma^2}} \exp \lp \frac{-(Y_i - \mu)^2}{2\sigma^2}\rp
\end{equation}

\textbf{Del 1.b) Opskriv log likelihood funktionen}

Vi tager logaritmen:

\begin{equation}
    \log L (\mu, \sigma^2 \mid Y_1, Y_2, \cdots Y_n) = \sumn - \frac{1}{2}\log(2\pi) - \frac{1}{2}\log(\sigma^2) + \frac{- (Y_i - \mu)^2}{2\pi^2}
\end{equation}

\textbf{Del 1.c) Find estimatorerne}

Kig side 70.

\begin{equation}
    \frac{\partial}{\partial \mu} \log L (\mu, \sigma) = \sumn \frac{Y_i - \mu}{\sigma^2}
\end{equation}

\begin{equation}
    - \frac{1}{2\sigma^2} + \frac{(Y_i - \mu)^2}{2\sigma^4}
\end{equation}

Sæt disse lig med 0 og vi kan se at:

\begin{equation}
    \hat{\mu} = \frac{1}{n}\sumn Y_i
\end{equation}

\begin{equation}
    \hat{\sigma^2} = \frac{1}{n}\sumn (Y_i - \hat{\mu})^2
\end{equation}

Dette er de samme formler som blev præsenteret i starten af bogen for det empiriske gennemsnit og varians.

\textbf{Del 2) Find estimaterne}

Vi har givet at $\sumn y_i = 627.6$ og $\sumn y_i^2 = 3807.2$. v har at $n=125$

Vi bruger hintet givet i opgaven:

\begin{equation}
    \frac{1}{n}\sumn (y_i - \hat{\mu})^2 = \sumn y_i^2 + n \hat{\mu}^2 - 2\hat{\mu} \sumn y_i
\end{equation}

Vi finder først estimatet for $\mu$

\begin{equation}
    \hat{\mu} = \frac{1}{125} \cdot 627.6 = 5.0208
\end{equation}

Herefter finder vi det for $\sigma^2$

note: Skriv måske formlen op igen for variansen!

\begin{equation}
    \hat{\sigma^2} = 3807.2 + 125 \cdot 5.0208^2 - 2\cdot5.0208 \cdot 627.6
\end{equation}


\textbf{Likelihood og Log Likelihood (under equi-dispersion $\mu = \sigma^2 = \phi$)}


Dette tilfælde kaldes \textit{equi dispersion}.

tætheden er givet ved:

\begin{equation}
    f_{Y_i} (y \mid \phi) = \frac{1}{\sqrt{2\pi \phi}} \exp \lp \frac{- (y - \phi)^2 }{2\phi}\rp
\end{equation}

Vi opskriver Likelihood funktionen:

\begin{equation}
    L(\phi \mid Y_1, Y_2 \cdots Y_n) = \prodn  \frac{1}{\sqrt{2\pi \phi}} \exp \lp \frac{- (Y_i - \phi)^2 }{2\phi}\rp
\end{equation}

Vi opskriver også log likelihood funktionen

\begin{equation}
    \log L( \phi) = \sumn \frac{1}{2}\log(2\pi) - \frac{1}{2}\log (\phi) + \lp -\frac{(Y_i - \phi)^2}{2\phi} \rp
\end{equation}

Som kan omskrives til:

\begin{equation}
    \log L( \phi) = \frac{n}{2}\log(2\pi) - \frac{n}{2}\log (\phi)  -\frac{ \sumn (Y_i - \phi)^2}{2\phi}
\end{equation}

\textbf{Del 4) Udled scoren}

\begin{equation}
    S(\phi) = \frac{\partial}{\partial \phi} \log L (\phi \mid Y_1, \cdots , Y_n) = \frac{\sumn Y_i - n\phi^2 - n\phi}{2\phi^2}
\end{equation}

Man ser man kan brække udtrykket (fra før) over i to dele når man differentiere:


Den første del $\frac{n}{2}\log (\phi)$
\begin{equation}
    \frac{\partial}{\partial\phi} \lp \frac{n}{2}\log (\phi) \rp = \frac{n}{2\phi}
\end{equation}

Den anden del:

\begin{equation}
    \frac{\partial}{\partial \phi} \lp  -\frac{ \sumn (Y_i - \phi)^2}{2\phi} \rp = - \sumn \frac{\partial}{\partial \phi} \lp  \frac{ (Y_i - \phi)^2}{2\phi} \rp
\end{equation}

Dette skrives som (husk hint i første skridt - differentiation af brøker)
:

\begin{align}
    - \sumn \frac{\partial}{\partial \phi} \lp  \frac{ (Y_i - \phi)^2}{2\phi} \rp &= - \sumn \frac{(-2(Y_i - \phi)2\phi) - (2(Y_i - \phi)^2)}{4\phi^2} \\
    &= - \sumn \frac{-4Y_i\phi + 4\phi^2 - 2Y_i^2 - 2\phi^2 + 4Y_i\phi}{4\phi^2} \\
    &= - \sumn \frac{2 \phi^2 - 2 Y_i^2}{4\phi^2} \\
    &= - \sumn \frac{\phi^2 - Y_i^2}{2\phi^2} \\
    &= \frac{\sumn( Y_i ) - n \phi^2}{2\phi^2} \\
\end{align}

Hvis vi sætter resultaterne sammen får vi:

(husk minusset fra det orindelige udtryk)

\begin{equation}
     \frac{\sumn( Y_i ) - n \phi^2}{2\phi^2} - \frac{n}{2\phi}  =
\end{equation}

VI ser at:

\begin{equation}
    \frac{n}{2\phi}  = \frac{n\phi}{2\phi^2}
\end{equation}

sådan at:

\begin{equation}
     \frac{\sumn( Y_i ) - n \phi^2}{2\phi^2} - \frac{n\phi}{2\phi^2}  = \frac{\sumn( Y_i ) - n \phi^2 - n\phi}{2\phi^2}
\end{equation}

Som var det vi skulle vise!

\textbf{Del 5) Find estimatoren}

\begin{align}
     \frac{\sumn( Y_i ) - n \hat{\phi}^2 - n \hat{\phi}}{2\hat{\phi}^2} &= 0
\end{align}

\begin{equation}
    \implies \qquad \sumn( Y_i ) = n \hat{\phi}^2 + n\hat{\phi}
\end{equation}


Kan løses med solver:

\begin{equation}
\hat{\phi} = \frac{\sqrt{4 \frac{1}{n} \sumn (Y_i)  + 1} - 1}{2}
\end{equation}

\textbf{Del 6) Find estimatet}

indsætter $n=125$ og $\sumn Y_i = 627.6$

\begin{equation}
\hat{\phi} = \frac{\sqrt{4 \cdot \frac{627.6}{125}  + 1} - 1}{2} = (\sqrt{21,08} - 1) / 2 = (4.591 - 1) / 2 = 1,795
\end{equation}





\end{document}
