\subsection{Kapitel 7: Model kontrol}

\subsubsection{Den prediktive distribution}

Man har antaget en distribution ogfundet estimatet $\htn$. Hvordan kan vi vide, vi har valgt den rigtige distribution?

En metode er, at sammenligne den empiriske distribution, men den predikterede distribution. Måder at gøre dette på er:

\begin{itemize}
    \item Histogram af empirisk distribution (alternativt kde), mod den predikterede distribution.
    \item plotte den empiriske CDF med den predikterede CDF.
    \item plotte et QQ-plot af de predikterede kvantiler mod de empiriske.
\end{itemize}

\subsubsection{Misspecifikation tests}

Læs bog for eksempler.

Pointen er at man antager en mere generel model, hvor den man har konstrueret til at starte med er en restrikteret udgave. Hvis man ikke kan afvise den restrikterede model, kunne det pege på at det er den sande model til at beskrive data.
