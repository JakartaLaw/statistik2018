\horizline

\subsection{Øvelse 3}

\textbf{17/9/2017, Opgaver: 2.1 og 2.3 fra Sørensen (2015) samt opgaverne B.1, B.2 og B.3}
\subsubsection{Opgave 2.1}

\begin{itemize}
    \item 1 rød terning
    \item 1 sort terning
    \item $Y := \min(r,s)$
    \item $Z := \max(r,s)$
\end{itemize}

\textbf{Fordelingen for Y}

\textbf{TEGN TERNINGEMATRICEN}

\begin{align}
    P(Y = 1) &= P(\{(1,1),(1,2),\cdots,(1,6),(2,1),\cdots,(6,1)\}) = \frac{11}{36} \\
    P(Y = 2) &= P(\{(2,2),(2,3), \cdots (2,6),(3,2) \cdots (6,2) \})= \frac{9}{36} \\
    P(Y = 3 ) &= \cdots = \frac{7}{36}
\end{align}

Den resterende fordeling for $Y$ er: $P(Y=4) = \frac{5}{36}, P(Y=5) = \frac{3}{36}, P(Y=6) = \frac{1}{36}$.

\textbf{Fordelingen for Z}

\textbf{TEGN TERNINGEMATRICEN}

\begin{align}
    P(Z = 1) &= P(\{(1,1)\}) = \frac{1}{36} \\
    P(Z = 2) &= P(\{(2,1),(2,2),(2,1) = \frac{3}{36} \\
    P(Z = 3 ) &= \cdots = \frac{5}{36}
\end{align}

Den resterende fordeling for $Z$ er $P(Z=4) = \frac{7}{36}, P(Z=5) = \frac{9}{36}, P(Z=6) = \frac{11}{36}$.

Den simultane fordeling er \ref{tab:Y_Z_terning}:


$Y$ er vandret, $Z$ lodret: Vi ved at det må være en øvre trekantsmatrice.

Til diagonalen: Vi ved at der er kun måde at min og maks kan være ens $min(T_1,T_2) = max(T_1, T_2) \implies T_1 = T_2$.

Til den øvre trekant: $Y = 1, Z_2 \implies T_1 = 1, T_2 = 2 \lor T_1=2, T_2 = 1$. Dette kan gøres for alle elementer af den øvre trekant

\begin{table}[ht]
\label{tab:Y_Z_terning}
\centering
\caption{Simultan fordeling}
\begin{tabular}{|l|l|l|l|l|l|l|}
\hline
      & $Y = 1$ & $Y=2$ & $Y=3$ & $Y=4$ & $Y=5$ & $Y=6$ \\ \hline
$Z=1$ & 1/36    & 2/36  & 2/36  & 2/36  & 2/36  & 2/36  \\ \hline
$Z=2$ & 0       & 1/36  & 2/36  & 2/36  & 2/36  & 2/36  \\ \hline
$Z=3$ & 0       & 0     & 1/36  & 2/36  & 2/36  & 2/36  \\ \hline
$Z=4$ & 0       & 0     & 0     & 1/36  & 2/36  & 2/36  \\ \hline
$Z=5$ & 0       & 0     & 0     & 0     & 1/36  & 2/36  \\ \hline
$Z=6$ & 0       & 0     & 0     & 0     & 0     & 1/36  \\ \hline
\end{tabular}
\end{table}


\subsubsection{Opgave 2.3}

\begin{itemize}
    \item Stokastisk variabel er beskrevet i bogen
\end{itemize}

$Y=t(X)$

\begin{align}
    P(Y=1) &= P(X\in\{1,2,3\}) = 0.12 + 0.8 + 0.20 = 0.4 \\
    P(Y=2) &= P(X \in \{4,5\}) = 0.11 + 0.19 = 0.30 \\
    P(Y=3) &= P(X \in \{6,7\}) = 0.14 + 0.06 = 0.20 \\
    P(Y=4) &= P(X \in \{8\}) =0.10
\end{align}

Fordelingsfunktion (CDF):

\begin{align}
    P(Y \leq 0) &= 0 \\
    P(Y \leq 1) &= 0.4 \\
    P(Y \leq 2) &= 0.7 \\
    P(Y \leq 3) &= 0.9 \\
    P(Y \leq 4) &= 1.0 
\end{align}{}

\subsubsection{Opgave B.1}

\begin{itemize}
    \item stokastiske variable $X_1, X_2$
    \item $X_1 = 1$ hvis der var en stor nyhed (ellers 0)
    \item $X_2 = 1$ hvis aktiemarkedet steg/faldt (0 hvis ikke)
    \item $P(X_1 = 1) = \frac{6}{10}$
    \item $P(X_2 = 1) = \frac{3}{10}$
\end{itemize}

\textbf{Simultane fordeling under antagelse af uafhængighed!}

Brug definition 2.4.1 (sørensen)

\begin{align}
    P(X_1 = 0, X_2 = 0) &= P(X_1=0)P(X_2 = 0) = \frac{4}{10} \frac{7}{10} = \frac{28}{100} \\
    P(X_1 = 0, X_2 = 1) &= P(X_1=0)P(X_2=1) = \frac{4}{10}\frac{3}{10} = \frac{12}{100} \\
    P(X_1 = 1, X_2 = 0) &= P(X_1=1)P(X_2=0) = \frac{6}{10}\frac{7}{10} = \frac{42}{10} \\
    P(X_1 = 1, X_2 = 1) &= P(X_1 =1 )P(X_2 = 1) = \frac{6}{10}\frac{3}{10} = \frac{18}{10}
\end{align}

\textit{TEGN BI-MATRICE}

\textbf{DEL 2: Antag IKKE uafhængighed - Hvad er den simultane fordeling $(X_1, X_2)$}

\textit{tegn bimatrice og fyld værdier i løbende!}

\begin{equation}
    P(X_2 = 1 \mid X_1 = 1) = \frac{4}{10}
\end{equation}

Udvid \textbf{Definition 1.4.3}

\begin{equation}
    P(B) = \sum_{j=1}^n P(B \mid A_j)P(A_j) = \sum_{j=1}^n P(B, A_j)
\end{equation}

\begin{equation}
    P(X_2 = 1) = P(X_2 = 1 \mid X_1 =0)P(X_1 = 0) + P(X_2 = 1 \mid X_1 = 1) P(X_1 = 1)  
\end{equation}

\begin{equation}
    P(X_2 = 1) = P(X_2 = 1, X_1 =0) + P(X_2 = 1, X_1 = 0)  
\end{equation}

Vi husker at $P(X_2) = \frac{3}{10}$

\begin{equation}
    \frac{3}{10}= \underset{P(X_1 = 1, X_2 =1)}{\underbrace{\frac{4}{10} \frac{6}{10}}} + P(X_2 =1 \mid X_1 = 0 )P(X_1 = 0)
\end{equation}

\begin{equation}
    \implies P(X_1=0 , X_2 =1) = \frac{6}{100}
\end{equation}

Vi har allerede set at:

\begin{equation}
    P(X_1 = 1, X_2 = 1) = \frac{24}{100}
\end{equation}

Vi går videre:

\begin{equation}
    P(X_1 = 1) = P(X_1 = 1, X_2 = 0) + P(X_1 = 1, X_2 =1)
\end{equation}

Vi indsætter de værdier vi kender:

\begin{equation}
    \frac{6}{10} =   P(X_1 = 1, X_2 = 0) + \frac{24}{100} \implies P(X_1 = 1, X_2 = 0) = \frac{36}{100}
\end{equation}

Vi mangler kun sidste værdi nu:

\begin{equation}
    P(X_2 = 0) = P(X_2=0, X_1=0) + P(X_2=0, X_1 = 1)
\end{equation}

Husker værdier: $P(X_2 = 0) = \frac{7}{10}$ og $P(X_1 = 1, X_2 = 0) = \frac{36}{100}$

\begin{equation}
    \frac{7}{10} = \frac{36}{100} + P(X_1 =0, X_2=0) \implies P(X_1 = 0, X_2=0) = \frac{34}{100}
\end{equation}{}

\textbf{Ændrer fordelingen sig for $X_1$, $X_2$, $X$}

\textit{Spørg klassen}

De marginale distributioner er ens,
De betingede og den simultane er forskellig

\subsubsection{Opgave B.2}

\begin{itemize}
    \item Test for cancer
    \item Den gætter rigtig med 95 \% ssh.
    \item 1 ud af 100.000 mennesker har denne kræft form
\end{itemize}

Lad $X$ for cancer testen $X=1$ implicerer positiv test . Lad $Y$ være en stokastisk variabel som angiver om man har kræft $Y=1$ betyder man har kræft.

Vi kan skitserer nogle sandsynligheder:

\begin{equation}
    P(X=1 \mid Y=1)=0.95, \quad P(X=0 \mid Y=1)=0.05
\end{equation}

\begin{equation}
    P(X=0 \mid Y=0)=0.95, \quad P(X=1 \mid Y=0)=0.05
\end{equation}


\begin{equation}
    P(Y=1) = \frac{1}{100000} = 0.00001
\end{equation}

Brug bayes formel (sætning 1.4.7):

\begin{equation}
    P(A_k \mid B) = \frac{P(B \mid A_k)P(A_k)}{\sum_{j=1}^n P(B\mid A_j)P(A_j)}
\end{equation}


\begin{align}
    &P(Y=1 \mid X=1) \\ &= \frac{P(X=1\mid Y=1)P(Y=1)}{P(X=1 \mid Y=1)P(Y=1) + P(X=1 \mid Y=0)P(Y=0)} 
\end{align}

\begin{equation}
    P(Y=1 \mid X=1) = \frac{0.95 \cdot 0.00001}{0.95 \cdot 0.00001 + 0.05 \cdot 0.99999} = 0.0001899 
\end{equation}

\subsubsection{Opgave B.3}

Kig github!



