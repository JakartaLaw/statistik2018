\horizline

\subsection{Øvelse 2}

\textbf{15/09/2018, opgaver: 1.6, 1.7, 1.9, 1.15, 1.18, 1.28 og 1.30 (og 1.12 hvis der er tid)}

\subsubsection{1.6}
\begin{itemize}
    \item 1 ternning
    \item 2 slag
\end{itemize}

\textbf{Ssh for mindst 1 sekser}

\begin{align}
    &P(\{\text{mindst en sekser}\}) = \\
    &P(\{(1,6),(2,6),\cdots,(6,6),(6,1),\cdots, (6,5) = \\
    &\frac{5 + 6}{36} = \frac{11}{36}
\end{align}

\textbf{Ssh. for mindst 1 sekser eller mindst 1 toer}

\begin{align}
    &P(\{\text{mindst en sekser} \}) = \\
    &P(\{(1,6),(2,6),\cdots,(6,6),(6,1),\cdots,(5,6), \\
    &(1,2),\cdots(5,2),(2,1)\cdots(2,5) \}) = \\
    &\frac{6+5+5+4}{36} = \frac{20}{36}
\end{align}

\subsubsection{Opgave 1.7}

\begin{itemize}
    \item 1 mønt
    \item 10 kast
\end{itemize}

\textbf{Hvad er ssh. for mindst 2 plat}

Find sandsynligheden for komplimenter hændelsen: 

$A:$ Er hændelsen for at få mindst 2 plat.

$A^C :$ Er Komplementær hændelsen - altså maks 1 plat:

\begin{equation}
    A^C = \{\text{slå 0 plat} \} \cup \{\text{slå 1 plat} \}
\end{equation}

\begin{equation}
    P(\{\text{slå 0 plat}\}) = \frac{1}{2^{10}}
\end{equation}

\begin{equation}
    P(\{\text{slå 1 plat}\}) = \frac{10}{2^{10}}
\end{equation}

Noter at $\{\text{slå 0 plat}\} \cap \{\text{slå 1 plat}\} = \emptyset$

\begin{equation}
    P(A^C)=\frac{1}{2^{10}} + \frac{10}{2^{10}} = \frac{11}{2^{10}}
\end{equation}

\begin{equation}
    P(A) = 1 - P(A^C) = 1 - \frac{11}{2^{10}} = \frac{1013}{2^{10}}
\end{equation}

\subsubsection{Opgave 1.9}

\begin{itemize}
    \item 1 spil kort (52 kort)
    \item 13 kort trækkes
\end{itemize}

\textbf{Hvad er Ssh. for 0 billedkort eller esser}

Antal billedkort og esser (kaldet billedkort fra nu): $4*4=16$

Kig på komplementær hændelsen:
\begin{equation}
    P(\{\text{kort 1 ikke billedkort}\}) = \frac{52 - 16}{52}
\end{equation}

Vi har trukket 1 kort nu $\implies$ $51$ kort tilbage, men stadig $12$ billedkort
\begin{equation}
    P(\{\text{kort 2 er billedkort}\}) = \frac{51 - 16}{51}
\end{equation}

\begin{equation}
    P(\{\text{man trækker 0 billedkort}\}) = \prod_{i=0}^{12} \frac{52 - i - 16}{52 - i} = 0.0036
\end{equation}

\textbf{Alternativt}

\begin{equation}
    \# E = 52 \cdot 51 \cdots 40 = \frac{52!}{39!}
\end{equation}

\begin{equation}
    \# A = 36 \cdot 35 \cdots 24 = \frac{36!}{23!}
\end{equation}

\begin{equation}
    P(\{\text{man trækker 0 billedkort}\}) = \frac{\# A}{ \# E} = 0.0036
\end{equation}

\subsubsection{Opgave 1.15}


\begin{itemize}
    \item 4 slag med terning
    \item mindst 1 sekser
    \item demere mente $4\times \frac{1}{6}$
\end{itemize}

\textbf{Hvorfor tog han fejl?}

Klasse diskussion:

Kig på komplementærhændelsen: \textit{Ingen seksere}

\begin{equation}
    P(\{\textbf{Ingen seksere}\}) = (\frac{5}{6})^{4} = \frac{5^4}{6^4} = 0.49
\end{equation}

Da dette er komplementær hændelsen kan vi i stedet sige: 

\begin{equation}
    P(\{\textbf{mindst 1 sekser}\} = 1 - 0.49 = 0.51
\end{equation}

\textbf{Ssh for en dobbelt sekser i 24 kast}

\begin{itemize}
    \item 24 kast
    \item mindst 1 dobbelt sekser
\end{itemize}

Sandsynligheden for 1 dobbelt sekser i et slag.
\begin{equation}
    P(\{\textbf{En dobbelt sekser} \}) = \frac{1}{6}\frac{1}{6} = \frac{1}{36}
\end{equation}

Brug komplementær hændelsen: Dvs. ssh for ikke at få en dobbelt sekser i 24 slag:

\begin{equation}
    P(\{\textbf{Ingen dobbelt sekser i 24 slag} \}) = (\frac{35}{36})^{24} = 0.509
\end{equation}

\begin{equation}
    P(\{\textbf{mindst en dobbelt sekser i 24 slag} \}) = 1 - 0.509 = 0.491
\end{equation}

Så ikke langt fra!

\subsubsection{Opgave 1.18}

\begin{itemize}
    \item 1 mønt
    \item 10 kast
\end{itemize}

\textbf{Hvad er ssh. for at få krone den 10'ende gang givet 9 plat}

Lad os definerer hændelserne:

$A:$ Man har fået 9 plat på de første 9 slag af de 10 slag

$B:$ Man får krone på det sidste slag ud af de 10 slag

Brug definition for betingede ssh (1.4.1):

\begin{equation}
    P(A \mid B) = \frac{P(A \cap B)}{P(A)}
\end{equation}

\begin{equation}
    P(A \cap B) = (\frac{1}{2})^{10} = \frac{1}{2^{10}}
\end{equation}

\begin{equation}
    P(A) = (\frac{1}{2})^{9} = \frac{1}{2^9}
\end{equation}

\begin{equation}
    P(B \mid A) = \frac{P(A \cap B)}{P(A)} = \frac{ \frac{1}{2^{10}}}{\frac{1}{2^9}} = \frac{1}{2}
\end{equation}

\textbf{SSh for den 10 bliver krone, givet 9 af de 10 kast blive plat}

Lad os definerer hændelserne:

$A:$ Man har fået 9 plat ud af de 10 slag

$B:$ Man får krone på det sidste slag ud af de 10 slag

\begin{equation}
    P(A \cap B) = (\frac{1}{2})^{10} = \frac{1}{2^{10}}
\end{equation}

\begin{equation}
    P(A) = 10 \times (\frac{1}{2})^{10} = \frac{10}{2^{10}}
\end{equation}

\begin{equation}
    P(B \mid A) = \frac{1}{10}
\end{equation}

\subsubsection{Opgave 1.28}

\begin{itemize}
    \item 1 terning
    \item 1 kast
    \item Hændelse $A:$ kast er 1,2,3
    \item Hændelse $B:$ kast er 1 eller 4
\end{itemize}

\textbf{Vis at $A$ og $B$ er uafhængige}

Brug Definition 1.5.1: 

\begin{equation}
    P(A \cap B) = P(A) \dot P(B)
\end{equation}

\begin{equation}
    P(A) = \frac{1}{2}    
\end{equation}

\begin{equation}
    P(B) = \frac{1}{3}
\end{equation}

Hvad er fælles mængden af de to hændelser: \textit{at terningen bliver 1}

\begin{equation}
    P(A \cap B) = P(\{\textbf{Terningen bliver 1} \}) = \frac{1}{6} = P(A) \dot P(B)
\end{equation}

Og vi har herved vist, at hændelserne er uafhængige!

\subsubsection{Opgave 1.30}

\textbf{Lad eleverne prøve!}

\begin{itemize}
    \item 3 hændelser: $A, B, C$
    \item $A \independent B$
    \item $A \independent C$
\end{itemize}

\textbf{Kan man fra ovenstående slutte at: 
 $A \independent B \cup C$}

\begin{equation}
    A \independent B \implies P(A)\cdot P(B) = P(A \cap B)
\end{equation}

\begin{equation}
    A \independent C \implies P(A)\cdot P(C) = P(A \cap C)
\end{equation}

Bevis via. modeksempel

$A = \{\textbf{Spar eller hjerter} \}$

$ B = \{\textbf{Spar eller ruder}\}$

$C = \{\textbf{hjerter eller ruder}\}$

$P(A\cap B) = \frac{1}{4} = P(A) P(B)$

$P(A \cap C) = \frac{1}{4} = P(A)P(C)$

$P(A \cap (B \cup C)) = \frac{1}{2} \neq P(A)P(B \cup C) = \frac{1}{2}\frac{3}{4}$

\subsubsection{Opgave 1.12}

\begin{itemize}
    \item 1 slag
    \item 5 terninger
\end{itemize}

\textbf{Sandsynligheden for at få mindst 1 sekser}

Udregn ssh for komplementærhændelsen at få 0 seksere!

Definér hændelsen $A:$ At få mindst 1 sekser
\begin{equation}
    P(A^C) = P(\{\textbf{0 seksere}\} = \left( \frac{5}{6} \right)^5 = 0.402
\end{equation}

\begin{equation}
    1 - A^C = 0.598
\end{equation}