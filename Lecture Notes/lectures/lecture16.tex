\horizline

\subsection{Øvelse 16}

\textbf{9/11/2018, opgaver: 45.5, 45.6, 45.7}

\subsubsection{Opgave U45.5}

Lav i klassen!

\begin{itemize}
    \item $(Y, X)$ er en 2-dimensionel stokastisk vektor
    \item $(X,Y)$ er fordelt med $N(\mu, \Omega)$
    \item Vi ved at:
    \begin{equation}
        \E(Y \mid X) = X\qquad \Var(Y \mid X) = 1
    \end{equation}
    \item derudover ved vi:
    \begin{equation}
        \E(X) = 0\qquad \Var(X) = 1
    \end{equation}
\end{itemize}

Vi kan først konkludere at:

husk at $\omega = \sigma_{YX}/\sigma_{X}^2$

\begin{align}
    \E(Y\mid X) &= \mu_Y + \omega(X - \mu_X) \\
    &= \E(\mu_Y + \omega X) \\
    &= \E(\mu_Y + (\sigma_{XY}/ \sigma^2_{X}) X) \\
    &= \E(\mu_Y + \sigma_{XY} X) \\
    &= \E(\mu_Y) + \sigma_{XY}\E(X)
\end{align}

Fra det kan vi konkludere at: $\E(\mu_Y) + \sigma_{XY}\E(X) = X \implies$ $\mu_Y = 0$ og $\sigma_{XY} = 1$.

Vi udnytter dette:

\begin{align}
    \Var(Y \mid X) &= \sigma_Y^2 - \omega \sigma_{XY} \\
    &= \sigma_Y^2 - \frac{\sigma_{YX}}{\sigma_X^2}\sigma_{XY} \\
    &= \sigma_Y^2 - \frac{1^2}{1^2} \\
    &= \sigma_Y^2 - 1 = 1
\end{align}

Hvorfra vi kan konkludere at $\sigma_Y^2 = 2$

Vi kan herfra opskrive den funktionelle form:

\begin{equation}
    \mu = \lp 
    \begin{array}{cc}
         0 \\
         0 
    \end{array}\rp \qquad
    \Omega = \lp
    \begin{array}{cc}
        2 & 1 \\
        1 & 1
    \end{array} \rp
\end{equation}

\subsubsection{Opgave U45.6}

\begin{itemize}
    \item $Z_1 \independent Z_2$
    \item $Z_1, Z_2 \sim N(0,1)$
    \item VI har følgende to stokastiske variable:
    \begin{align}
        Y &:= 2Z_1 +Z_2 \\
        X &:= 3Z_1
    \end{align}
\end{itemize}

\textbf{Del 1) Hvordan er $(Y, X)$ fordelt}

Denne opgave følger eksemplet i Sørensen 8.3.3 meget tæt

Husk $\E(Z_1) = \E(Z_2) = 0$ og $\Var(Z_1) = \Var(Z_2) = 1$

Vi ser hurtigt:

\begin{equation}
    \E(Y) = \E(2Z_1 +Z_2) = 2\E(Z_1) + \E(Z_2) = 0 + 2\cdot0 = 0
\end{equation}

\begin{equation}
    \E(X) = \E(3Z_1) = 3\E(Z_1) = 0
\end{equation}

Variansen af de to er:

\begin{equation}
    \Var(Y) = \Var(2Z_1 +Z_2) = 2^2 \Var(Z_1) + \Var(Z_2) = 4 + 1 = 5
\end{equation}

\begin{equation}
    \Var(X) = \Var(3Z_1) = 3^2 \Var(Z_1) = 9
\end{equation}

Nu skal kovariansen findes mellem $X$ og $Y$:

Kig formlen på s. 236 Sørensen. Her ser vi at:

\begin{equation}
    \Cov(X, Y) = ab
\end{equation}

hvor $a$ og $b$ er de konstanter der ganget på den stokastiske variabel som er går igen i udtrykket for henholdsvis $X$ og $Y$. Altså i vores tildælde $2$ for $Y$, og $3$ for $X$.

\begin{equation}
    \Cov(X,Y) = 2 \cdot 3 = 6
\end{equation}

Vi har med andre ord $(Y,X)$ er distribueret med $N(\mu, \Omega)$, hvor:

\begin{equation}
    \mu = \lp
    \begin{array}{cc}
         0\\
         0 
    \end{array}
    \rp \qquad
    \Omega = \lp
    \begin{array}{cc}
        5 & 6 \\
        6 & 9
    \end{array} \rp
\end{equation}

\textbf{Del 2) Find $\E(Y \mid Z_1)$}


\begin{align}
    \E(Y \mid Z_1) &= \E(2Z_1 +Z_2 \mid Z_1) \\
    &= 2\E(Z_1 \mid Z_1) + \E(Z_2\mid Z_1) \\
    &= 2Z_1
\end{align}

\textbf{Del 3) Find $\E(X \mid Z_2)$}

\begin{align}
    \E(X \mid Z_2) &= \E(3Z_1 \mid Z_2) \\
    &= 0
\end{align}

\textbf{Del 4) Find $\E(Y \mid X)$}

Husk $\omega = \sigma_{XY} / \sigma_X^2$

\begin{align}
    \E(Y \mid X) &= \mu_Y + \omega(X - \mu_X) \\
    &= 0 + \frac{\sigma_{XY}}{\sigma_{X}^2}(X - 0) \\
    &= \frac{6}{9}X = \frac{2}{3}X
\end{align}

\subsubsection{Opgave U45.7}

\begin{itemize}
    \item $X$ er diskret ligefordelt på $\{-1, 0, 1\}$
    \item $Y$ er kontinuært ligefordelt på intervallet $(-1,1)$.
\end{itemize}

\textbf{Del 1) Find $\E(X)$, $\Var(X)$, $P(X>0)$}

Find relevante formler på wiki

\begin{equation}
    \E(X) = (a + b)/2 = (-1 + 1)/2 = 0
\end{equation}

\begin{equation}
    \Var(X) = \frac{(b - a + 1)^2 - 1}{12} = \frac{(1 - (-1) +1)^2 -1}{12} = \frac{8}{12} = \frac{3}{4}
\end{equation}

\begin{equation}
    P(X > 0) = \frac{\text{\# Gunstige}}{\text{\# Mulige}} = \frac{1}{3} 
\end{equation}

\textbf{Del 2) Find $\E(X \mid X >0)$}

\begin{equation}
    \E( X \mid X > 0) = 1
\end{equation}

klart da $X \mid X > 0$ kun kan antage værdien 1.

\textbf{Del 3) Find $\E(Y)$ og $\Var(Y)$ samt $P(Y>0)$}

\begin{equation}
    \E(Y) = (a + b)/2 = (-1 + 1)/2 = 0
\end{equation}

\begin{equation}
    \Var(Y) = \frac{(b - a)^2}{12} = \frac{((1 - (-1) )^2}{12} = \frac{4}{12} = \frac{1}{3}
\end{equation}

\begin{equation}
    P(Y > 0) = \int_{0}^{1} \frac{1}{2}\1_{(-1,1)}(x) dx = \frac{1}{2}\lp (1) - (0) \rp = \frac{1}{2} 
\end{equation}

\textbf{Del 4) Find $\E(Y \mid Y >0)$}

husk $P(X,Y) = P(X \mid Y) P(Y)$ - hvor $X,Y$ er arbitrære navne for at illustrere matematikken.

\begin{align}
    \E(Y \mid Y>0) &= \int_{-\infty}^{\infty} y \frac{P(Y, Y>0)}{P(Y > 0)} dy\\&= \int_{-\infty}^{\infty} y 
    \frac{\frac{1}{2}\1_{(-1 , 1)}(y) \1_{(0,1)}(y)}{\frac{1}{2}} dy \\
    &= \int_0^1
    y \1_{(0,1)}(y)\\
    &= \lsp \frac{1}{2} y^2 \rsp_0^{1} = \frac{1}{2}
\end{align}
