\horizline

\subsection{Øvelse 13}

\textbf{28/10/2018, opgaver: 44.1.1, 44.1.2, 44.1.3 44.1.4}

\subsubsection{Opgave 44.1.1}

\begin{itemize}
    \item to terninger (stokastiske variable) $X_1, X_2$ 
    \item $(X_1, X_2) \in \lcp 1, 2, 3, 4, 5, 6\rcp^2 = \{x_{1,i}\} \times \{x_{2,j}\}, \quad i,j \in \lcp 1, 2, 3, 4, 5, 6\rcp$
    \item $Z = X_1 + X_2$
\end{itemize}

Den vigtige regel:

\begin{equation}
    p_x(x) =  \int_{y} p_{x, y}(x,y) dy= \int_{y} p_{x\mid y}(x \mid y)p_y(y) dy 
\end{equation}

\textbf{Find $P(X_1 = i \mid Z \geq 4)$}

Vi noterer først vi ikke har kontinuerte stokastiske variable!

Man får en god idé

\begin{equation}
    P(X_1 = i \mid Z \geq 4) = \frac{P(X_1 = i , Z \geq 4)}{P(Z\geq 4)}
\end{equation}

(skits summen af to terninger på tavlen) 

Vi indser hurtigt at $P(Z \geq 4) = \frac{33}{36}$

Vi indser også at:

\begin{equation}
    P(X_1 = i \mid Z \geq 4) = \frac{P(X_1 = i , X_1 + X_2 \geq 4)}{33/36}  = \frac{P(X_1 = i ,  X_2 \geq 4 - i)}{33/36}
\end{equation}

Vi indser at $i$ er en konstant og vi nu har uafhængighed i den simultane sandsynlighed således at:

\begin{equation}
    P(X_1 = i)P(X_2 \geq 4 -i )
\end{equation}

Husk at $P(X_1 = i) = \frac{1}{6}$

Vi kan opskrive det hele i et samlet udtryk:

\begin{equation}
        P(X_1 = i)P(X_2 \geq 4 -i ) = \frac{1}{6}P(X_2 \geq 4 -i )
\end{equation}

\begin{equation}
    P(X_1 = i)P(X_2 \geq 4 -i ) = \frac{1}{6}\cdot \frac{4 - 1 + i}{6} \quad i<4
\end{equation}
    
over i siger man bare $1/6 \times 1/6$
    
\begin{equation}
    P(X_1 = i)P(X_2 \geq 4 -i ) = \frac{3+ i}{36} \quad i <4
\end{equation}

Vi husker at dele med $33/36$ 


\begin{equation}
    P(X_1 = i \mid Z \geq 4)=
        \begin{cases}
            4/33 &i=1 \\
            5/33 &i=2 \\
            6/33 &i\geq3
        \end{cases}
\end{equation}

\textbf{Find $\E(X_1 = i  \mid Z \geq 4)$}

vi ved at $P(X_1 =i )= \frac{1}{36}$. VI kan derfor sige:

\begin{equation}
    \sum_{i=1}^6 i \cdot \min\lp \frac{3+i}{33}, \frac{6}{33} \rp = \frac{4\cdot 1 + 5 \cdot 2 + (3 + 4 + 5 +6)\cdot 6}{33}
\end{equation}

\subsubsection{Opgave 44.1.2}

\begin{itemize}
    \item $Z \in \{1, 2\}$ angiver kommune
    \item $V \in \{0, 1\}$ angiver om man er velhavende
    \item $P(V=1 \mid Z=1) = 0.8 = 1- P(V=0 \mid Z = 0)$
    \item $P(V=1 \mid Z=2) = 0.1$
\end{itemize}

\textbf{Udregn $\E(V \mid Z=1)$ og $\E(V \mid Z =2$}

Udtrykkene er udtryk for sandsynligheden for at være velhavende betinget på hvilken kommune man kommer fra.

\begin{equation}
    \E(V \mid Z=1) =  0 \cdot P(V=0 \mid Z = 0) + 1 \cdot (V=1 \mid Z=1) = 0.2 \cdot 0 + 0.8 \cdot 1 = 0.8
\end{equation}

For kommune 2:

\begin{equation}
    \E(V \mid Z = 2) = 1 \cdot P(V=1 \mid Z=2) +  0\cdot  P(V=0 \mid Z=2) = 0.1 
\end{equation}

\textbf{Vis udtrykket:}

\begin{equation}
    \E(V \mid Z=z) = f(z) = 0.8 \cdot \1(z=1) + 0.2 \cdot \1(Z=2)
\end{equation}

Man ser at hvis $z=1 \implies E(V \mid Z=1) = 0.8$

og omvendt: $z=2 \implies E(V \mid Z=2) = 0.1$

\textbf{Hvad udtrykker $\E(V \mid Z=z)$}

Det betyder at vores forventning er afhængig af realization af $z$.

\textbf{Del 4) }

Man definerer nu den stokastiske variabel \textit{Den betingede middelværdi af V givet Z}.

\begin{equation}
    \E(V \mid Z) = f(z)
\end{equation}

Vis at:

\begin{equation}
    \E(f(z)) = \E(\E(V \mid Z)) = 0.8 P(Z=1) + 0.1 P(Z=2)
\end{equation}

Det følger næsten naturligt:

\begin{equation}
    \E(f(z)) = \E(0.8 \cdot \1(z=1) + 0.1 \cdot \1(z=2))
\end{equation}

Herfra følger det da $V\in \{0,1\}$

\begin{equation}
    \E(f(z)) = 0.8 \cdot P(Z=1) + 0.1 \cdot P(Z=2)
\end{equation}

\subsubsection{Opgave 44.1.3}


\begin{itemize}
    \item $X$ er ligefordelt på $A = [0,10]$
\end{itemize}

\textbf{Del 1) Opskriv tætheden $p(x)$ for $X$ og vis $P(X) > 5 = \frac{1}{2}$}

Tegn tæthedsfunktionen.

Man ved at $F(x) \rightarrow 1$ for $x\rightarrow \infty$. nærmere bestemt ved man at $F(10) = 1$.
Man ved at $\int \1_{A}(x)$ vil være $x$, så man skal gange en konstant på for at få $F(10) = 1$. Hel konkret $10\cdot c = 1 \implies c = 1/10$

\begin{equation}
    p(x) = \frac{1}{10}\1_{A}(x)
\end{equation}

\begin{equation}
    P(X > 5) = \int_{0}^{5} \frac{1}{10}\1_{A}(x) = \frac{1}{10}\int \1_{A}(x) = \frac{1}{10} [x]_{0}^{5}= \frac{1}{10}(5 - 0) = 0.5
\end{equation}

\textbf{Del 2) Find $\E(X)$}

\begin{equation}
    \E(x) = \int_{-\infty}^{\infty} p(x)x
\end{equation}

Vi ved at indikator funktionen kun er defineret i intervallet $[0,10]$. så vi kan skrive:

\begin{align}
    \E(X) &= \int_{0}^{10} \frac{1}{10} x \cdot \1_{A}(x)  \\
    &= \frac{1}{0} \int_{0}^{10} x \\
    &= \frac{1}{10}\lsp\frac{1}{2}x^2\rsp_0^{10} \\
    &= \frac{1}{10}\cdot\frac{1}{2}\cdot10^2 = 5
\end{align}

\textbf{Vis at tætheden for $X\mid X>5$  kan skrive som:}

Skitser det givne på en tegning!

\begin{equation}
    q(x) = \frac{2}{10}\1(5 < x < 10)
\end{equation}

Man indser hurtigt at: $X \in [0, 5] \cap X \in (5,10] = \emptyset$. Vi kan altså herfra konkludere at $X \mid X>5$ kun er defineret på intervallet $(5, 10]$. 

$X \mid X>5$ er stadig uniformt fordelt, og vi kan derfor sige at: $q(x) = c \cdot \1(5 < x \leq 10)$. Igen ved vi også at $Q(10) = 1$. Vi kan hurtige udlede at $c=\frac{1}{5}$. hvormed det ønskede resultat er vist.

\textbf{Del 4) Er $\E(X \mid X>5) = 7.5?$}

Først se på tegningen. Herfra burde det fremgår tydeligt. Mere formelt:

\begin{align}
    \E(X \mid X>5) &= \int_{-\infty}^{\infty}x \cdot \frac{1}{5}\1(5 < x <10) \\
    &= \frac{1}{5}\int_{5}^{10} x \cdot \1(5 < x <10) \\
    &= \frac{1}{5}\lsp \frac{1}{2}x^2 \rsp_{5}^{10} \\
    &= \frac{1}{5}\frac{1}{2}(10^2 - 5^2) \\
    &= \frac{1}{5}\frac{1}{2} \cdot 75 \\
    &= 7.5
\end{align}


\subsubsection{Opgave 44.1.4}

\begin{itemize}
    \item $X$ angiver ratingen fra 0 til 1
    \item $Y$ angiver værdipapirets værdi i 1000 \$
    \item $X, Y$ er ligefordelt på mængden $B$
    \item $B = \{(x,y) \in \R^2 \mid 0 < x < 1, 0.5 + 2x \leq y \leq 2.5 + 2x \}$
\end{itemize}

Lad os starte med at tegne $B$. Kig github!


\textbf{Find tæthedsfunktionen $f_{X, Y}(x,y)$ for den simultane fordeling for $(X,Y)$}

Vi hurtigt indser at de marginale fordelinge må blive 1. Den hurtigste måde at konstanten $c$ på (tænk simultan fordeling $f(x,y) = c \1_{B} (x, y)$). er at finde arealet af $B$.

\begin{equation}
    \frac{1}{c}= h \cdot l = 1 \cdot 2 = 2 \implies c = \frac{1}{2}
\end{equation}

tæthedsfunktionen er:

\begin{equation}
    f_{x,y}(x,y) = \frac{1}{2}\1_{B}(x,y)
\end{equation}

\textbf{Del 2) Find $P(Y>2)$}

Tegn på tegningen hvad det egentlig medfører. Altså på mængden $B$.

Først og fremmest ved vi at vi må integrere $X$ ud af tætheden.

\begin{equation}
    p_y(y) = \int_{\R} \frac{1}{2}\1_{B}(x,y) dx
\end{equation}

Vi lavet et trick og skærer mængden $B$ ud i to mængder $M_1$, $M_2$.

\begin{equation}
    M_1 = \{x, y \mid 0 < x <1, 0.5 + 2x < y < 2.5\}
\end{equation}

\begin{equation}
    M_2 = \{x,y \mid 0 < x < 1, 2.5 < y < 2.5 + 2x\} 
\end{equation}

\begin{equation}
    p_Y (y) = \frac{1}{2}\int_\R \1_{M_1} (x,y) dx + \frac{1}{2}\int_\R \1_{M_2} (x,y) dx 
\end{equation}

Vi håndterer først $M_1$:

Vi ser at vi skal differentiere $x$ ud. Mængden er er altså defineret i $y$-intervallet [0.5, 2.5]. Vi isolerer $x$ som en funktion af $y$:

NOTE: Tegn diagrammet på tavlen og forklar intuitionen!

\begin{equation}
    y = 0.5 + 2x \implies \frac{1}{2}(y - 0.5) = x
\end{equation}

Hvor vi husker at: $y \in [0.5, 2.5]$

Vi kan nu finde at arealet for $M_1:$

\begin{equation}
    \int_{0}^{\frac{1}{2}(y-0.5)} 1 dx = [x]_{0}^{\frac{1}{2}(y-0.5)} = \frac{1}{2}y - 0.25
\end{equation}

Analogt for $M_2$:

(KIG PÅ TAVLESKITSE)
\begin{equation}
    y = 2.5 + 2x \implies \frac{1}{2}(y - 2.5) 
\end{equation}

\begin{equation}
    \int^{1}_{\frac{1}{2}(y - 2.5)} 1 dx = [x]_{\frac{1}{2}(y - 2.5)}^{1} = 1 - \lp \frac{1}{2} y - 1.25\rp = 2.25 - \frac{1}{2}y 
\end{equation}

hvor vi husker at $y \in (2.5 , 4.5]$

Vi opskriver $p_Y (y)$. Man husker at gange konstanten $\frac{1}{2}$ på.

\begin{equation}
    p_Y(y) = 
    \begin{cases}
        \frac{1}{2}\lp 2.25 - \frac{1}{2}y \rp &,y \in (2.5 , 4.5] \\
        \frac{1}{2}\lp \frac{1}{2}y - 0.25 \rp & ,y\in [0.5, 2.5]
    \end{cases}
\end{equation}

Vi kan opskrive $P(Y>2) = 1- P(Y\leq2) = 1- \int_{0.5}^{2} \frac{1}{2}\lp\frac{1}{2}y - 0.25\rp dy$

\begin{align}
   1- \int_{0.5}^{2} \frac{1}{2}\lp\frac{1}{2}y - 0.25\rp dy &= 1 - \frac{1}{4}\int_{0.5}^{2} y - 0.5 dy \\  
   &= 1 - \frac{1}{4} \lsp \frac{1}{2}y^2 - 0.5y \rsp \\
   &= 1 - \frac{1}{4}\lp \lp\frac{1}{2}2^2 - \frac{1}{2} \cdot 2 \rp - \lp \frac{1}{2}0.5^2 - 0.5 \cdot 0.5\rp \rp \\
   &= 1 - \frac{1}{4}( 2 - 1) + \frac{1}{4}\lp \frac{1}{8} - \frac{1}{4} \rp \\
   &= 1 - \frac{1}{4} - \frac{1}{4}\frac{1}{8} \\
   &= 0.71875
\end{align}

\textbf{Del 4) Angiv den betingede fordeling af $X$ givet $Y=1$}

Vi skal finde $p_{X \mid Y=1}(x)$

Vi kan altså bruge vores regel:

\begin{equation}
    p_{X \mid Y} (x) p_{Y}(y) = p(x , y) \implies p_{X \mid Y} = \frac{p(x,y)}{p_Y(y)}
\end{equation}

Vi ved at $Y=1$. Vi bruger dette:

\begin{equation}
    p_{Y}(1) = \frac{1}{2}\lp \frac{1}{2}(1) - 0.25\rp = \frac{1}{4} - \frac{1}{8} = \frac{1}{8}
\end{equation}

Vi indsætter $Y=1$ i den øverste del af brøken. Vi ved vi er i den nederste mængde $M_1$. Dette implicerer:

\begin{equation}
    0.5 + 2x < y \land y = 1 \implies  0.5 +2x < 1 \implies x < \frac{1}{4}
\end{equation}

Vi kan herfra konkludere at:

\begin{equation}
    p_{X\mid Y=1}(x) = \frac{1}{2}\frac{\1_{[0, 0.25]}(x)}{1/8} = 4\cdot \1_{[0, 0.25]}(x)
\end{equation}

\textbf{Del 5) udregn forventede rating når $Y=1$ og når $Y=2$}

Vi kender formlen for forventningen af en ligefordeling: $\E(x) = \frac{a + b}{2}$

Vi har svaret for $E(X \mid Y=1) = \frac{0 + 0.25}{2} = \frac{1}{8}$

Vi skal nu analogt finde den betingede tæthed når $Y=2$

\begin{equation}
    p_Y(2) =  \frac{1}{2}\lp \frac{1}{2}(2) - 0.25\rp = \frac{1}{2} - \frac{1}{8} = \frac{3}{8} 
\end{equation}

Vi ser igen på mængden $M_2:$

\begin{equation}
    0.5 + 2x < y \land y=2 \implies 0.5 + 2x < 2 \implies x < \frac{3}{4}
\end{equation}

Vi kan herfor konkluderer at den betingede fordeling for $X \mid Y=2$ må være:

\begin{equation}
    p_{X \mid Y=2} (x) = \frac{1}{2}\frac{\1_{\lsp0,\frac{3}{4}\rsp}(x)}{\frac{3}{8}} = \frac{4}{3} \cdot \1_{\lsp0,\frac{3}{4}\rsp} 
\end{equation}

Vi finder forventningen som må være:

\begin{equation}
    \E(X\mid Y=2) = \frac{1}{2}\frac{3}{4}=\frac{3}{8}
\end{equation}

\textbf{Del 6) Find variansen $\Var(X \mid Y = 1$ og $\Var(X\mid Y=2)$}

I stedet for at bruge hintet kigger vi på distributionen og bruger regnereglen for ligefordelinger:

\begin{equation}
    Var(X) = \frac{1}{12}(a - b)^2
\end{equation}

\begin{equation}
    \Var( X \mid Y =1) = \frac{1}{12}\lp 0 - \frac{1}{4} \rp^2 = \frac{1}{16}\frac{1}{12} = \frac{1}{192}
\end{equation}

\begin{equation}
    \Var(X \mid Y = 2) = \frac{1}{12} \lp 0 - \frac{3}{4} \rp^2 = \frac{1}{12}\frac{9}{16} = \frac{9}{192}
\end{equation}

\textbf{Hvornår er den betingede varians størst - dvs. variansen af ratingen betinget på prisen}

Kig på tegningen: Det rigtige svar må være $Y=2500$.

Man overvejer følgende:

\begin{equation}
    Var(X) = \frac{1}{12}(a - b)^2
\end{equation}

I intervallet $y \in [0.5, 2.5]$ ved vi at:

\begin{equation}
    Var(X \mid Y= y) = \frac{1}{12}(0 - a)^2
\end{equation}

Hvor at er øvre grænse:

\begin{equation}
    0.5 + 2x < y \implies x = \frac{1}{2}(y - 0.5)
\end{equation}

er monotont stigende med højere i $y$ i intervallet $[0.5, 2.5)$.

Vi kan derfor sige at:

I intervallet $[0.5, 2.5)$ finder vi den højeste varians ved $Y=2.5$. 

Analogt kan man den højeste varians i i intervallet $[2.5 , 4.5]$ til at være $Y=2.5$ 

Illustrer på tavle!


