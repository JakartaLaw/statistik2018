\horizline

\subsection{Øvelse 15}

\textbf{5/11/2018, opgaver: U45.1, U45.2, U45.3, U45.4}

\subsubsection{Opgave U45.1}

\begin{itemize}
    \item $X \sim N(\mu, \sigma^2)$
\end{itemize}

\textbf{Vis at $Y=\frac{1}{\sqrt{\sigma^2}}(X -\mu)$ er standard normal fordelt $N(0,1)$}

Vi kan starte med at opskrive den parametriske form op for tætheden af en normalfordeling med $mu$ og $sigma^2$ som henholdsvis middelværdi og varians

\begin{equation}
    p(x) = \frac{1}{\sqrt{2 \pi \sigma^2}} \exp \lp-\frac{(x- \mu)^2}{2\sigma^2} \rp 
\end{equation}

Vi husker et par regneregler:

Vi ved at $\E(X) = \mu$.

Det vil sige at:

\begin{equation}
    \E[Y] = \E \lsp \frac{1}{\sqrt{\sigma^2}}(X -\mu) \rsp = \frac{1}{\sqrt{\sigma^2}} \E \lsp (X -\mu)  \rsp
\end{equation}

Som sagt: $\E(X) = \mu$.

\begin{equation}
    \frac{1}{\sqrt{\sigma^2}} \E \lsp (X -\mu)  \rsp =\frac{1}{\sqrt{\sigma^2}}( \E[X] - \E[\mu]) =\frac{1}{\sqrt{\sigma^2}}  \lp \mu - \mu \rp = 0 = \E[Y] 
\end{equation}

Vi finder variansen. Man husker at $\Var(aX) = a^2 \Var(X) $

Vi ved at $\Var(X) = \sigma^2$

\begin{equation}
    \Var(Y) = \Var\lp \frac{1}{\sqrt{\sigma^2}}X\rp
\end{equation}

Vi ignorerer $\mu$ da dette er en konstant
\begin{equation}
    \Var\lp \frac{1}{\sqrt{\sigma^2}}X\rp = \frac{1}{\sigma^2}\Var(X) = \frac{1}{\sigma^2}\sigma^2 = 1 = \Var(Y)
\end{equation}

Og vi har nu vidst at $Y\sim N(0,1)$

\textbf{Del 2) $(X,Y)$ er en 2-dimensionel stokastisk vektor med middelværdi $mu$ og covarians matrice = $\Omega$}

\textbf{Vis at $Z = \frac{1}{\sqrt{\sigma_X^2}}(X - \mu_X)$ er standard normalfordelt}

\begin{equation}
    \mu = \lp
    \begin{array}{cc}
         \mu_X  \\
         \mu_Y 
    \end{array} \rp \qquad
    \Omega = \lp
    \begin{array}{cc}
    \sigma^2_Y & \sigma_{XY} \\
    \sigma_{XY} & \sigma^2_X
    \end{array}
    \rp
\end{equation}

Samme argument som før!

\textbf{Lad $(X,Y)$ være som i spørgsmål 2, men med $\mu = (0,0)^T$. Vis at $Z = Y-\beta X$ er normaltfordelt med $N(0,\sigma^2)$}

Hvor $\sigma^2$ er 

\begin{equation}
    \sigma^2 = \sigma^2_Y - \beta \sigma_{XY}
\end{equation} 

og $\beta$ er:

\begin{equation}
    \beta = \sigma_{X,Y} / \sigma^2_X
\end{equation}

Vi husker at summen af to normalfordelte stokastiske variable er normalfordelt. Dvs: $Y - \beta X$ nødvendigvis må være normalfordelt!

Vi finder middelværdien først:

\begin{equation}
    \E[Z] = \E[Y - \beta X] = \E[Y] - \beta \E[X] = 0 - \beta\cdot 0 = 0
\end{equation}

Nu finder vi variansen:

\begin{equation}
    \Var(Z) = \Var(Y - \beta Z)
\end{equation}

Vi husker der er covarians mellem $X$ og $Y$.

Det vil sige:

\begin{align}
    \Var(Z) 
    &= \Var(Y - \beta x) \\
    &= \Var(Y) + \beta^2 \Var(X)  + 2\Cov(Y, - \beta X) \\
    &= \Var(Y) + \beta^2 \Var(X)  - 2\beta \Cov(Y, X) \\
    &= \sigma^2_Y + \beta^2 \sigma^2_X - 2\beta \sigma_{X,Y} \\
    &= \sigma^2_Y + (\sigma_{X,Y} / \sigma^2_X)^2 \sigma^2_X - 2(\sigma_{X,Y} / \sigma^2_X) \sigma_{X,Y}\\
    &= \sigma^2_Y + \frac{\sigma_{X,Y}^2}{\sigma^2_X} - 2\frac{\sigma_{X,Y}^2}{\sigma^2_X} \\
    &= \sigma^2_Y - \frac{\sigma_{X,Y}^2}{\sigma^2_X}
\end{align}

Hvilket var det vi ønskede at vise:

\begin{equation}
    Z \sim N(0, \sigma^2) = N \lp 0,\sigma^2_Y - \frac{\sigma_{X,Y}^2}{\sigma^2_X} \rp
\end{equation}

\textbf{Del 4)}

Vis at:

\begin{equation}
    \E((Y - \beta X)X) = 0
\end{equation}

Hvilket betyder at $Z$ og $X$ er uafhængige!

\begin{align}
    \E((Y - \beta X)X) &=
    \E(YX - \beta X^2) \\
    &= \E(YX) - \beta\E(X^2) \\
\end{align}

Her ser man at $\E(X) = \E(Y) =0 \implies \E(XY) = \sigma_{XY}$ og at $\E(X) = 0 \implies \E(X^2) = \sigma^2_X$

\begin{align}
    \E(YX) - \beta\E(X^2) &= \sigma_XY - \beta \sigma_X^2 \\
    &= \sigma_{XY} - (\sigma_{X,Y} / \sigma^2_X)\sigma_X^2 \\
    &= 0
\end{align}

For at afgøre om de er uafhængige definerer vi først fejlleddet $\epsilon$:

\begin{equation}
    \epsilon = Y - \beta X
\end{equation}

\begin{align}
    \Cov(\epsilon, X) = \E(\epsilon X ) - \E(\epsilon) \E(X) = \E(\epsilon X ) = 0
\end{align}

Det betyder at $\epsilon$ er ukorreleret med $X$. Og da både $X$ og $\epsilon$ er normalfordelte kan vi konkludere de er uafhængige!

\subsubsection{Opgave U45.2}

Laves i klassen!

\begin{itemize}
    \item stokastisk vektor $(X,Y)$
    \item distribueret med $N(m, \Omega)$
\end{itemize}

\begin{equation}
    m = \lp 
    \begin{array}{cc}
         1  \\
         0 
    \end{array} \rp
    \qquad \Omega = \lp
    \begin{array}{cc}
         1 & \rho  \\
         \rho & 1
    \end{array} \rp
\end{equation}

\textbf{Del 1) Hvad er $\E(X)$ og $\Var(X)$}

Vi kan direkte aflæse svarene $\E(X) = 1$ og $\Var(X) = 1$.

\textbf{Del 2) Hvordan er $Y$ fordelt:}

Aflæses i $m$ og $\Omega$   

\begin{equation}
    Y \sim N(0,1)
\end{equation}

\textbf{Del 3) Hvad er $\Cov(X,Y)$}

Dette kan aflæses i kovarians-matricen off-diagonal elementer: $\sigma_{X Y} =  \rho$

\textbf{Del 4) Hvad er $\E(Y \mid X = x)$}

Fra Rahbeks note (property G.3) finder vi formlen:

\begin{equation}
    \E(Y \mid X =x ) = \mu_{Y \mid X} = \mu_Y + \omega (x - \mu_X)
\end{equation}

Hvor $\omega = \sigma_{YX}/\sigma_X^2$

Hvilket implicerer at:

\begin{equation}
    \E(Y \mid X =x ) = \mu_{Y \mid X} = \mu_Y + (\sigma_{YX}/\sigma_X^2)(x - \mu_X) 
\end{equation}

Vi finder de passende værdier i kovarians-matricen: $\mu_X = 1$, $\mu_Y = 0$, $\sigma_X^2=1$ og $\sigma_{XY} = \rho$

\begin{equation}
    \E(Y \mid X = x) = 0 + \lp  \frac{\rho}{1}\rp(x - 1) = \rho(x -1)
\end{equation}

\textbf{Del 5) Hvad er $\E(X \mid Y = y)$}

Vi bruger samme formel som før, bare hvor: $\omega = \sigma_{YX}/\sigma_Y^2$

\begin{equation}
    \E(X \mid Y = y) = \mu_x + \omega(y - \mu_Y) = 1 + \rho y
\end{equation}

\textbf{Del 6) Hvad er $\Var(Y\mid X= x)$}

Vi kigger igen i Rahbeks note (property G.3).

Finder formlen:

\begin{equation}
    \Var(Y\mid X = x) = \sigma_Y^2 - \omega \sigma_{XY} = \sigma_Y^2 - \frac{\sigma_{XY}^2}{\sigma_X^2}
\end{equation}

Og vi har stadig $\omega = \sigma_{YX}/\sigma_Y^2$. Altså samme $\omega$ som i del 5.

Vi har de passende værdier:
$\sigma_X^2 = 1$, $\sigma_Y^2 =1$ og $\sigma_{XY} = \rho$

\begin{equation}
    \Var(Y\mid X = x) = \sigma_Y^2 - \frac{\sigma_{XY}^2}{\sigma_X^2} = 1 - \frac{\rho^2}{1} = 1 -\rho^2
\end{equation}

\textbf{Del 7) Hvad gælder for $(X,Y)$ hvis $\rho = 0.9$ og $\rho=0$}

Hvis $\rho=0.9$ er $X,Y$ stærkt positivt korrelerede. Det vil sige en høj realisering af $Y$ implicerer en høj realisering $X$.

Omvendt $\rho=0$ implicerer $X$ og $Y$ er ukorrelerede! Da de begge er normalt fordelte er de nødvendigvis også uafhængige!

\subsubsection{Opgave U45.3}

\begin{itemize}
    \item $(X, Y)$ er stokastisk vektor med som er normalfordelt med $N(\mu, \Omega)$
\end{itemize}

\begin{equation}
    \mu = \lp
    \begin{array}{cc}
         0  \\
         2 
    \end{array}\rp \qquad
    \Omega = \lp
    \begin{array}{cc}
        1 & 0.5 \\
        0.5 & 1
    \end{array} \rp
\end{equation}

\textbf{Opskriv tætheden $p(x,y)$}

Kig formlen på s.237 Sørensen (ligning 8.3.6)

\begin{equation}
    p(x,y) = \frac{1}{2\pi}\frac{1}{\sqrt{\det(\Omega)}}\exp\lp-\frac{1}{2}(x - \mu_X, y - \mu_Y)\Omega^{-1}
    \lp
    \begin{array}{cc}
         x - \mu_X  \\
         y - \mu_Y 
    \end{array} 
    \rp
    \rp
\end{equation}

Hvis det ser uklart ud, da kan vi hurtigt lige definerer vektoren $K = (x - \mu_X, y - \mu_Y)$

Hvilket forsimpler utrykket til (Gøres for klargøre at det er en vektor $K$ man "opløfter i 2"): 

\begin{equation}
    p(x,y) = \frac{1}{2\pi}\frac{1}{\sqrt{\det(\Omega)}}\exp\lp-\frac{1}{2}K \Omega^{-1} K^T    \rp
\end{equation}

Vi finder de to centrale ting:

\begin{equation}
    \det(\Omega) = 1^2 - 0.5^2 = 0.75
\end{equation}


Kig i Sørensen s.237
\begin{equation}
    \Omega^{-1} = \frac{1}{1 -\rho^2} \lp
    \begin{array}{cc}
        1 & -0.5  \\
        -0.5 & 1
    \end{array}    
    \rp = 
    \lp
    \begin{array}{cc}
         \frac{4}{3} & -\frac{2}{3} \\
         -\frac{2}{3} & \frac{4}{3}
    \end{array} \rp
\end{equation}

Og vi har fundet tætheden!

\subsubsection{Opgave U45.4}

\begin{itemize}
    \item $Y$ er afkast på amerikansk aktie (Microsoft)
    \item $X$ er et aktie-indeks (SP500)
    \item $Y := \beta X + \epsilon$
    \item $\epsilon \sim N(0,\sigma^2)$
\end{itemize}

\textbf{Del 1) Fortolkning af $\beta$}

$\beta$ er relateret til kovariansen mellem $X$ og $Y$ og angiver altså samvariansen mellem en given aktie og hvordan hele markedet bevæger sig. En aktie med negativ $\beta$ vil altså kunne reducerer volatiliteten i en portfølje da den er modsat korreleret med de andre aktier.

\begin{equation}
    \beta = \frac{\sigma_{XY}}{\sigma^2_X}
\end{equation}

Står beskrevet i Rahbeks Note s. 10

\textbf{Del 2) En anden model blev repræsenteret}

\begin{equation}
    Y := \epsilon_Y, \qquad X := \epsilon_X
\end{equation}

hvor $\epsilon_Y \sim N(0, \sigma_Y^2)$ og $\epsilon_X \sim N(0, \sigma_X^2)$

\textbf{Forklar hvordan denne kan stemme overens med modellen præsenteret ovenfor}

\begin{equation}
    \E(Y) = 0
\end{equation}

Modellen ovenfor var implicit en betinget model for $Y$:

\begin{align}
    \E(Y \mid X) &= \E( \beta X + \epsilon \mid X) \\
    &= \beta X
\end{align}

Da $\E(\epsilon \mid X) = 0$

Altså så før så vi på en betinget model, men nu er det to marginale modeller, som er opstillet!

Man bruger altså i den betingede model information om hvordan en aktie samvarierer med markedet

